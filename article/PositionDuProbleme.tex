% contient une exposition du problème (description d'une homographie, source du problème, intérêt du problème, problématique, raison pour laquelle on se bat dans la vie, etc...)




	Nous nous intéressons dans cet article à l'élaboration d'une méthode la plus exacte possible permettant de déformer une image par une homographie. \\
 	Une homographie est une déformation du plan qui transforme les droites en droites ; elle est déterminée par 8 paramètres et s'écrit sous la forme :
	\[H : (x,y)\mapsto\left( \frac{ax+by+p}{rx+sy+t},\frac{cx+dy+q}{rx+sy+t}\right)\] 
On peut voir l'effet d'une homographie figure \ref{effethom}.\\

 \begin{figure}
 
   \centering
   \subfigPDP{I(x,y)}{hom_before.png}
    \arrowPDP 
   \subfigPDP{I(H(x,y))}{hom_after.png}
   \caption{Effet d'une homographie}
\label{effethom}
 \end{figure}

	Les homographies correspondent en fait à des changements de perspective. Elles ont donc de nombreuses applications en traitement d'image comme par exemple le \emph{texturing} (dans le jeux vidéo, l'animation, etc...), dans la constitution de panorama ou encore le recalage précis d'une suite d'images prises du même endroit (ce qui est la première étape de plusieurs méthodes de super-résolution).


