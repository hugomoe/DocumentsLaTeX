PARTOUT
	référer les figure
	les reference peuvent disparaitre sans gêner la phrase
	mettre des captions et des labels sur toutes les figures
	remplacer tous les "homographie particulière" par "homographie unidirectionnelle"
	mettre des phrases dans les propositions et redéfinir les termes
	proposition de shmuel : mettre l'échelle dans les captions (pour connaitre la taille réelle)
	article mathématique : pas de texte libre (disons presque pas...?)
		soit : notation, remarque
	mettre le moins de texte libre possible
	la description d'un algo, si elle est courte, doit être dans le \caption







HOMOGRAPHIE (? ailleurs qu'ici)
	c’est un mvt d’une caméra idéale



AFFINITE
	formule convolution discrète avec zoom (reference pseudo-code supp adaptatif)


DECOMP GEOM 
	-mettre convergence en espace ? (L2 et uniformément).
	-page 26 : dire n=4


MVT DE CAMERA
	Prop 2 = revoir les notations (h ‘est’ un mvet de cam -> h une appli de R^2, h s'identifie à)


HOMO PARTICULIERE (-> HOMO UNIDIRECTIONNELLE)
	Rentrer « on en déduit » de fin de partie precedente, on le met là
	inclure image qui montre la séparation (damier par exemple), chacun en partant de l'image originale
	Lemme pour la convergence des convolées



EXPERIENCES
	comparaison toute méthode
	mettre en comparaison la méthode naïve
