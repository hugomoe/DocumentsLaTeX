The new method of homographies resampling is based on the decomposition of an homographie which can be seen as camera moves. The section below presents that decompostion.


%La nouvelle méthode de traitement des homographies repose sur la décomposition d'une homographie qui permet d'interpréter cette dernière en terme de mouvement de caméra. La partie ci-dessous présente cette décomposition.

\ssse{Modelisation of a camera mouvement}
%\ssse{Modélisation de mouvement de caméra}
\label{mouv_de_camera}
On étudie ici un cas a priori particulier d'homographie $h$ : les homographies que l'on peut interpréter comme un mouvement de caméra idéale. On montrera par la suite que c'est en fait un cas général.

Nous modéliserons la situation en supposant que la scène filmée est plane. Cela suppose que l'on filme une surface soit sans aucun relief, soit avec un relief négligeable devant la distance à la caméra, afin qu'il ne soit pas perceptible. La figure \ref{shmdecomp} illustre la modélisation utilisée pour la caméra idéale. La caméra idéale se modélise donc par la projection d'un plan sur un autre en passant par un foyer $F$, en négligeant les lentilles ou les dispositifs correcteurs présents dans les caméras réelles.

\begin{figure}[h!]

\centering
\includegraphics[width=10cm]{shema_decomp.png}
\caption{Illustration d'un mouvement de caméra $(X_v =0)$ (les translations ont été omises pour plus de clarté. $F$ représente le point focal de la caméra, le plan rouge est le plan image de la caméra. Un point du plan image est le projeté passant par $F$ d'un point du plan $(x,y)$. (cf. partie \ref{mouv_de_camera})}
\label{shmdecomp}
\end{figure}


On se place dans l'espace vectoriel $\mathbb{R}^{3}$ on note $O$ l'origine $(0,0,0)$ et $(\xbf_0,\ybf_0,\zbf_0)$ la base canonique.
\begin{itemize}
\item Soient $F$ et $C_0$ deux points distincts de  $\mathbb{R}^{3}$.
\item Soit $\mathcal{P}$ le plan affine passant par $C_0$ de vecteur normal $\overrightarrow{FC_0}$.
\item On note $\wbf$ le vecteur $\frac{\overrightarrow{FC_0}}{|| \overrightarrow{FC_0}||}$
\item On note $\delta$ la distance  $|| \overrightarrow{FC_0}||$
\end{itemize}
\begin{Def}
L'application projective $H$, est l'application qui à un point $X$ de $\mathbb{R}^{3}$ associe le point d'intersection entre la droite $(XF)$ et le plan $\mathcal{P}$. $H$ dépend du triplet $(F,\wbf,\delta)$.
\end{Def}
\begin{remarques}
\begin{itemize}
\item $F$ est la position de l'objectif de la caméra et le plan $\mathcal{P}$ est l'écran de la caméra sur lequel est projetée l'image. L'axe optique de la caméra est la droite $(FC_0)$ de vecteur directeur $\wbf$, $\delta$ est la distance entre l'objectif et l'écran.
\item A un point de $\mathbb{R}^3$  l'application $H$ associe le  point de $\mathcal{P}$ correspondant à son image a travers la caméra.
\end{itemize}
\end{remarques}
Soit $\mathcal{P}'$ le plan affine de $\mathbb{R}^{3}$ passant par $F$ et parallèle au plan $\mathcal{P}$.
\begin{lem}
$H$ est définit sur $\mathbb{R}^3 \setminus \mathcal{P}'$ et on a  
\begin{equation}
H(X) = C_0 +  \delta \frac{\overrightarrow{XF}-(\overrightarrow{XF}\cdot \wbf )\wbf}{\wbf \cdot \overrightarrow{XF}} 
\label{formul_lem_app_proj}
\end{equation}
\label{lem_app_proj}
\end{lem}
\begin{proof}
Si $X\in \mathbb{R}^3 \setminus \{F\}$ la droite $(XF)$ et admet un point d'intersection avec $\mathcal{P}$ si et seulement elle n'est pas parallèle à $\mathcal{P}$ donc si et seulement si $X\in \mathbb{R}^3 \setminus \mathcal{P}'$. Dans ce cas il existe $t_X\in \mathcal{R}$ tel que 
\begin{equation*}
H(X)=X+t_{X}\overrightarrow{XF}.
\end{equation*}
Comme $H(X)\in P_{2}$ alors
\begin{equation*}
\overrightarrow{FH(X)}\cdot \wbf =\delta.
\end{equation*}
En réinjectant l'expression de $H(X)$ on obtient
\begin{equation*}
t_{X}=1+\frac{\delta}{\wbf \cdot \overrightarrow{XF}},
\end{equation*}
 On en déduit que 
\begin{equation*}
H(X) = C_0 +  \delta \frac{\overrightarrow{XF}-(\overrightarrow{XF}\cdot \wbf )\wbf}{\wbf \cdot \overrightarrow{XF}}.
\end{equation*}
\end{proof}
\begin{remarque}
Le plan $\mathcal{P}'$ correspond à un angle mort de la caméra, dans le cas d'une caméra réelle cet angle mort devrait occuper au moins un demi espace 
\end{remarque}
On note $P$ le plan $(O,\xbf_0,\ybf_0)$.
\begin{Def} On appelle point visé le point d'intersection entre la droite $(FC_0)$ et le plan $P$ lorsqu'il existe. Le point visé existe si et seulement si $(FC_0)$ n'est pas parallèle à  $\mathcal{P}$. Lorsqu'il existe on le note $X_v$ et on a
\begin{equation*}
X_v=F-\wbf \delta'~~~~~~\delta'=\frac{\overrightarrow{OF}\cdot \zbf_0}{\wbf \cdot \zbf_0}
\label{formule_point_vise}
\end{equation*}
\label{point_vise}
\end{Def}
\begin{remarque}
Le point visé est le point de $P$ visé par la caméra, les mouvements de la caméra se font autour de ce point. Il est possible qu'une caméra n'ait pas de point visé dans se cas il est situé à l'infini, la caméra vise l'horizon. 
\end{remarque}
On suppose dans la suite que le point visé existe.
\begin{Def}
L'application projective planaire $H^*$ associée à $H$ est la restriction de $H$ à $P\setminus (\mathcal{P}'\cap P)$ 
Si $\wbf \perp P $ alors $H^*$ est définit sur $P$ sinon $H^*$ est définit sur $P\setminus D$ où $D$ est la droite
\begin{equation*}
D=\left\{ X\in P | \overrightarrow{XF}\cdot \wbf = 0\right\}
\end{equation*}
\end{Def}
\begin{remarque}
\begin{itemize}
\item Si on suppose que la scène filmée est suffisamment plane pour que l'on puisse  négliger tout relief on peut l'assimiler au plan $P$ . Alors à un point $X\in P$  de la scène filmée, l'application $H^*$ associe le point de $P_2$ correspondant à son image à travers la caméra.
\item La droite $D'=\{ X \in \mathcal{P} | \overrightarrow{XF} \cdot \zbf_0 =0 \}$ est appelé l'horizon de $H^*$.
\end{itemize}

\end{remarque}
On peut munir le plan affine $\mathcal{P}$ d'un repère orthonormé direct $(C,\ubf,\vbf)$ 
\begin{Def}
 L'homographie associée à $H^*$ dans le repère $(C,\ubf,\vbf)$ est l'application
\begin{equation}
h : (x,y)  \mapsto \left( \overrightarrow{CH^*(X)}\cdot \ubf , \overrightarrow{CH^*(X)}\cdot \vbf \right)
\label{formule_homographie_H}
\end{equation}


où $X=x~\xbf_0 + y~\ybf_0 $.
\label{def_homographie_H}
\end{Def}
Si on note encore $D,D'$ les droites de $\mathbb{R}^2$ obtenue lorsque $P$ et $\mathcal{P}$ sont munit des repère $(O,\xbf,\ybf)$ et $(C,\ubf,\vbf)$ alors $h:\mathbb{R}^2  \setminus D \mapsto \mathbb{R}^2  \setminus D'$ est une bijection.
\begin{remarque}
A un point $X\in P$ de coordonnées $(x,y)$  l'application $h$ associe les coordonnées  du point $H^*(X)$ dans l'image numérique obtenue après acquisition par la caméra. 
\end{remarque}
L'orientation de la base $(\ubf,\vbf,\wbf)$ par rapport à la base $(\xbf_0,\ybf_0,\zbf_0)$ peut être définie grâce aux trois angles d'Euler $(\phi , \theta ,\psi )$ (figure \ref{img_angles})
\begin{itemize}
\item $\phi$ est la précession autour de l'axe $(X_v ,\zbf_0)$
\item $\theta$ est l'inclinaison par rapport à l'axe $(X_v , \zbf_0 )$
\item $\psi$ est la rotation propre autour de l'axe $(X_v , \wbf )$
\end{itemize}
On note $\cbf= \left (\overrightarrow{C_0 C}\cdot \ubf , \overrightarrow{C_0 C}\cdot \vbf \right)$ et $\xbf_v=\left( \overrightarrow{O X_v}\cdot \ubf , \overrightarrow{O X_v}\cdot \vbf \right )$.\\
\begin{prop}
 L'homographie  $h$ se factorise sous la forme
 
\begin{equation}
h = \tau_{\cbf}   \circ R_{\psi} \circ z_{\frac{\delta}{\delta'}} \circ h_{\theta,\delta'} \circ R_{\phi} \circ \tau_{\xbf_v}
\label{formul_decomp}
\end{equation}
Où $R_{\alpha}$ est la rotation d'angle $\alpha$,$z_\lambda$ est la dilatation de facteur $\lambda$, $\tau_\ybf$ est  la translation du vecteur $-\ybf$ et $h_{\theta,\delta'}$ est l'homographie unidirectionnelle (cf. définition \ref{homo_uni_direc})
\begin{equation}
h_{\theta,\delta'}(x,y)=\left(\frac{-cos(\theta)x}{1-\frac{sin(\theta)}{\delta'}x} ,\frac{-y}{1-\frac{sin(\theta)}{\delta'}x}\right)
\label{mise_perspective}
\end{equation}
\label{prop_decomp}
\end{prop}
\begin{Def}
Une homographie unidirectionnelle est une application $h:\mathbb{R}^{2} \ra \mathbb{R}^{2}$ définie par 
\begin{equation*}
h(x,y)=\left ( \frac{ax+p}{rx+t} , \frac{cy+p}{rx+t} \right)
\end{equation*}
Où $a,p,c,q,r,t$ sont des réels.
\label{homo_uni_direc}
\end{Def}
\begin{figure}[h!]
\centering
\subfigure[rotation d'angle $\phi$]{\includegraphics[width=5cm]{graphe1.jpg}}
\subfigure[rotation d'angle $\theta$]{\includegraphics[width=5cm]{graphe2.jpg}}
\caption{(cf partie \ref{figure_de_rotations_18})}
\label{img_angles}
\end{figure}

\begin{figure}[h!]
\centering
\includegraphics[width=5cm]{graphe3.jpg}
\caption{rotation propre (cf partie \ref{figure_de_rotations_18})}
\label{decompgeo_rotationPropre}
\end{figure}





\begin{proof}
Using formula (\ref{formule_homographie_H}), definition (\ref{def_homographie_H}) and the formula (\ref{formul_lem_app_proj}) of the lemma (\ref{lem_app_proj}), we have
\begin{equation*}
h((x,y))=\left( \delta \frac{\overrightarrow{XF}\cdot \ubf}{\overrightarrow{XF} \cdot \wbf} +\overrightarrow{CC_0 } \cdot \ubf, \delta \frac{\overrightarrow{XF}\cdot \vbf}{\overrightarrow{XF} \cdot  \wbf}+\overrightarrow{CC_0 }\cdot \vbf \right)
\end{equation*}
where $X=x \xbf_0 + y \ybf_0 $.\\

By introducing the target point $X_v$ (definition \ref{point_vise} formula \ref{formule_point_vise}) and the translation $\tau_c$ we have
\begin{equation}
(\tau_\cbf^{-1} \circ h)((x,y)) = \left ( -\delta \frac{\overrightarrow{X_vX}\cdot \ubf}{\delta' -\overrightarrow{X_vX} \cdot \wbf},-\delta \frac{\overrightarrow{X_vX}\cdot \vbf}{\delta' -\overrightarrow{X_vX} \cdot \wbf} \right)
\label{decomp_formul_intermediaire_1}
\end{equation}


Intermediate basis $(\xbf_1 ,\ybf_1 ,\zbf_1)$  $(\xbf_2 ,\ybf_2 ,\zbf_2)$ can be defined, in order to decompose the three rotations $\phi,\theta,\psi$ (figures \ref{img_angles} and \ref{decompgeo_rotationPropre} ).

Then we have the relations
\begin{equation*}
\ubf=\cos(\psi)\xbf_{2}+\sin(\psi)\ybf_{2} , \vbf=-\sin(\psi)\xbf_{2}+\cos(\psi)\ybf_{2} \text{ et } \wbf=\zbf_2.
\end{equation*}

Denoting $R_{s}$ the rotation of angle $s$, we obtained, by using the formula (\ref{decomp_formul_intermediaire_1}) 
\begin{equation*}
(\tau_{\cbf}^{-1} \circ h)((x,y)) = R_{\psi}\left(\frac{\delta \overrightarrow{X_v X}\cdot \xbf_{2} }{\delta'-\zbf_2 \cdot \overrightarrow{X_v X}},\frac{\delta \overrightarrow{X_v X}\cdot \ybf_{2}}{\delta'-\zbf_2 \cdot \overrightarrow{X_v X}}  \right).
\end{equation*}
Let $i$ be the mapping such that $i(x,y)=x \xbf_0 + y \ybf_0$. Since $i(\xbf_v)=\overrightarrow{O X_v}$ we have
\begin{equation*}
(R_{\psi}^{-1} \circ \tau_{\cbf}^{-1}  \circ h)((x,y))=\delta \left(\frac{-i(\tau_{\xbf_v} ((x,y)))\cdot \xbf_{2} }{\delta'-\zbf_2 \cdot i(\tau_{\xbf_v} ((x,y)))},\frac{-i(\tau_{\xbf_v} ((x,y)))\cdot \ybf_{2}}{\delta'-\zbf_2 \cdot i(\tau_{\xbf_v} ((x,y)))}  \right) 
\end{equation*}

Since $\zbf_{2}=cos(\theta)\zbf_{1}+sin(\theta)\xbf_{1}$, $\xbf_{2}=cos(\theta)\xbf_{1}-sin(\theta)\zbf_{1}$ (figure \ref{img_angles}) and $\zbf_{1}\perp P_{1}$, we have

\begin{equation*}
(R_{\psi}^{-1} \circ \tau_{\cbf}^{-1}  \circ h)((x,y))=\frac{\delta}{\delta'}\left(\frac{-\cos(\theta)i(\tau_{\xbf_v} ((x,y)))\cdot \xbf_{1} }{1-\frac{sin(\theta)}{\delta'}\xbf_{1}\cdot i(\tau_{\xbf_v}((x,y)))}, \frac{-i(\tau_{\xbf_v} ((x,y)))\cdot \ybf_{1}}{1-\frac{sin(\theta)}{\delta'}\xbf_{1}\cdot i(\tau_{\xbf_v}((x,y)))}  \right) 
\end{equation*}

Let $h_{\theta,\delta'}$ be defined by

\begin{equation*}
h_{\theta,\delta'}(x',y')=\left(\frac{-\cos(\theta)x'}{1-\frac{\sin(\theta)}{\delta'}x'} ,\frac{-y'}{1-\frac{\sin(\theta)}{\delta'}x'}\right)
\end{equation*}

Then

\begin{equation*}
(R_{\psi}^{-1} \circ \tau_{\cbf}^{-1} \circ h)((x,y))= \frac{\delta}{\delta'}h_{\theta,\delta'}\left ( i(\tau_{\xbf_v}((x,y))) \cdot \xbf_{1}, i(\tau_{\xbf_v}((x,y))) \cdot \ybf_{1}\right)
\end{equation*}
\label{figure_de_rotations_18}
Since $\xbf_{1}=\cos(\phi)\xbf_{0}+\sin(\phi)\ybf_{0}$ et $\ybf_{1}=-\sin(\phi)\xbf_{0}+\cos(\phi)\ybf_{0}$ (figure \ref{img_angles}), we have

\begin{eqnarray*}
(R_{\psi}^{-1} \circ \tau_{\cbf}^{-1} \circ h)((x,y)) &=& \frac{\delta}{\delta'}h_{\theta,\delta'}\left ( R_{\phi}(i(\tau_{\xbf_v}((x,y))) \cdot \xbf_{0}, i(\tau_{\xbf_v}((x,y))) \cdot \ybf_{0})\right)\\
                                               &=&\frac{\delta}{\delta'} (h_{\theta,\delta'}\circ R_{\phi} \circ \tau_{\xbf_v})((x,y))
\end{eqnarray*}

Denoting zooms $z_{\lambda}:X\rightarrow \lambda X$, the claim is verified since

\begin{equation*}
h = \tau_{\cbf} \circ R_{\psi} \circ z_{\frac{\delta}{\delta'}} \circ h_{\theta,\delta'} \circ R_{\phi} \circ \tau_{\xbf_v}
\end{equation*}

\end{proof}


\begin{remarques}
\begin{itemize}
\item Resampling an image by the homography $h$ simulate a change of the point of view on the input image. This has parameters $(\phi,\theta,\psi,\delta,\delta',\xbf_v,\cbf_v)$, which are not independent.
\item The case in which the camera is targeting the horizon has not been discussed, translations $\tau_\cbf$ allow to avoid this.
\end{itemize}
\end{remarques}



\begin{remarque}
The previous method does not modeled every affine transform. Indeed the function $h$ defined by 
\begin{equation*}
h = \tau_{\cbf}   \circ R_{\psi} \circ z_{\frac{\delta}{\delta'}} \circ h_{\theta,\delta'} \circ R_{\phi} \circ \tau_{\xbf_v}
\end{equation*}
is affine if and only if $\theta=0$. In this case we have
\begin{equation*}
h= \tau_{\cbf} \circ z_{-\frac{\delta}{\delta'}} \circ R_{\phi+\psi} \circ \tau_{\xbf_{v}}
=\tau' \circ z_{-\frac{\delta}{\delta'}} \circ  R_{\phi+\psi}
\end{equation*}
However the affinity can be seen as a limit case. Let $k$ be the ratio $\frac{\delta}{\delta'}$. If $\delta'$ and $\delta$ tense to $+\infty$, the function $h_{\theta,\delta'}$, the function $h_{\theta,\delta'}$ tense to $h_{\theta,\infty}$ defined by
\begin{equation*}
h_{\theta,\infty}=(x,y)=(-\cos(\theta)x,-y)
\end{equation*}
Physically, it is equivalent to moving away from the plane while increasing the focal distance in order to keep the size of the output image constant.

If $h_\infty = z_{-\frac{\delta}{\delta'}} \circ \tau_{\xbf} \circ R_{\phi} \circ h_{\theta,\infty} \circ R_{\psi} \circ \tau_{\xbf_{v}}$, the linear part $h_{\infty}$ can be represented by a matrix $2\times2$

\begin{equation*}
R_{\psi} \cdot 
\begin{pmatrix}
-k\cos(\theta)&0\\
0&-k
\end{pmatrix}
\cdot R_{\phi}
\end{equation*}

If $M$ is an invertible matrix $2\times 2$, then we have the following lemma using the singular value decomposition \cite{morel2009asift}.
\begin{lem}
There exists two rotation matrix $R_1$ and $R_2$ and a diagonal matrix $D$ such that $M = R_1 \cdot D \cdot R_2$.
\label{decomp_valeur_sing}
\end{lem}

Using the lemma (\ref{decomp_valeur_sing}) it can be deduced that for all bijective affinity $A$, there exists a camera movement $h$ such that $h_\infty = A$. Moreover it can be assumed that $h$ has no output translation.
\end{remarque}






\subsubsection{Application of the decomposition to homographies}
The previous results show that all homographies can be decomposed using the following Theorem

%Le résultat précédent montre que certaine homographie peuvent se décomposer de la
\begin{thm}
Let $h$ be an homography, if $h$ is not an affine map then there exists parameters $(\phi,\theta,\psi,\delta,\delta',(x_1,y_1),(x_2,y_2))$ such that
\begin{equation*}
h = \tau_{(x_2,y_2)} \circ R_{\psi} \circ z_{\frac{\delta}{\delta'}} \circ h_{\theta,\delta'} \circ R_{\phi} \circ \tau_{(-x_1,-y_1)}
\end{equation*}
This decomposition is not unique. More precisely for all $\lambda \in ]0,1[$,

  \begin{equation*}
h = \tau_{(x_2,y_2)} \circ z_{\frac{\delta}{\delta'}}  \circ R_{\psi} \circ h_{\theta,\delta'} \circ R_{\phi} \circ \tau_{(-x_1,-y_1)}
  \end{equation*}
  where 
 \begin{equation*}
x_2=\frac{ar+sb+\hat r \lambda}{r^2 +s^2}, y_2=\frac{cr+sd+\hat s \lambda}{r^2 +s^2}, (x_1 , y_1) = h^{-1}(x_{2},y_{2})
  \end{equation*}
 \begin{equation*}
 \cos( \phi )= - \frac{r}{\sqrt{r^2 + s^2}}, \sin( \phi )= - \frac{s}{\sqrt{r^2 + s^2}},\cos( \psi ) =- \frac{\hat r}{\sqrt{\hat r^2 + \hat s^2}}, \sin( \psi ) = \frac{\hat s}{\sqrt{\hat r^2 + \hat s^2}}
 \end{equation*}
 \begin{equation*}
 \frac{\delta}{\delta'}=|\lambda|\sqrt{\frac{\hat r^2 + \hat s^2}{r^2 + s^2}}^{3}, \cos(\theta)=\lambda, \sin(\theta)=\sqrt{1-\lambda^2}, \delta'=  \frac{\sqrt{(r^2 + s^2)(1-\lambda^2)}}{|\lambda| (\hat r^2+\hat s^2)}
 \end{equation*}
\label{thepropdecomp}
\end{thm}

\begin{corollaire} If $h$ an homography and $h$ is not affine, then there exists a translation $\tau$, two rotations $R_\phi ,R_\psi$ and a unidirectional homography $\tilde{h}$ such that
\begin{equation}
h=\tau \circ R_\psi \circ \tilde{h} \circ R_\phi
\label{formule_decomposition_effective}
\end{equation}
and this decomposition is not unique.
\end{corollaire}

		This formula is used in the algorithm \ref{pseudoCodeDecompo}.
		
		\begin{proof}
	 Using theorem (\ref{thepropdecomp}),there exists $(\phi,\theta,\psi,\delta,\delta',\xbf_v,\cbf)$ such that 
	 \begin{equation*}
	 h = \tau_{\cbf} \circ R_{\psi} \circ z_{\frac{\delta}{\delta'}} \circ h_{\theta,\delta'} \circ R_{\phi} \circ \tau_{\xbf_v}
	 \end{equation*}
	 $R_\psi$ and $z_{\frac{\delta}{\delta'}}$ commute. Let $\tau'$ be the theorem such that $\tau' \circ R_\phi =  R_\phi \circ \tau_{\xbf_v}$.\\
	 Then with $\tilde{h} = z_{\frac{\delta}{\delta'}} \circ 
	 h_{\theta,\delta'} \circ \tau'$ it can be easily verified that $\tilde{h}$ is indeed a unidirectional homography.
	 \end{proof}
	\label{ref_schema_decomp_cool}
	\begin{figure}
		\centering
		\subfigure[Input image]{
		\centering
		{\includegraphics[scale=0.24]{vue_fps_identity.png}}
		{\includegraphics[scale=0.35]{vue_tps_identity.png}}}
		\subfigure[After a first rotation (of angle $\phi$)]{
		\centering
		{\includegraphics[scale=0.24]{vue_fps_rotation_phi.png}}
		{\includegraphics[scale=0.35]{vue_tps_rotation_phi.png}}}
		\subfigure[After the unidirectional homography]{
		\centering
		{\includegraphics[scale=0.24]{vue_fps_hom_part.png}}
		{\includegraphics[scale=0.35]{vue_tps_hom_part.png}}}
		\subfigure[Output image (after the rotation of angle  $\psi$)]{
		\centering
		{\includegraphics[scale=0.24]{vue_fps_rotation_psi.png}}
		{\includegraphics[scale=0.35]{vue_tps_rotation_psi.png}}}
		\caption{Steps to process an homography, represented as camera moves, on the left the view from the camera, on the left a motionless point of view. The translations are not represented to simplify the situation. On motionless views, $F$ is the focus of the camera, the red plane is the image plane of the camera. (cf. \ref{ref_schema_decomp_cool})}
		\label{schema_decomp_cool}
		\label{SchemaEtapesDecompoGeometrique}
	\end{figure}
	\clearpage
