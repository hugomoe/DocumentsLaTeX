% contient l'introduction
	Les homographies sont des applications projectives du plan ; elles correspondent aux mouvements d'une caméra devant une scène plane, ou plus généralement à tout changement de point de vue. Elles ont donc de nombreuses applications en traitement d'image comme par exemple le \emph{texturing} \cite{heckbert1983texture} (dans le jeux vidéo, l'animation, etc...) ou encore le recalage précis d'une suite d'images prises du même endroit \cite{brown2007automatic} (ce qui est la première étape de plusieurs méthodes de super-résolution). Ce problème est actuellement traité à l'aide de variantes du Mipmap \cite{williams1983pyramidal}. Nous comparons la méthode présentée ici à l'une d'entre elle, le Ripmap \cite{akenine2008real}.

	Une méthode efficace pour traiter les affinités, qui sont un cas particulier d'homographie, a été présentée récemment \cite{szeliski2010high}. Mais le traitement des homographies n'en découle pas trivialement puisqu'il s'agit de faire des zooms différents pour chaque point de l'image, et qu'il faut éviter le flou et l'\emph{aliasing}. Une méthode naïve de traitement des homographies comme décrite en première partie produit par exemple beaucoup d'\emph{aliasing}. Cet article présente une nouvelle méthode permettant de traiter toute homographie à l'aide d'une décomposition géométrique. Elle réduit le flou et l'\emph{aliasing}, mais elle est plus lente que le Ripmap.

	La partie \ref{Exposition_du_probleme} de cet article traite du problème des homographies en général et fait une présentation non-exhaustive des méthodes actuelles pour traiter les homographies. Elle présente aussi une méthode de traitement des affinités qui est utilisée dans la méthode par décomposition.

	La partie \ref{decomp_geo_hom} présente une interprétation géométrique d'une homographie en terme de mouvement de caméra. Elle permet de comprendre la théorie qui justifie cette nouvelle méthode. Elle explique aussi comment décomposer une homographie à partir de cette interprétation.

	La partie \ref{experiences} montre les performances et le gain en qualité de la méthode de traitement des homographies par décomposition par rapport aux méthodes existantes (notamment le Ripmap).

	Les pseudo-codes permettant de mettre en oeuvre cette nouvelle méthode sont présents en annexe.
