

Le but de cette partie est d'établir la théorème (\ref{thepropdecomp}).

 On rappelle d'abord les notions sur les homographies qui seront utiles dans la suite. Le lien entre les homographies et les espaces projectifs ne sera pas utlisé. On rappelle que dans ce document, une homographie $h$ est une application bijective de la forme :
	\[h:(x,y)\ra \left(\frac{ax+by+p}{rx+sy+t},\frac{cx+dy+q}{rx+sy+t}\right)\]
(l'ensemble d'arrivée est $\mathbb R^2$ privé d'une droite).

Les applications affines sont un cas particulier d'homographie ; si une homographie n'est pas une application affine alors elle est définie sur le plan privé d'une droite.

L'ensemble des homographies a une structure de groupe pour la loi de composition.

On peut associer à l'homographie $h$ la matrice $H$ définie par
  
\begin{equation*}
	H=\begin{pmatrix}
	a&b&p\\c&d&q\\r&s&t
	\end{pmatrix}
\end{equation*}
 
 On notera
 \begin{equation*}
 H^{-1}=\begin{pmatrix} \hat a&\hat b&\hat p\\ \hat c&\hat d&\hat q\\ \hat r&\hat s&\hat t \end{pmatrix}
 \end{equation*}

Cette matrice est inversible car $h$ est inversible ; elle n'est pas unique car la matrice $\lambda H$ définit la même homographie pour $\lambda \in \mathbb{R}_+$.

Cette notation rend compatible le produit matriciel et la composition des homographies. On obtient un morphisme de groupe de $GL_{3}(\mathbb{R})$ dans le groupe des homographies du plan. Ce morphisme n'est pas injectif, la matrice d'une homographie est définie à proportionnalité près, mais il se factorise à travers $SL_{3}(\mathbb{R})$ en un isomorphisme.

Dans la suite on notera $\sim$ la relation d'équivalence  sur $GL_{3}(\mathbb{R})$ définie par \[A\sim B \iff \exists \lambda\in \mathbb{R}^{*} , A=\lambda B\] c'est-à-dire si, et seulement si, $A$ et $B$ définissent la même homographie.


\begin{proof}

On fixe $h$ une homographie et $H$ une matrice qui lui est associée. On cherche à prouver qu'il existe des paramètres $(\theta,\phi,\psi,\delta,\delta',\cbf,\xbf_v)$ tels que :
\begin{equation*}
h= \tau_{\cbf} \circ z_{\frac{\delta}{\delta'}}\circ R_{\psi} \circ h_{\theta,\delta'} \circ R_{\phi} \circ \tau_{\xbf_v}
\end{equation*}
On conserve ici les notations de la partie précédente, on suppose sans perte de généralité que $\det (H)=1$.\\
Les transformations intervenant dans la décomposition sont des homographies, on peut réecrire cette relation sous forme matricielle. On cherche à prouver qu'il existe $(\theta,\phi,\psi,\delta,\delta',\cbf,\xbf_v)$ tels que
\begin{equation*}
H\sim T_{\cbf} Z_{\frac{\delta}{\delta'}}  R_{\psi}  H_{\theta,\delta'} R_{\phi}  T_{\xbf_v}
\end{equation*}
avec
\begin{equation*}
R_{\alpha}=\begin{pmatrix}
\cos(\alpha)&\sin(\alpha)&0\\-\sin(\alpha)&cos(\alpha)&0\\0&0&1
\end{pmatrix}
, H_{\theta,\delta'}=\begin{pmatrix}
-\cos(\theta)&0&0\\0&-1&0\\-\frac{\sin(\theta)}{\delta'}&0&1
\end{pmatrix},
\end{equation*}
\begin{equation*}
Z_{\lambda}=\begin{pmatrix}
\lambda&0&0\\0&\lambda&0\\0&0&1
\end{pmatrix}
\text{ et } T_{(\alpha,\beta)}=\begin{pmatrix}
1&0&-\alpha\\0&1&-\beta\\0&0&1
\end{pmatrix}
\end{equation*}
 On cherche dans un premier temps à déterminer les translations. Soient $(x_1 , y_1 )$ et $(x_2 , y_2 )$ deux vecteurs on considère la matrice $H_t$ définie par $H_t = T_{-(x_2 , y_2 )}  \cdot H \cdot T_{(x_1 , y_1 )}$, on cherche à prouver que pour certaines valeurs de $(x_1 , y_1 )$ et $(x_2 , y_2 )$  il existe  $(\theta,\phi,\psi,\delta,\delta')$ tels que   $H_t=Z_{\frac{\delta}{\delta'}} \cdot R_{\psi} \cdot H_{\theta,\delta} \cdot R_{\phi}$.
 
 Par un calcul on obtient que  pour tous $(\theta,\phi,\psi,\delta,\delta')$ la matrice $Z_{\frac{\delta}{\delta'}} \cdot R_{\psi} \cdot H_{\theta,\delta} \cdot R_{\phi}$ est égale à : 
  \begin{equation*}
\begin{pmatrix}
 -\frac{\delta}{\delta'}\cos(\psi)\cos(\theta)\cos(\phi)+\frac{\delta}{\delta'}\sin(\psi)\sin(\phi)& -\frac{\delta}{\delta'}\cos(\psi)\cos(\theta)\sin(\phi)-\frac{\delta}{\delta'}\sin(\psi)\cos(\phi)&0\\
  \frac{\delta}{\delta'}\sin(\psi)\cos(\theta)\cos(\phi)+\frac{\delta}{\delta'}\cos(\psi)\sin(\phi)& \frac{\delta}{\delta'}\sin(\psi)\cos(\theta)\sin(\phi)-\frac{\delta}{\delta'}\cos(\psi)\cos(\phi)&0\\ -\frac{\sin(\theta)}{\delta'}\cos(\phi)&-\frac{\sin(\theta)}{\delta'}\sin(\phi)& 1
 \end{pmatrix}
 \end{equation*}
 On a de plus pour tout $(x_1,y_1,x_2,y_2)$
 \begin{equation*}
 H_t=\begin{pmatrix}
 a-x_2 r&b-x_2 s& a x_1 + b y_1 + p -x_2 (r x_1 +s y_1 +t)\\
  c-y_2 r&d-y_2 s& c x_1 + d y_1 + q -y_2 (r x_1 +s y_1 +t)\\
  r & s & r x_1 + s y_1 +t
 \end{pmatrix}
 \end{equation*}
 On a donc l'équivalence 
 \begin{equation*}
 (H_t)_{1,3}=(H_t)_{2,3}=0 \iff (x_2,y_2)=h(x_1,y_1) \iff (x_1,y_1)=h^{-1}(x_2,y_2)
 \end{equation*}
 On pose $(x_1,y_1)=h^{-1}(x_2,y_2)$ car les calculs intermédiaires sont moins fastidieux, on suppose $(x_2,y_2)$ tels que $\hat r x_2 +\hat s y_2 + \hat t \ne 0$.
 
On pose alors
\begin{equation*}
H_t
  \sim 
  \begin{pmatrix}
 (a-x_2 r)(\hat r x_2 + \hat s y_2 +\hat t)&(b-x_2 s)(\hat r x_2 + \hat s y_2 +\hat t)& 0\\
  (c-y_2 r)(\hat r x_2 + \hat s y_2 +\hat t)&(d-y_2 s)(\hat r x_2 + \hat s y_2 +\hat t)& 0\\
  r(\hat r x_2 + \hat s y_2 +\hat t) & s(\hat r x_2 + \hat s y_2 +\hat t) &1
  \end{pmatrix} 
\end{equation*}
On pose alors 
 \begin{equation*}
 -\frac{\sin(\theta)}{\delta'}\cos(\phi)=r(\hat r x_2 + \hat s y_2 +\hat t)\text{ et } -\frac{\sin(\theta)}{\delta'}\sin(\phi)=s(\hat r x_2 + \hat s y_2 +\hat t)
 \end{equation*}
 On sait que $r^{2}+s^{2}=0$ si et seulement si l'homographie associée à H est une affinité. Ce cas ci sera traité de façon indépendante, on suppose ici que l'homographie $h$ n'est pas une affinité. On peut donc écrire :
 \begin{eqnarray*}
 \cos(\psi) &=& \sgn\left(-\frac{\sin(\theta)}{ \delta'}\right)\sgn(\hat r x_2 + \hat s y_2 +\hat t)\frac{r}{\sqrt{r^{2}+s^{2}}}\\
 \sin(\psi) &=& \sgn\left(-\frac{\sin(\theta)}{ \delta'}\right)\sgn(\hat r x_2 + \hat s y_2 +\hat t)\frac{s}{\sqrt{r^{2}+s^{2}}}
 \end{eqnarray*}
 On peut se restreindre au cas $\frac{\sin(\theta)}{\delta'}>0$ :
 En effet si $\frac{\sin(\theta)}{\delta'}<0$ on obtient alors
 \begin{equation*}
 R_{\psi} \cdot H_{\theta,\delta'} \cdot R_{\phi}=R_{\psi} \cdot Z_{-1}\cdot H_{\theta,-\delta'}\cdot Z_{-1} \cdot R_{\phi}= R_{\psi+\pi} \cdot H_{\theta,-\delta'}\cdot R_{\phi+\pi}
 \end{equation*}
 et on a $-\frac{\sin(\theta)}{\delta'}>0$.


 On obtient alors 
 \begin{equation*}
 \cos( \phi )= -\sgn(\hat r x_2 +\hat s y_2 +\hat t) \frac{r}{\sqrt{r^2 + s^2}} \text{ et } \sin( \phi )= -\sgn(\hat r x_2 +\hat s y_2 +\hat t) \frac{s}{\sqrt{r^2 + s^2}}
 \end{equation*}
 Comme 
 \begin{equation*}
 H_t \cdot R_{\phi}^{-1} \sim
 \begin{pmatrix}
 -|\hat r x_2 +\hat s y_2 +\hat t|\frac{(ar+sb)-(r^2 + s^2)x_2}{\sqrt{r^2 + s^2}}&-|\hat r x_2 +\hat s y_2 +\hat t|\frac{\hat s}{\sqrt{r^2 + s^2}}&0\\
 -|\hat r x_2 +\hat s y_2 +\hat t|\frac{(cr+sd)-(r^2 + s^2)y_2}{\sqrt{r^2 + s^2}}&|\hat r x_2 +\hat s y_2 +\hat t|\frac{r}{\sqrt{r^2 + s^2}}&0\\
 -|\hat r x_2 +\hat s y_2 +\hat t|\sqrt{r^2 + s^2}&0&1
 \end{pmatrix}
 \end{equation*}
 Et d'un autre côté 
 \begin{equation*}
Z_{\frac{\delta}{\delta'}} \cdot R_{\psi} \cdot H_{\theta,\delta}  \sim 
 \begin{pmatrix}
 -\frac{\delta}{\delta'}\cos(\psi)\cos(\theta)&
-\frac{\delta}{\delta'}\sin(\psi)&
0\\
\frac{\delta}{\delta'}\sin(\psi)\cos(\theta)&
-\frac{\delta}{\delta'}\cos(\psi)&
0\\
-\frac{\sin(\theta)}{\delta'}&
0&
1
 \end{pmatrix}
 \end{equation*}
Comme $h$ n'est pas une affinité on a $\hat r^2 + \hat s^2 \ne 0$ 
 \begin{equation*}
  \cos( \psi ) =- \sgn(\frac{\delta}{\delta'})\frac{\hat r}{\sqrt{\hat r^2 + \hat s^2}}\text{ et } \sin( \psi ) = \sgn(\frac{\delta}{\delta'})\frac{\hat s}{\sqrt{\hat r^2 + \hat s^2}}
 \end{equation*}
On peut se restreindre au cas $\frac{\delta}{\delta'}>0$ :
En effet si $\frac{\delta}{\delta'}<0$ on obtient alors $Z_{\frac{\delta}{\delta'}} \cdot R_{\psi}=Z_{\left|\frac{\delta}{\delta'}\right|}\cdot Z_{-1} \cdot R_{\psi}=Z_{\left|\frac{\delta}{\delta'}\right|}\cdot R_{\pi} \cdot R_{\psi}=Z_{\left|\frac{\delta}{\delta'}\right|}\cdot R_{\psi+\pi}$.


Et donc 
 \begin{equation*}
  \cos( \psi ) =- \frac{\hat r}{\sqrt{\hat r^2 + \hat s^2}} \text{ et } \sin( \psi ) = \frac{\hat s}{\sqrt{\hat r^2 + \hat s^2}}
 \end{equation*}



 On obtient alors 
\begin{equation*}
R_{\psi}^{-1} \cdot H_t \cdot R_{\phi}^{-1} \sim 
 \begin{pmatrix}
 -|\hat r x_2 +\hat s y_2 +\hat t|(\hat r x_2 +\hat s y_2 +\hat t)\sqrt{\frac{r^2 + s^2}{\hat r^2 + \hat s^2}}&0&0\\
 |\hat r x_2 +\hat s y_2 +\hat t|\frac{\Delta_H(x_2 , y_2)}{\sqrt{r^2 + s^2}\sqrt{\hat r^2 + \hat s^2}}&-|\hat r x_2 +\hat s y_2 +\hat t|\sqrt{\frac{\hat r^2 + \hat s^2}{r^2 + s^2}}&0\\
 -|\hat r x_2 +\hat s y_2 +\hat t|\sqrt{r^2 + s^2}&0&1
 \end{pmatrix}
\end{equation*}
Où l'on a posé 
\begin{equation*}
\Delta_H(x_2 , y_2 ) =\hat r ((rc+sd)-(r^2 + s^2)y_2) - \hat s ((ar+sb)-(r^2 + s^2 )x_2)
\end{equation*}
Les solutions de $\Delta_H(x_2 , y_2 )=0$ sont
\[ \left\lbrace \left( x_2=\frac{ar+sb+ \hat r \lambda}{r^2 +s^2}, y_2=\frac{cr+sd+\hat s \lambda}{r^2 +s^2}\right), \lambda \in \mathbb R \right\rbrace\]
On a dans ce cas
\begin{equation*}
\hat r x_2 +\hat s y_2 +\hat t = \frac{\hat r^2 +\hat s^2}{r^2 + s^2} \lambda
\end{equation*}
Le paramètre $\lambda$ doit donc être pris différent de zéro car 
\begin{equation*}
R_{\psi}^{-1} \cdot H_t \cdot R_{\phi}^{-1} \sim 
 \begin{pmatrix}
 -| \lambda | \lambda \sqrt{\frac{\hat r^2 + \hat s^2}{s^2 + r^2}}^{3}&0&0\\
0&-| \lambda | \sqrt{\frac{\hat r^2 + \hat s^2}{r^2 + s^2}}^{3}&0\\
 -|\lambda|\frac{\hat r^2 + \hat s^2}{\sqrt{r^2 + s^2}}&0&1
 \end{pmatrix}
\end{equation*}
 
 
 On peut poser 
 \begin{equation*}
 \frac{\delta}{\delta'}=|\lambda|\sqrt{\frac{\hat r^2 + \hat s^2}{r^2 + s^2}}^{3}
 \end{equation*}
On obtient finalement
\begin{equation*}
Z_{\frac{\delta}{\delta'}}^{-1} \cdot R_{\psi}^{-1} \cdot H_t \cdot R_{\phi}^{-1} \sim 
 \begin{pmatrix}
 -\lambda&0&0\\
0&-1&0\\
 -|\lambda|\frac{\hat r^2 + \hat s^2}{\sqrt{r^2 + s^2}}&0&1
 \end{pmatrix}
 \end{equation*}
 Cette matrice doit être de la forme $H_{\theta,\delta'}$. Pour qu'elle le soit, on doit avoir 
 \begin{equation*}
  \lambda^2 + \lambda^2 \delta'^2 \frac{(\hat r^2 + \hat s^2)^2}{r^2 + s^2}=1
 \end{equation*}
 C'est-à-dire
 \begin{equation*}
  \delta'^2 = (r^2 + s^2) \frac{1-\lambda^2}{\lambda^2 (\hat r^2+\hat s^2)^2}
 \end{equation*}
\end{proof}
