% contient l'introduction

An homographies is a projective mapping on a two dimensional space which is a good model of camera motion and point of view's changes. Therefore they are used a lot in image processing such as in texturing \cite{heckbert1983texture} (for instance in video games, animation...), for panoramic image stitching \cite{brown2007automatic}








Les homographies sont des applications projectives du plan ; elles correspondent aux mouvements d'une caméra devant une scène plane, ou plus généralement à tout changement de point de vue. Elles ont donc de nombreuses applications en traitement d'image comme par exemple le \emph{texturing} \cite{heckbert1983texture} (dans le jeux vidéo, l'animation, etc...) ou encore le recalage précis d'une suite d'images prises du même endroit \cite{brown2007automatic} (ce qui est la première étape de plusieurs méthodes de super-résolution). Les homographies sont difficiles à traiter puisqu'il faut faire à la fois des zoom-in et des zoom-out avec différents facteurs pour une même image ce qui empêche de simplement interpoler par spline (à cause du zoom out).  Ce problème est actuellement traité à l'aide du Mipmap qui permet de traiter \cite{williams1983pyramidal} décrit en annexe \ref{Mipmap_pseudo_code_jt} ou de variantes. Il est parfois combiné avec un filtre anisotropique à plusieurs échantillons  \cite{barkans1997high}. Le Mipmap permet de traiter les zooms-out ; cependant il produit parfois beaucoup d'aliasing.  D'autres filtres tels que le Elliptic Weighted Average (EWA) convolue l'image directement avec un filtre gaussien \cite{greene1986creating}. Cependant le filtrage gaussien est connu pour créer de l'aliasing et pour trop flouter. Ce filtre étant par ailleur isotropique il ne filtre pas bien toutes les fréquences. De plus la méthode EWA est assez lente, même si des algorithmes plus rapide basés sur le Mipmap on été élaborés \cite{mccormack1999feline,huttner1999fast}. Nous comparons la méthode présentée ici à une variante anisotropique du Mipmap : le Ripmap \cite{akenine2008real} qui est décrite en annexe \ref{Ripmap_pseudo_code_jt}.

	Une méthode efficace pour traiter les affinités, qui sont un cas particulier d'homographie, a été présentée récemment \cite{szeliski2010high}. Mais le traitement des homographies n'en découle pas trivialement puisqu'il s'agit de faire des zooms différents pour chaque point de l'image, et qu'il faut éviter le flou et l'\emph{aliasing}. Cela réclame un ré-échantillonnage d'une grande précision. Une méthode naïve de traitement des homographies comme décrite en première partie produit par exemple beaucoup d'\emph{aliasing}. Cet article présente une nouvelle méthode permettant de traiter toute homographie à l'aide d'une décomposition géométrique. Elle réduit le flou et l'\emph{aliasing}, mais elle est plus lente que le Ripmap.

	La partie \label{szeliski_section} présente une méthode de traitement des affinités qui est utilisée dans la méthode par décomposition.

	La partie \ref{decomp_geo_hom} présente une interprétation géométrique d'une homographie en terme de mouvement de caméra. Elle permet de comprendre la théorie qui justifie cette nouvelle méthode. Elle explique aussi comment décomposer une homographie à partir de cette interprétation.

	La partie \ref{experiences} montre les performances et le gain en qualité de la méthode de traitement des homographies par décomposition par rapport aux méthodes existantes (notamment le Ripmap).

	Les pseudo-codes permettant de mettre en oeuvre cette nouvelle méthode, ainsi qu'une description du Mipmap et du Ripmap sont présents en annexe.
