Memoire -> article



code :
	4 pt -> animation linéaire entre les pts de choix et pts du carré.
	-> Mipmap/Ripmap -> JUSTE ripmap



Article : a reduire

PARTOUT
renforcer biblio.
remarque : formaliser 

MIPMAP/RIPMAP
	-> mentionner dans l’introduction$
		spline résout sans zoom arrière (+ref biblio)
		Mipmap : spé zoom arrière
	Mipmap Ripmap -> succint
	présentation Mipmap -> appendice
	fction de distance à comprimer (figure 6 + existe autre)

SZELISKI
	-> en appendice
	attention a affine map / affinity (-> n’importe…)
	

DECOMP GEO
	-la mythique 2.1 -> 
		clairement formuler : projective (R3->R2, def par 2 points) / 
			projective planaire (R2->R2, restriction avec choix d’axe)/ 
			mvt de caméra (paramétriser par theta/phi/psi de projective planaire) 
				-> autre nom, avec remarque que c’est mvt de caméra
		retirer le «si le point viser existe » systématique
		fin : remarque _ mvt de caméra 
		preuve calculatoire en appendice et prop plus clair

HOMO UNI
	c’est bien

ROTATIONS
	-> yaroslavski / szeliski très rapide

PSEUDO-CODE
	retirer Mipmap ? -> plutôt non
	mettre szeliski général et expliquer brièvement changement.





trad : petit plagiat pour intro
règle pour demo :
	formule dans les phrases intégrées naturellement.
	toute notion introduite
	câblage partout




PLAN FINAL

How to do affine and projective map ?

	1Intro
	2Affinity
	3Homo
		decomp
		homographie unidirectionnelle
	4exp
	
