% contient la description des différentes méthodes pour traiter les rotations de la décomposition

Pour les deux rotations dans la décomposition, nous avons testés plusieures méthodes : la méthodes de Yaroslavsky \cite{unser1995convolution} et la méthode de traitement des affinités multiétapes \cite{szeliski2010high}.

\subsubsection{Méthode de Yaroslavsky}

La méthode de Yaroslavsky consiste à décomposer une rotation en trois "shear".//

Sous forme matricielle un rotation s'écrit de manière générale :
\begin{equation*}
	H=\begin{pmatrix}
	cos \theta&-sin \theta\\sin \theta&cos \theta
	\end{pmatrix}
	\end{equation*}

La décomposition en trois shear est la suivante :
\begin{equation*}
	H=\begin{pmatrix}
	cos \theta&-sin \theta\\sin \theta&cos \theta
	\end{pmatrix}=\begin{pmatrix}
	1&-tan \frac{\theta}{2}\\0&1
	\end{pmatrix}\begin{pmatrix}
	1&0\\sin \theta&1
	\end{pmatrix}\begin{pmatrix}
	1&-tan \frac{\theta}{2}\\0&1
	\end{pmatrix}
	\end{equation*}

	Pour faire la rotation complète on fait chaque shear par fourier et on recentre l'image après chaque shear.\\
	En pratique on se ramène toujours à des rotations d'angle entre $[\frac{-\pi}{4},\frac{\pi}{4}]$ (on peut en effet faire des rotations d'angle $\pi$ et $\frac{\pi}{2}$ de manière exacte...)  
