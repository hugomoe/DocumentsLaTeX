% contient une explication du fonctionnement du Mip Map et du Rip Map, et des schémas explicatifs. En bref, un résumé de l'article de Williams
% parler éventuellement de la méthode naïve, ou d'autres méthodes actuelles qui ne fonctionnent pas

  Afin de traiter les homographies plusieures méthodes ont été développées. Elles sont pour la plupart des variantes de Mipmap présentée par Williams en 1983 \cite{williams1983pyramidal}. Cette partie présente le Mipmap et le Ripmap. Le Ripmap comparé avec la nouvelle méthode dans les expériences.

\sse{Présentation du Mipmap}
\label{Mipmap}
Le principe du Mipmap est de précalculer des  moyennes locales de l'image à plusieurs échelles, pour pouvoir ensuite calculer en temps réel une homographie. 

Pour cela on choisit de supposer que l'image est carrée de taille une puissance de 2 (quitte à faire un premier zoom). On calcule ensuite la valeur moyenne de certains carrés de l'image de taille une puissance de 2.
Le Mipmap est donc représenté par une suite d'images chacune deux fois plus petite que la précédente. Ainsi c'est un suite de \emph{zoom-out} de l'image d'origine de facteur une puissance de 2. 

Un exemple est représenté figure \ref{MipMap_real}.
\label{exemple_de_mipmap_figure}
\begin{figure}[h!]
\centering
\includegraphics[scale=0.4]{MipMap_real} %scale=0.4 ça fait vachement petit, non ?
\caption{Un exemple de Mipmap (cf. partie \ref{exemple_de_mipmap_figure})}
\label{MipMap_real}
\end{figure}

Quand on cherche la valeur d'un pixel de l'image d'arrivée, on approche la zone de l'image de départ à laquelle il correspond à l'aide de la différentielle de l'homographie inverse. On obtient alors un parallélogramme, qu'on essaye d'approcher avec des carrés. 


On appelle distance d'un pixel la taille du carré choisi pour l'approximation (car plus un pixel est "loin", plus les carrés qui l'approchent sont grands). 

On suppose que l'on dispose d'une formule qui nous donne la distance d'un point quelconque. La géométrie du Mipmap ne permettant que des distances puissance de 2, on fait une approximation trilinéaire : 

\begin{itemize}
  \item d'une part, on fait une interpolation linéaire entre deux étages dont les profondeurs encadrent la distance.
  \item d'autre part, dans chaque étage du Mipmap, on fait une interpolation bilinéaire entre les quatre carrés qui encadrent le pixel.
\end{itemize}

La figure \ref{intertrilineaire} schématise cette interpolation. 

\label{figure_schema_inter_trilin}
\begin{figure}[h!]
\centering
\includegraphics[scale=0.5]{intertrilineaire.jpg}
\caption{Schéma d'interpolation trilinéaire (cf. partie \ref{figure_schema_inter_trilin})}
\label{intertrilineaire}
\end{figure}


\sse{Les fonctions de distance}
\label{fonctiondistance}

On a supposer l'existence d'une fonction qui a un point associe une distance, on va maintenant s'intéresser au fonctions possibles.

Le choix de la distance est crucial car si $d$ est trop grand, l'image est inutilement floutée (on parle d'\emph{over-blurring}), et si $d$ est trop petit, l'image est aliasée.

Toutes les fonctions de distance dépendent de la différentielle de l'homographie inverse.
On note $(u,v)$ les coordonnées dans l'image d'origine et $(x,y)$ celles dans la fenêtre d'arrivée.

Il est aisé de calculer les coefficients des dérivées partielles. On note $H$ l'homographie inverse.

%re image prgm, avec les dérivé partielle indiqués

% On a jugé de la performance des méthodes "à vue". On compte à terme l'évaluer sur des cosinus/sinus.

On a ainsi $(u,v)=H(x,y)$, et on note par exemple $\frac{\dr u}{\dr x}$ la dérivée partielle de $u$ par rapport à $x$.

On donne ici sa formule.

$$ D(x,y) = \max \left(\sqrt{\left(\frac{\dr u}{\dr x}\right)^2 + \left(\frac{\dr v}{\dr x}\right)^2},\sqrt{\left(\frac{\dr u}{\dr y}\right)^2 + \left(\frac{\dr v}{\dr y}\right)^2}\right)$$

On prend le plus grand côté du parallélogramme comme côté du carré. Cette formule vient d'un article de Heckbert \cite{heckbert1983texture}. On la représente figure \ref{methode_plus_grand_cote}.
% On prend le plus grand côté du parallélogramme comme côté du carré \cite{heckbert1983texture}. Le résultat est bon.

\begin{figure}[h!]
\centering
\includegraphics[scale=0.5]{methode_plus_grand_cote.jpg}
\caption{Méthode du plus grand côté (ici A)}
\label{methode_plus_grand_cote}
\end{figure}


\sse{Algorithme amélioré : Le Ripmap}
\label{Ripmap}
Une des grandes faiblesses du Mipmap est l'isotropie : il ne privilégie aucune direction. Ainsi si le parallélogramme à approcher est un rectangle très plat, il n'y a pas d'approximation raisonnable à l'aide d'un carré.

Pour tenter de résoudre ce problème on utilise le Ripmap \cite{akenine2008real}. C'est en fait un Mipmap où l'on a aussi calculé la moyenne de tous les rectangles dont les côtés sont des puissances de deux. Un exemple est présenté dans la figure \ref{Ripmap_real}, ainsi que deux algorithmes de construction en annexe en \ref{pseudo_code_Ripmap}. L'algorithme \ref{buildRipmap1} est naïf, en pratique c'est l'algorithme \ref{buildRipmap2} qui est implémenté, il convole par une gaussienne avant chaque sous-échantillonage.

La fonction de distance ne renvoie plus une seule valeur mais une pour chaque côté du rectangle. On réalise alors une interpolation quadrilinéaire (bilinéaire entre les étages et bilinéaire dans chacun d'eux). Elle est représentée dans la figure \ref{interbibilineaire}. L'algorithme \ref{interbibi1} disposé en annexe en \ref{pseudo_code_Ripmap} décrit cette interpolation.


\label{label_figure_Ripmap_jt}
\begin{figure}[h!]
\centering
\includegraphics[scale=0.4]{Ripmap_real}
\caption{Un exemple de Ripmap (cf. partie \ref{label_figure_Ripmap_jt})}
\label{Ripmap_real}
\end{figure}


\label{label_schema_interp_quadrilin_jt}
\begin{figure}[h!]
\centering
\includegraphics[scale=0.5]{interbibilineaire.jpg}
\caption{Schéma d'interpolation quadrilinéaire (cf. partie \ref{label_schema_interp_quadrilin_jt})}
\label{interbibilineaire}
\end{figure}


On a utilisé le plus petit rectangle qui contient le parallélogramme. Ainsi la formule est :
$$D(x,y) = \left( \left|\frac{\dr u}{\dr x}\right|+\left|\frac{\dr u}{\dr y}\right|,\left|\frac{\dr v}{\dr x}\right|+\left|\frac{\dr v}{\dr y}\right|\right)$$
On a de plus décalé les points pour que le rectangle considéré soit bien celui qui contient le parallélogramme, comme on le voit sur la figure \ref{methode_distance_ripmap}.
\label{label_meth_petit_rect_jt}
\begin{figure}[h!]
\centering
\includegraphics[scale=0.5]{methode_distance_ripmap.jpg}
\caption{Méthode du plus petit rectangle pour le Ripmap (cf. partie \ref{label_meth_petit_rect_jt})}
\label{methode_distance_ripmap}
\end{figure}

On peut consulter le pseudo-code pour le Ripmap en section \ref{pseudo_code_Ripmap}.

Cela améliore certes la méthode, mais ne résout pas par exemple le cas d'un rectangle fin en diagonale.
