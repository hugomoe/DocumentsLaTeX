%\subsubsection*{Comparaison entre la méthode de Yaroslavsky et la méthode de traitement des affinités multi-étapes}

\subsection{Comparaison between Yaroslavsky's method and the multi-pass resampling method}

%	Dans les deux paragraphes suivants, l'erreur \emph{RMSE} correspond à la norme L2 de la différence entre l'image initiale et l'image ayant subie une rotation, l'erreur L1 est la norme L1 de cette même différence, et l'erreur max est la norme sup de cette différence.\\

	In the following paragraphs, the RMSE (Root Mean Square Error) is the $L^2$ norm of the difference between the initial image and the final image, the MAE (Mean Absolute Error) is the $L^1$ norm of this difference and the maximal error is the $L^\infty$ norm of this difference.\\

%	Afin d'observer des différences significatives entre les différentes méthodes de rotations, on effectue 360 rotations de 1 degré (sur trop peu de rotations, les erreurs sont faibles). On obtient les résultats figure \ref{troiscentrotations}. On y voit que la méthode multi-étapes ne résiste pas à tant de rotations, mais que l'interpolation par les b-splines peut être substituée à la convolution par un \emph{raised cosine-weighted sinc} sans création d'\emph{aliasing}, car les rotations sont d'angles faible.
%\label{pleinsderotations}

	To enhance the difference between rotation methods, 360 rotations of 1 degree were performed on the same image. The results are presented in figure \ref{troiscentrotations}. The multi-pass resampling method does not resist that many rotations, but since with small rotations the zoom factor is small, B-splines can be used instead of convolutions without introducing aliasing.
	
\label{pleinsderotations}

 \begin{figure}[h]
 \centering
   \subfigPDP{original lena.png}{lena.png}
   \subfigPDP{lena.png after 360 rotations by linear interpolation}{linear_lena.png}
   \subfigPDP{lena.png after 360 rotations by the multi-pass resampling method (interpolation filter : raised cosine-weighted sinc)}{raised-cosine_beta0-36_lena.png}
   \subfigPDP{lena.png after 360 rotations by the multi-pass resampling method (interpolation by B-spline of order 3)}{b-spline_order3_lena.png}
   \subfigPDP{lena.png after 360 rotations by the multi-pass resampling method (interpolation by B-spline of order 9)}{b-spline_order9_double_lena.png}
   \subfigPDP{lena.png after 360 rotations with Yaroslasky's method}{lena_360_rotations_yaro.png}
  
 \subfigure[Errors on the 360 rotations of 1 degree]{\begin{tabular}{|c|c|c|c|c|}
  \hline
  Method & RMSE  & MAE & maximal error & duration (s) \\
  \hline
  linear interpolation & 38.090 & 27.526 & 184.13 &  \bf{17.909}\\
  multi-pass resampling, b-spline of order 9 & \bf{6.8430} &  \bf{4.0370} & \bf{86.855} &  6114.1\\
  multi-pass resampling, b-spline of order 3 & 14.869 & 8.5208 & 158.04 & 1268.1\\
  Yaroslavsky & 14.187 & 7.7791 & 211.17 & 1839.9 \\
  multi-pass resampling, raised cosine-weighted sinc &  828.85 & 503.15 & 13638 & 778.42\\
  \hline
\end{tabular}} 
\caption{Resampling error after applying 360 one degree interpolations (see section \ref{pleinsderotations})}
\label{troiscentrotations}
 \end{figure}
	
%	Dans la décomposition géométrique d'une homographie, on ne fait que deux rotations. Il est donc plus pertinent de réaliser seulement 10 rotations de l'image lena.png de 36 degrés.\\

	In the geometric decomposition of an homography, only two rotations are necessary. It is more relevant to compute only 10 rotations of 36 degrees of the image lena.png.\\

%	On ne compare plus la méthode de traitement des affinités multi-étapes avec interpolation par b-spline car l'angle des rotations est cette fois trop grand pour considérer que $r_h$ et $r_v$ sont proches de 1.

	In that case, the multi-pass resampling method can not use b-spline to interpolate anymore : the angle is too large to consider $r_h$ and $r_v$ close to 1.
	
%	Les résultats sont sur la figure \ref{rotalena}. La méthode de traitement des affinités multi-étapes avec le \emph{raised cosine-weighted sinc} comme filtre d'interpolation et la méthode de Yaroslavsky semblent parfaites. Il reste donc à choisir entre ces deux dernières méthodes.\\

	The result are on figure \ref{rotalena}. The multi-pass resampling method using raised cosine-weighted sinc as interpolation filter and Yaroslavsky's method seem perfect.

%En terme de complexité, la méthode de Yaroslavsky est en $O(n log(n))$ (où $n$ est le nombre de pixel de l'image), tandis que la méthode de traitement des affinités multi-étapes est en $O(n)$. Ainsi pour des images de grande taille, il vaut mieux choisir la méthode de traitement des affinités multi-étapes.\\

	Computationnally, Yaroslavksy's method is more expensive ($O(n \log(n))$, $n$ the number of pixels) than the multi-pass resampling method ($O(n)$). Thus, for large-sized images, the multi-pass resampling method is better.\\

%Pour les expériences qui suivent, la méthode de traitement des affinités multi-étapes, avec le \emph{raised cosine-weighted sinc} comme filtre d'interpolation a été choisie pusiqu'elle est toujours dans les méthodes les plus rapides quelque soit la taille de l'image, et parce qu'elle donne des résultats excellents sur peu de rotations. 
	In the following experiment, this method has been chosen, since it is quiet fast and still gives excellent results on few rotations.
%On pourrait remplacer l'interpolation par convolution par une b-spline si on accordait moins d'importance au temps de calcul et si on avait une assurance que les rotations seront faibles (par exemple, si les homographies sont déjà presque unidirectionnelles), .
	The convolution could be replaced by b-spline interpolation if the computation time had been less important and if the rotations had been ensured to have small angles.

 \begin{figure}[h]
   \centering
   \subfigPDP{original lena.png}{lena.png}
   \subfigPDP{lena.png after 10 rotations by linear interpolation}{linear_10rot_lena.png}
   \subfigPDP{lena.png after 10 rotations by the multi-pass resampling method (interpolation filter : raised cosine-weighted sinc)}{lena_10_rotations_szeli.png}
   \subfigPDP{lena.png after 10 rotations by Yaroslasky's method}{lena_10_rotations_yaro.png}
   \subfigPDP{difference between lena.png and lena.png after ten rotations by multi-pass resampling method (interpolation filter : raised cosine-weighted sinc)}{raised-cosine_beta0_36_10rot_lena_error.png}
   \subfigPDP{difference between lena.png and lena.png after 10 rotations by Yaroslasky's method}{lena_10_rotations_yaro_error.png}
 \subfigure[Errors on 10 rotations of 36 degrees]{\begin{tabular}{|c|c|c|c|c|}
  \hline
  Method & RMSE  & MAE & maximal error  \\
  \hline
  linear interpolation & 14.722 & 8.5485 & 147.43  \\
  Yaroslavsky's method & 12.817 & 7.3749 & 159.84  \\
  multi-pass resampling &   \bf{5.9791} & \bf{3.6243} & \bf{72.129} \\
  \hline
 \end{tabular}}
 \caption{Effect for 10 rotations (see section \ref{pleinsderotations})}
 \label{rotalena}
 \end{figure}
