\documentclass{article}
\usepackage[latin1]{inputenc}
\usepackage{amsfonts}
\usepackage{graphicx}
\usepackage{stmaryrd}
\usepackage{nccmath}
\usepackage[french]{babel}
\selectlanguage{french} 
\newcommand{\A}[0]{{\cal A}}
\newcommand{\C}[0]{{\cal C}}
\newcommand{\se}[1]{\medbreak \medbreak \section*{#1}}
\newcommand{\sse}[1]{\medbreak \subsection*{#1}}
\newcommand{\ssse}[1]{\subsubsection*{#1}}
\newcommand{\Def}[0]{\ssse{Définition}}
\newcommand{\The}[0]{\ssse{Théorème}}
\newcommand{\Pro}[0]{\ssse{Propriété}}
\newcommand{\Exo}[1]{\sse{Exercice #1}}
\newcommand{\Que}[1]{\ssse{Question #1}}
\newcommand{\llb}[0]{\llbracket}
\newcommand{\rrb}[0]{\rrbracket}
\newcommand{\dd}[0]{\txt{d}}
\newcommand{\ssi}[0]{\Leftrightarrow}
\newcommand{\ra}[0]{\rightarrow}
\newcommand{\Ra}[0]{\Rightarrow}
\newcommand{\la}[0]{\leftarrow}
\newcommand{\La}[0]{\Leftarrow}
\newcommand{\txt}[1]{\textrm{#1}}
\newcommand{\ol}[1]{\overline{#1}}
\newcommand{\ul}[1]{\underline{#1}}
\newcommand{\mb}[1]{\mathbb{#1}}
\newcommand{\abs}[1]{|#1|}
\newcommand{\iso}[0]{\simeq}
\newcommand{\dr}[0]{\partial}
\newcommand{\floor}[1]{\lfloor #1\rfloor}
 \newcommand{\BlockIf}[1]{\KwSty{Start If} \\ #1 \\ \KwSty{End If}}
 \newcommand{\BlockElseIf}[1]{\KwSty{Start Else If} \\ #1 \\ \KwSty{End Else If}}
 \newcommand{\BlockElse}[1]{\KwSty{Start Else} \\ #1 \\ \KwSty{End Else}}
\usepackage[french,onelanguage]{algorithm2e}
\usepackage{algorithmique}
\begin{document}
\se{Peudo-code pour l'homographie particulière}

\sse{Conventions}

Les images sont des matrices. On a ignoré les éventuelles multiples niveaux de couleurs (qui seront traités indépendamment).

Il y a en faites une origine sur les images (pour pouvoir faire des translations instantanées).

\medbreak
Les fonctions $Extraire$ se contente de renvoyer la n-ième ligne/colonne.
Les fonctions $Integrale$ construisent simplement l'image intégrale de la ligne/colonne correspondante, on ne les a pas précisées.

\sse{Corps de l'algorithme}

\begin{algorithm}
\caption{$ApplyHomo(img,imgf,H)$}
\KwData{Une image de départ $img[1..w][1..h]$, une image d'arrivé $imgf[1..w_f][1..h_f]$, une homographie $H = \left( \begin{array}{ccc} a_1 & 0 & t_1 \\ 0 & a_2 & t_2 \\  \lambda & 0 & t_3\\ \end{array} \right)$}
$f_1 = x\mapsto \frac{a_1x + t_1}{\lambda x + t_3}$ \;
$f_2 = x,y\mapsto \frac{a_2y + t_2}{\lambda x + t_3}$ \;
\For{$j \leq h$}{
	$iimgw = ExtraireVerticale(img,j)$ \;
	\For{$i\leq w_f$}{
		$x = f_1(i)$\;
		$d = |f_1'(i)|$\;
		$img1(i,j) = Interpole(iimgw,x,d)$\;
	}
}
\For{$i\leq w$}{
	$iimgh = ExtraireHorizontale(img,i)$\;
	$d=\frac{\dr}{\dr y} f_2(i,0)$ (indépendant de $j$)\;
	\For{$j\leq h_f$}{
		$y=f_2(i,j)$\;
		$imgf(i,j)=Interpole(iimgh,y,d)$\;
	}
}
\end{algorithm}

\sse{Interpolation à l'aide de l'image intégrale}

Une première version naive. Nous n'avons pas réglé les cas où $InterpoleInt$ n'est pas définie.

\begin{algorithm}
\caption{$Interpole(iimg,u,d)$}
\KwData{Un vecteur $iimg$, une position $x\in\mb{R}$, une distance $d\in\mb{R}$}
$u=u-\frac{d}{2}$\;
$iimgint = Integrale(iimg)$\;
$id=floor(d)$\;
$iu=floor(u)$\;
$x = d-id$\;
$y = u-iu$\;
$a=InterpoleInt(iimgint,iu,id)$\;
$b=InterpoleInt(iimgint,iu+1,id)$\;
$c=InterpoleInt(iimgint,iu,id+1)$\;
$d=InterpoleInt(iimgint,iu+1,id+1)$\;
\KwRet{$(1-x)(1-y)*a+(1-x)y*b+x(1-y)*c+xy*d$}
\end{algorithm}

\begin{algorithm}
\caption{$InterpoleInt(iimgint,i,d)$}
\KwData{Un vecteur $iimgint$, deux entiers $i,d$}
\KwRet{$\frac{iimgint[i+d-1]-iimgint[i-1]}{d}$}
\end{algorithm}


\sse{A l'aide de la convolé de trois portes}

Notons $h_3$ la convolé de trois portes. On a $h_3'' = \mb{1}_{[-\frac{3}{2} -\frac{1}{2}]}-2*\mb{1}_{[-\frac{1}{2},\frac{1}{2}]}+\mb{1}_{[\frac{1}{2},\frac{3}{2}]}$.
Ainsi en notant $F_2$ l'intégrale seconde de l'image, on a $\int f h_3 = \int F h_3' = \int F_2 h_3''$.
Ainsi on peut prendre plutôt : 

\begin{algorithm}
\caption{$Interpole(iimg,u,d)$}
\KwData{Un vecteur $iimg$, une position $x\in\mb{R}$, une distance $d\in\mb{R}$}
$id=floor(d)$\;
$iu=floor(u)$\;
$x = d-id$\;
$y = u-iu$\;
$iimgint3 = Integrale(Integrale(Integrale(iimgint)))$\;
$a=Interpole3(iimgint3,iu,id)$\;
$b=Interpole3(iimgint3,iu+1,id)$\;
$c=Interpole3(iimgint3,iu,id+1)$\;
$d=Interpole3(iimgint3,iu+1,id+1)$\;
\KwRet{$(1-x)(1-y)*a+(1-x)y*b+x(1-y)*c+xy*d$}
\end{algorithm}

\begin{algorithm}
\caption{$Interpole3(iimg,i,d)$}
\KwData{Un vecteur $iimgint$, deux entiers $i,d$}
$a=InterpoleInt(iimgint3,i-d,d)$\;
$b=InterpoleInt(iimgint3,i,d)$\;
$c=InterpoleInt(iimgint3,i+d,d)$\;
\KwRet{$\frac{a-2*b+c}{d^2}$}
\end{algorithm}

\subsubsection{Pseudo code convolution d'une image $1D$ par l'image $4-$intégrale :}
 Dans ce pseudo code $img$ est l'image $1D$ d'entrée de taille $wh$ et $Img$ est une image de taille $4(wh+1)$ contenant les valeurs des images $i-integrales$ pour $i=1...4$ aux points $0...wh$ on calcule ces valeurs à l'aide de la méthode 2 énoncée au paragraphe ...
 On utilise essentiellement le format double lors de l'implémentation en c car les calculs demandent une certaine précision. La variable $xy$ est le point où l'on souhaite réechantillonner l'image et $d$ est le zoom effectué en $xy$ lors du ré-echantillonnage.
 
 \begin{algorithm}
 \caption{$build4Integrale(img,Img)$}
 \KwData{img,Img}
 \For{$i=0$ to $3$}{
 $Img[i(wh+1)]=0$
 }
 \For{$i=1$ to $wh$}{
 $Img[i]=Img[i-1]+img[i-1]$ image 1-intégrale\\
 $Img[wh+i+1]=Img[wh+i]+Img[i-1]+\frac{img[i-1]}{2}$ image 2-intégrale\\
 $Img[2 wh+i+2]=Img[2*wh+i+1]+Img[wh+i]+\frac{Img[i-1]}{2}+\frac{img[i-1]}{6}$ image 3-intégrale\\
 $Img[3 wh+i+3]=Img[wh+i+2]+Img[2wh+i+1]+\frac{Img[wh+i]}{2}+\frac{Img[i-1]}{6}+ \frac{img[i-1]}{24}$ image 4-intégrale
 }
 \end{algorithm}
 
  \begin{algorithm}
 \caption{$eval4Integral(img,Img,xy)$}
 \KwData{img,Img,point où l'on évalue xy }
 \uIf{$xy \ge wh$}{$r = xy - wh$\\
 $s = Img[4wh+3] + r \left (Img[3wh+2] + \frac{r}{2}  \left(Img[2wh+1] + \frac{r}{3}  Img[wh]\right)\right)$}
 \uElseIf{$xy<0$}{
  $s = 0$
 }
 \Else{
  $xys=\lfloor xy\rfloor$\\
  $r = xy-xys$\\
  $s = Img[3wh+3+xys] + r \left( Img[2wh+2+xys] + \frac{r}{2}  \left(Img[wh+1+xys] + \frac{r}{3}  \left(Img[xys]+ \frac{r}{4} img[xys] \right)\right)\right)$}
 \end{algorithm}
 
 \begin{algorithm}
 \caption{$convolImg(img,Img,xy,d)$}
 \KwData{img,Img,xy,d,wh}
 On calcule $D$ la largeur de la fonction porte servant à la convolution\\
 \eIf{$d>(\frac{0.49+0.01}{0.64})$}{
 $D = \sqrt{0.64 * d^{2} - 0.49}$
 }
 {
 $D=0.2$
 }
 $xy_{i}$ pour $ i=1...4$ sont les points où ré-échantillonner\\
 $xy_{1}  = xy + \frac{3D}{2}$\\
 $xy_{2}  = xy + \frac{D}{2}$\\
 $xy_{3}  = xy - \frac{D}{2}$\\
 $xy_{4}  = xy - \frac{3D}{2}$\\
 On réalise une première convolution pour lisser l'interpolation\\
 \For{$i=1...4$}{
 $a_{2i-1}=eval4Integrale(img,Img,xy_i+\frac{1}{2})$\\
 $a_{2i}=eval4Integrale(img,Img,xy_i-\frac{1}{2})$\\
 $b_i = a_{2i-1} - a_{2i}$ 
 }
 On réalise la dérivé discrète $3$ième pour convoler avec le filtre $g_3$\\
 $c_1 = \frac{b_1 - b_2}{D},c_2=\frac{b_2 - b_3}{D},c_2=\frac{b_3 - b_4}{D}$\\
 $d_1 = \frac{c_1 - c_2}{D},d_2=\frac{c_2 - c_3}{D}$\\
 $s = \frac{d_1 - d_2}{D}$\\
 \KwRet{s}
 \end{algorithm}
 
\end{document}
