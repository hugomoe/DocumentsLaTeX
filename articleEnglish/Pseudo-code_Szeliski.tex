\subsection{Pseudo-code pour la méthode multi-étapes de traitement des affinités}

 On définit des notations utilisées dans les pseudo-codes suivants :
 \begin{itemize}
 \item $x \ppcm y = \max(x,y)$, $x \pgcd y = \min(x,y)$
 \item $x^+ = 0 \ppcm x$, $x^- = 0 \ppcm (-x)$ les parties positives et négatives de $x$
 \end{itemize}
 
 Les algorithmes qui suivent correspondent à l'implémentation de la méthode multi-étapes de traitement des affinités. La méthode proposée dans l'article de Szeliski, Winder et Uyttendaele \cite{szeliski2010high} effectue uniquement l'affinité, d'une image de taille donnée dans une autre de même taille. Ici, on évite que la première étape de la décomposition ne rogne une partie de l'image en agrandissant le tableau sur lequel on travaille. On veut donc pouvoir préciser la taille des tableaux d'entrée et de sortie qui stockeront les images.
  
 \subsubsection{Corps de l'algorithme}
  
  Cet algorithme applique une affinité en appelant chacun des algorithmes qui suivent. On introduit des variables $x_i,y_i$ et $x_f,y_f$ qui représentent des coordonnées de points. Ces point est les centres de rectangles du plan, respectivement le rectangle contenant l'image initiale et le rectangle sur lequel créer l'image finale. Ainsi, on peut prendre en compte une translation en changeant $x_i,y_i$ avant d'effectuer l'opération $\mathcal R$. Cela évite de prendre en compte une translation dans $\mathcal R$, et donc de faire sortir l'image du tableau.
  
  Pour pouvoir sur-échantillonner (d'un facteur au plus 3 dans chaque direction), on doit plonger l'image d'entrée dans une image de dimensions 3 fois plus grandes. Pour créer des conditions au bords symétriques, il faut remplir par symétrie le reste de l'image plus grande. Il faut alors penser à mettre du noir, à la fin du traitement de l'homographie, là où il devait y en avoir (pour ne pas avoir des répliques symétrisées de l'image initiale autour de l'image de sortie).
  
   \begin{algorithme}
    \KwData{Une image d'entrée $img[1..m][1..n]$, une taille pour l'image de sortie $[1..m'][1..n']$, une matrice d'affinité $A = \pmatrice{a_{00} & a_{01} & t_0\\ a_{10} & a_{11} & t_1}$}
    $transpoOpt(img,A)$\;\ \\
    $u_{max}, v_{max} = frequencesMax(A)$\;
	$b_0 = a_{00}-\frac{a_{01}a_{10}}{a_{11}}$\;
	$b_1 = \frac{a_{01}}{a_{11}}$\;
	$r_v = \min(3,\max (1,|a_{01}|u_{max}+\min (1,|a_{11}|v_{max})$\;
	$r_h = \min(3,\max (1,|a_{10}/a_{11}|r_vv_{max}+\min (1,|b_0|u_{max})))$\;
	\ \\
	$dm,dn = \frac{m'-m}{2},\frac{n'-n}{2}$\;
	$img_0 = $ image vide de taille $3(m \ppcm m') \times 3(n \ppcm n')$\;
	Plonger $img$ au centre de $img_0$ et remplir le reste de $img_0$ par symétrie)\;
	\ \\
	$x_i = \frac{m}{2}, y_i = \frac{n}{2}$\;
	$x_f = x_i, y_f = \frac{r_v}{a_{11}}y_i$\;
	$img_1 = \mathcal{R}_v(img_0,\frac{1}{v_{max}},\frac{a_{11}}{r_v},0,(x_i,y_i),(x_f,y_f))$\;
	$x_i = x_f-t_2, y_i = y_f$;\tcc{translation selon les lignes}
	$x_f = \frac{r_h}{b_0} x_i - \frac{a_{01}r_h}{r_v b_0} y_i, y_f = y_i$\;
	$img_2 = \mathcal{R}_h(img_1,\frac{1}{u_{max}},\frac{b_0}{r_h},\frac{a_{01}}{r_v},(x_i,y_i),(x_f,y_f))$\;
	$x_i = x_f, y_i = y_f - \frac{t_1r_v}{a_{11}}$;\tcc{translation selon les colonnes}
	$x_f = x_i, y_f = \frac{n \ppcm n'}{2}$\;
	$img_3 = \mathcal{R}_v(img_2,r_v,r_v,\frac{a_{10}r_v}{a_{11}r_h},(x_i,y_i),(x_f,y_f))$\;
	$x_i = x_f, y_i = y_f$;
	$x_f = \frac{m \ppcm m'}{2}, y_f = \frac{n \ppcm n'}{2}$\;
	$img_4 = \mathcal{R}_h(img_3,r_h,r_h,0,(x_i,y_i),(x_f,y_f))$\;
	\KwRet{$img_4[m \ppcm m' .. (m \ppcm m') + m'][n \ppcm n' .. (n \ppcm n') + n']$}
    \caption{Traitement multi-étapes d'une affinité $applyAffinity(img,A)$}
    \label{algoPresqueAussiUniqueQueLesDeuxAutres}
   \end{algorithme}
   
  Dans le traitement de l'affinité totale, on transpose éventuellement l'image (et l'affinité) pour éviter le \emph{bottleneck problem}.
  
   \begin{algorithme}
    \KwData{Une image $img[1..m][1..n]$, une matrice d'affinité $A = \pmatrice{a_{00} & a_{01} & t_0\\ a_{10} & a_{11} & t_1}$}
    \bloc{Normalisation des lignes de $A$}{
     \For{$k \in \{0,1\}$}{
      $l_k=\sqrt{a_{k0}^2+a_{k1}^2}$\;
      $\hat{a}_{k0}=a_{k0}/l_k$\;
      $\hat{a}_{k1}=a_{k1}/l_k$\;
     }
    }
    \bloc{Transposition}{
     \If{$|\hat{a}_{00}|+|\hat{a}_{11}| < |\hat{a}_{01}|+|\hat{a}_{10}|$}{
      $img_{temp} = $ tableau de taille $n \times m$\;
      \For{$(i,j) \in \llbracket 1,m \rrbracket \times \llbracket 1,n \rrbracket$}{
       $img_{temp}[j][i]=img[i][j]$\;
      }
      $img=img_{temp}$\;
      $A=\pmatrice{a_{10} & a_{11} & t_1\\ a_{00} & a_{01} & t_0}$\;
      $\pmatrice{a_{00} & a_{01} & t_0\\ a_{10} & a_{11} & t_1}=A$\;
      $(m,n) = (n,m)$\;
     }
    }
    \caption{Transposition éventuelle $transpoOpt(img,A)$ (décrit en \ref{szeliski_transpoOpt_section})}
    \label{szeliski_transpoOpt}
   \end{algorithme}
 
 \subsubsection{Fréquences maximales $u_{max}, v_{max}$}

  Dans $frequencesMax$, on détermine $u_{max}$ et $v_{max}$ les fréquences conservées maximales, au-delà desquelles appliquer l'affinité coupera le spectre. Ce pseudo-code s'appuie sur deux autres, $intersectionsVerticales$ et $intersectionsHorizontales$, qui intersectent un segment (dont les extrémités ont pour coordonnées $(U_1,V_1)$ et $(U_2,V_2)$) avec deux côtés parallèles du carré $[-1,1]^2$ (soit les deux côtés verticaux, soit les deux horizontaux). La connaissance de ces intersections et la symétrie par rapport au point $(0,0)$ de l'application linéaire suffisent à connaître la totalité des sommets du polygone $[-1,1]^2 \cap \ ^t\!\!A^{-1}[-1,1]^2$, et donc les points de coordonnées maximales.
  
  NB : $^t\!\!A^{-1}[-1,1]^2$ est l'ensemble des fréquences conservées par l'effet de l'affinité $A$.

  \begin{algorithme}
   \KwData{$A$ matrice d'application linéaire $2 \times 2$}
   $\pmatrice{u_1\\v_1} =\ ^t\!\!A^{-1}\pmatrice{1\\1}$\;
   $\pmatrice{u_2\\v_2} =\ ^t\!\!A^{-1}\pmatrice{1\\-1}$\;
   $M = $ tableau vide de taille $2 \times 0$\;
   \If{$|u_1|\leq1$ et $|v_1|\leq1$}{
    Concaténer $M$ et $(u_1,v_1)^T$\;
   }
   \If{$|u_2|\leq1$ et $|v_2|\leq1$}{
    Concaténer $M$ et $(u_2,v_2)^T$\;
   }
   \uIf{$u_1=u_2$}{
    \eIf{$v_1=-v_2$}{
     \KwRet{$(\min(|u_1|,1),\min(|v_1|,1)$}
    }{
     Concaténer $M$ et $intersectionsHorizontales(u_1,v_1,u_2,v_2)$\;
     Concaténer $M$ et $intersectionsVerticales(u_1,v_1,-u_2,-v_2)$\;
     Concaténer $M$ et $intersectionsHorizontales(u_1,v_1,-u_2,-v_2)$\;
    }
   }
   \uElseIf{$u_1=-u_2$}{
    \eIf{$v_1=v_2$}{
     \KwRet{$(\min(|u_1|,1),\min(|v_1|,1)$}
    }{
     Concaténer $M$ et $intersectionsHorizontales(u_1,v_1,-u_2,-v_2)$\;
     Concaténer $M$ et $intersectionsVerticales(u_1,v_1,u_2,v_2)$\;
     Concaténer $M$ et $intersectionsHorizontales(u_1,v_1,u_2,v_2)$\;
    }
   }
   \Else{
    Concaténer $M$ et $intersectionsVerticales(u_1,v_1,u_2,v_2)$\;
    Concaténer $M$ et $intersectionsVerticales(u_1,v_1,-u_2,-v_2)$\;
    \uIf{$v_1=v_2$}{
     Concaténer $M$ et $intersectionsHorizontales(u_1,v_1,-u_2,-v_2)$\;
    }
    \uElseIf{$v_1=-v_2$}{
     Concaténer $M$ et $intersectionsHorizontales(u_1,v_1,u_2,v_2)$\;
    }
    \Else{
     Concaténer $M$ et $intersectionsHorizontales(u_1,v_1,u_2,v_2)$\;
     Concaténer $M$ et $intersectionsHorizontales(u_1,v_1,-u_2,-v_2)$\;
    }
   }
   \tcc{À ce stade, $M$ est la liste presque complète des sommets $(M[1][i],M[2][i])$ du polygone dont on veut connaître les fréquences maximales}
   \eIf{$M$ est vide}{
    \KwRet{$(1,1)$}
   }{
   \KwRet{$(\max_{i}\{|M[1][i]|\},\max_{i}\{|M[2][i]|\})$}
   }
   \caption{$frequencesMax(A)$ (décrit en \ref{szeliski_frequencesMax_section})}
   \label{pseudo-code_umax_vmax}
  \end{algorithme}
  
  \begin{algorithme}
   \KwData{$(U_1,V_1)$ et $(U_2,V_2)$ coordonnées des extrémités d'un segment non vertical ($U_1\neq U_2$)}
   $u_+ = 1$\;
   $u_- = -1$\;
   $v_+ = V_1+(u_+-U_1)\frac{V_2-V_1}{U_2-U_1}$\;
   $v_- = V_1+(u_--U_1)\frac{V_2-V_1}{U_2-U_1}$\;
   $l =$ tableau de taille $2 \times 0$\;
   \If{$|u_+|\leq1$ et $|v_+|\leq1$ et ($U_1 \leq u_+ \leq U_2$ ou $U_2 \leq u_+ \leq U_1$)}{
    Concaténer $l$ et $(u_+,v_+)^T$\;
   }
   \If{$|u_-|\leq1$ et $|v_-|\leq1$ et ($U_1 \leq u_- \leq U_2$ ou $U_2 \leq u_- \leq U_1$)}{
    Concaténer $l$ et $(u_-,v_-)^T$\;
   }
   \caption{$intersectionsVerticales(U_1,V_1,U_2,V_2)$ (décrit en \ref{szeliski_frequencesMax_section})}
   \label{szeliski_intersectionsVerticales}
  \end{algorithme}

  \begin{algorithme}
   \KwData{$(U_1,V_1)$ et $(U_2,V_2)$ coordonnées des extrémités d'un segment non horizontal ($V_1 \neq V_2$)}
   $v_+ = 1$\;
   $v_- = -1$\;
   $u_+ = U_1+(v_+-V_1)\frac{U_2-U_1}{V_2-V_1}$\;
   $u_- = U_1+(v_--V_1)\frac{U_2-U_1}{V_2-V_1}$\;
   $M =$ tableau de taille $2 \times 0$\;
   \If{$|u_+|\leq1$ et $|v_+|\leq1$ et ($V_1 \leq v_+ \leq V_2$ ou $V_2 \leq v_+ \leq V_1$)}{
    Concaténer $M$ et $(u_+,v_+)^T$\;
   }
   \If{$|u_-|\leq1$ et $|v_-|\leq1$ et ($V_1 \leq v_- \leq V_2$ ou $V_2 \leq v_- \leq V_1$)}{
    Concaténer $M$ et $(u_-,v_-)^T$\;
   }
   \caption{$intersectionsHorizontales(U_1,V_1,U_2,V_2)$ (décrit en \ref{szeliski_frequencesMax_section})}
   \label{szeliski_intersectionsHorizontales}
  \end{algorithme}
  
  \subsubsection{Opérations de type $\mathcal R$}
  
  Les deux algorithmes ($\mathcal{R}_h$ et $\mathcal{R}_v$) fonctionnent de la manière suivante : dans un premier temps, on stocke à l'avance les valeurs prises par le filtre $h(\frac{\dot{}}{s})$ sur $\{k+p2^{-b},(k,p)\in \llbracket -N,N \rrbracket \times \llbracket 0,2^b-1 \rrbracket\}$. $s$ est donc un paramètre pour élargir éventuellement le support de la fonction filtre ; $b$ est un paramètre de précision des calculs. Dans un second  temps on utilise ces valeurs stockées pour convoler selon une direction avec ledit filtre.
  
   Dans ces pseudo-codes, on s'autorise à appeler des valeurs de l'image $img[i][j]$ sans vérifier que $i$ et $j$ sont compris dans l'image. On peut dans ce cas considérer le terme comme nul : même si cela crée de l'effet Gibbs, celui-ci ne sera visible que sur la réplique symétrisée de l'image, réplique qui sera mise à noir plus tard.
  
   \begin{algorithme}
    \KwData{Une image d'entrée $img[1..m][1..n]$, des coefficients $s, a_0, a_1 \in \mathbb{R}$, les coordonnées (dans le plan) du centre de l'image d'entrée $x_i,y_i$, les coordonnées (dans le plan) du centre de l'image de sortie $x_f,y_f$, un filtre d'interpolation $h$, $M$ tel que $\supp h \subset [-M,M]$, une précision de calcul en bits $b$}
    $N = \lceil sM \rceil$\;
    $H = $ tableau de taille $(2N+1) \times 2^b$\;
    Stockage dans $H$ des valeurs prises par $h$ ($H[k][p] = h(\frac{k+2^{-b}p}{s}$)\;\ \\
    $img_f = $ image vide de taille $m \times n$\;
    \bloc{Convolution (selon $i$) avec un noyau centré en $i'$ ; $\red j$ est constant(équation \ref{formule_convolution_discrete})}{
     \For{$(i,j)\in \llbracket 1,m \rrbracket \times \llbracket 1,n \rrbracket$}{
      $i' = \frac{m}{2}-x_i+a_0(i+x_f-\frac{m}{2})+a_1(j+y_f-\frac{n}{2})$\;
      $k^* = \lfloor i' \rfloor$\;
      $\phi = i'-k^*$\;
      $p = \lfloor 2^b\phi \rfloor$
      $img_f[i][\red j]=\displaystyle{\sum_{k=k^*-N}^{k^*+N}} H[k^*-k][p]*img[k][\red j]$\;
     }
    }
    \KwRet{$img$}
    \caption{$\mathcal{R}_h(img,s,a_0,a_1,(x_i,y_i),(x_f,y_f))$ (version modifiée de l'algorithme \ref{szeliski_rh})}
    \label{algoUniqueEnSonGenre}
   \end{algorithme}

   \begin{algorithme}
    \KwData{Une image $img[1..m][1..n]$, des coefficients $s, a_0, a_1, t \in \mathbb{R}$, les coordonnées (dans le plan) du centre de l'image d'entrée $x_i,y_i$, les coordonnées (dans le plan) du centre de l'image de sortie $x_f,y_f$, un filtre d'interpolation $h$, $M$ tel que $\supp h \subset [-M,M]$, une précision de calcul en bits $b$}
    $N = \lceil sM \rceil$\;
    $H = $ tableau de taille $(2N+1) \times 2^b$\;
    Stockage dans $H$ des valeurs prises par $h$\;\ \\
    $img_f = $ tableau de taille $m \times n$\;
    \bloc{Convolution (selon $j$) avec un noyau centré en $j'$ ; $\red i$ est constant(équation \ref{formule_convolution_discrete})}{
     \For{$(i,j)\in \llbracket 1,m \rrbracket \times \llbracket 1,n \rrbracket$}{
      $j' = \frac{n}{2}-y_i-y+a_i(i+x_f-\frac{m}{2})+a_0(j+y_f-\frac{n}{2})$\;
      $k^* = \lfloor j' \rfloor$\;
      $\phi = j'-k^*$\;
      $p = \lfloor 2^b\phi \rfloor$
      $img_f[\red i][j]=\displaystyle{\sum_{k=k^*-N}^{k^*+N}} H[k^*-k][p]*img[\red i][k]$\;
     }
    }
    \KwRet{$img_f$}
    \caption{$\mathcal{R}_v(img,s,a_0,a_1,(x_i,y_i),(x_f,y_f))$ (version modifiée de l'algorithme \ref{szeliski_rv})}
    \label{algoToutAussiUnique}
   \end{algorithme}

   \subsubsection{Complexité}
   
   $TranspoOpt$ est clairement linéaire en l'image d'entrée. $frequenceMax$ a une entrée de taille constante, et n'a pas de boucle, elle s'évalue donc en temps constant. En supposant le rapport hauteur/largeur borné, $img_0$ se déclare en $O(n_1+n_2)$ où $n_1$ est la taille de l'image d'entrée et $n_2$ celle de l'image de sortie.
   
   Pour ${\mathcal R}_v$ par exemple, chaque pixel s'évalue en $O(N)$, c'est-à-dire en $O(s)$. Ainsi la fonction est a première vue en $O(mns)$, mais dans l'algorithme $s$ vaut soit $r_v$ qui est borné par $3$, ou $\frac{1}{v_{max}}$. Dans le second cas on n'évalue qu'un nombre proportionnel à $v_{max}$ de pixels. Ainsi les ${\mathcal R}_v$ et ${\mathcal R}_h$ sont linéaire en leur image d'entrée dans cet algorithme.
   
   Or on voit que les $img_i$ sont de taille $O(n_1+n_2)$.
   \medbreak
   Finalement l'algorithme est donc en $O(n_1+n_2)$ où $n_1$ est la taille de l'image d'entrée et $n_2$ celle de l'image de sortie.
