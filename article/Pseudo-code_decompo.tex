\subsection{Pseudo-code pour la décomposition d'une homographie}
 
 On rappelle que dans les algorithmes qui suivent, on prend la convention suivante : $H = \pmatrice{a&b&p\\c&d&q\\r&s&t}$ est la matrice carrée de taille 3 (ou l'application homographique) telle que si $img$ est l'image d'entrée et $img_f$ est l'image de sortie, $img_f(x) = img(H(x))$. Ainsi, la décomposition $H \sim T_{c} R_{\psi}  \tilde H R_{\phi}$ est exécutée de gauche à droite. On notera cette décomposition $A \tilde H B$, et ces trois matrices seront les objets renvoyés par $decomposition$.
 
 Cet algorithme $decomposition$ (algorithme \ref{pseudoCodeDecompo}) présente la décomposition correspondant aux schémas \ref{SchemaEtapesDecompoGeometrique}. Arbitrairement, on choisit le degré de liberté de sorte à annuler une des translations (verticale ou horizontale) de $T_c$ : pour cela, on rappelle que $(x_2,y_2)$ (nommés $t_0,t_1$ dans l'algorithme) doivent vérifier
 \[\Delta_H(x_2 , y_2 ) \stackrel{\text{déf}}{=} R ((rc+sd)-(r^2 + s^2)y_2) - S ((ar+sb)-(r^2 + s^2 )x_2) = 0 \]
 On prend donc $t_0 = x_2 = -\frac{\Delta_H(0,0)}{(r^2+s^2)S}, t_1 = y_2 = 0$ ou $t_0 = x_2 = 0, t_1 = y_2 = -\frac{\Delta_H(0,0)}{(r^2+s^2)R}$, en choisissant de sorte à ne pas diviser par zéro.
 
   \begin{algorithme}
     \label{pseudoCodeDecompo}
     \caption{$decomposition(H)$}
     \KwData{Une matrice d'homographie $H \in \mathcal M_{3,3}$}
     \eIf{$as-br\neq0$}{
      $t_0 = -\frac{\Delta_H(0,0)}{(r^2+s^2)S}$\;
      $t_1 = 0$\;
     }{
      $t_0 = 0$\;
      $t_1 = -\frac{\Delta_H(0,0)}{(r^2+s^2)R}$\;
     }
     $N_\phi = \sqrt{r^2+s^2}$\;
     $R_\phi = \pmatrice{
      \frac{r}{N_\phi} & \frac{s}{N_\phi} & 0\\
      -\frac{s}{N_\phi} & \frac{r}{N_\phi} & 0\\
      0 & 0 & 1}$\;
     $N_\psi = \sqrt{((a+t_0r)s-(b+t_0s)r)^2+((c+t_1r)s-(d+t_1s)r)^2}$\;
     $R_\psi = \pmatrice{
      \frac{(c+t_1r)s-(d+t_1s)r}{N_\psi} & \frac{(a+t_0r)s-(b+t_0s)r}{N_\psi} & 0\\
      -\frac{(a+t_0r)s-(b+t_0s)r}{N_\psi} & \frac{(c+t_1r)s-(d+t_1s)r}{N_\psi} & 0\\
      0 & 0 & 1}$\;
     $A = T^{-1}R_\psi$\;
     $\tilde H = \pmatrice{
      -\frac{((a+t_0r)(d+t_1s)-(b+t_0s)(c+t_1r))N_\phi}{N_\psi} & 0 & \frac{p'((c+t_1r)s-(d+t_1s)r)-q'((a+t_0r)s-(b+t_0s)r)}{N_\psi}\\
      0 & -\frac{N_\psi}{N_\phi} & \frac{p'((a+t_0r)s-(b+t_0s)r)+q'((c+t_1r)s-(d+t_1s)r)}{N_\psi}\\
      N_\phi & 0 & t'}$\;
     $B = R_\phi$\;
     \KwRet{$A,\tilde H,B$}
   \end{algorithme}
   Dans une implémentation pratique, on n'inverse pas la matrice $H$ pour obtenir les valeurs de $t_0$ et $t_1$, car les quantités susmentionnées se simplifient en
 \[\frac{\Delta_H(0,0)}{(r^2+s^2)S} = \frac{(as-br)(ar+bs)+(cs-dr)(cr+ds)}{(r^2+s^2)(as-br)}\]
 \[\frac{\Delta_H(0,0)}{(r^2+s^2)R} = \frac{(as-br)(ar+bs)+(cs-dr)(cr+ds)}{(r^2+s^2)(cs-dr)}\]
 De plus, en pratique, la condition $as-br=0$ n'est jamais vérifiée, on préfère donc la remplacer par une condition $as-br<cs-dr$.
