% contient la description des différentes possibilités pour traiter l'homographie unidirectionnelle au centre de la décomposition
%simon
\subsubsection{Séparation d'une homographie unidirectionelle}
\label{homobox_paragraph}
On considère une homographie  $h$ de la forme 
\begin{equation*}
h:(x,y)\mapsto \left(\frac{-bx}{1-ax},\frac{-y}{1-ax}\right)
\end{equation*}
(les résultats énoncés pour cette homographie seront valables pour toute homographie unidirectionnelle)

On peut décomposer cette homographie en deux applications $h_1 , h_2$
\begin{equation*}
h_1:(x,y) \mapsto \left(\frac{-bx}{1-ax},y\right)~~~~~~h_2:(x,y) \mapsto \left(x,\frac{-y}{1-ax}\right)
\end{equation*}
On obtient $h=h_1  \circ h_2$, ce qui nous donne le schéma suivant 
\begin{equation*}
f\longrightarrow f'=f\circ h_1 \longrightarrow f''=f'\circ h_2
\end{equation*}
Chacune de ces deux transformations ne modifie l'image que dans une seule direction, cela permet d'effectuer des opérations sur des signaux unidimentionnels . Cette méthode n'est en revanche pas séparable.\\ 
La première transformation est une homographie en une dimension que l'on doit réaliser sur chaque ligne.\\ %sur ?
La seconde est un zoom d'un facteur différent sur chaque colonne.

%a étoffer
\paragraph{Sous échantillonnage gaussien :}
Dans le cas d'un zoom gaussien on utilise la convolution $f*G_{d}$. Le paramètre $d$ doit être choisi tel que 
\begin{equation*}
z^2 c^2=c^2 + d^2     ~~~~~~~c= 0.8
\end{equation*}
On obtient donc la formule $d=c\sqrt{z^2 - 1}$, nous renvoyons à l'article \cite{morel2011sift} pour les détails de ce raisonnement et une justification de la valeur expérimentale de c.\\

\paragraph{Sous échantillonnage utilisant les images intégrale }
Soit $d>0$, on veut convoler une fonction d'une variable par une approximation d'une gaussienne d'écart type $\delta$.\\
Soit $g_n^d$ le noyau défini par 
\begin{equation*}
g_1^d(x)=\frac{1}{d}\mathds{1}_{]-\frac{d}{2},\frac{d}{2}[}(x) ~~~~~~~g_{n+1}^d= g_n^d * g_1^d
\end{equation*}
Si on pose $G_n^d(x)=\sqrt{n}g_n^d(\sqrt{n} x)$ on peut montrer que 
\begin{equation*}
\widehat{G_n^d}(\omega)\underset{n\rightarrow\infty}{\rightarrow} \exp\left(-\frac{\omega^2 d^2}{12}\right)
\end{equation*}
\begin{proof}
On a $\widehat{g_1^d}(\omega)=\text{sinc}\left(\frac{\omega d}{2}\right)~~$  donc $~~\widehat{g_n^d}(\omega)=\text{sinc}\left(\frac{\omega d}{2}\right)^n~~$ et 
$~~\widehat{G_n^d}(\omega)=\text{sinc}\left(\frac{\omega d}{2\sqrt{n}}\right)^n~~$\\
on obtient le résultat voulu en réalisant un développement limité de cette fonction 
\end{proof}
Si on définit $D_d$ l'opérateur de "dérivation discrète" par $D_d f(x)=\frac{f(x+\frac{d}{2})-f(x-\frac{d}{2})}{d}$ et on pose $F^{n+1}(x)= \int_{-\infty}^{x}F^{n}(y)dy$ où $F^{0}(x)= f(x)$, on a alors le lemme suivant 
\begin{prop} Pour toute fonction $f$ continue par morceau à support compact :
\begin{equation*}
(f*g_n^d)(y)=D_d ^n F^{(n)}(y)= \frac{1}{d^n}\underset{0 \le k\le n}{\sum} \binom{n}{k}(-1)^{k} F^{(n)}(y+\frac{(n-2k)d}{2})
\end{equation*}
\end{prop}
\begin{proof}
Pour toute fonction $f$ continue par morceau à support compact on a 
\begin{eqnarray*}
(f * g_1^d )(y)&=&\frac{1}{d} \int_{[y-\frac{d}{2},y+\frac{d}{2}]} f(x) dx\\
               &=&\frac{F^{(1)}(y+\frac{d}{2})-F^{(1)}(y-\frac{d}{2})}{d}\\
               &=&D_d F^{(1)}(y)
\end{eqnarray*}
On en déduit alors 
\begin{equation*}
(f*g_{n+1}^{d})=(f*g_1^d * g_{n}^{d})= ((D_d F^{(1)})*g_n )
\end{equation*}
Par récurrence on obtient la formule 
\begin{equation*}
(f*g_n^d)(y)=D_d ^n F^{(n)}(y)= \frac{1}{d^n}\underset{0 \le k\le n}{\sum} \binom{n}{k}(-1)^{k} F^{(n)}(y+\frac{(n-2k)d}{2})
\end{equation*}

\end{proof}
Dans nos algorithmes  on utilisera cette formule pour $n=3$ :
\begin{equation*}
(f*g_3)(y)=\frac{1}{d^3}(F^{(3)}(y+\frac{3d}{2})-3F^{(3)}(y+\frac{d}{2})+3F^{(3)}(y-\frac{d}{2})-F^{(3)}(y-\frac{3d}{2}))
\end{equation*}
On doit cependant calculer une valeur approchée des fonctions $F^{(k)}$  car on ne connait que les échantillons de la fonctions $f$.\\
On peut utiliser la méthode suivante on pose par récurrence
\begin{equation*}
F^{(n+1)}(k)=\underset{l\le k-1}{\sum}F^{(n)}(l)
\end{equation*}
On utilise ensuite une méthode d'interpolation par splines cubiques de Hermite afin de pouvoir évaluer cette fonction en des valeurs non entières.\\
Une autre méthode consiste à poser 

\begin{equation*}
F^{(0)} (x) =\underset{0\le k \le m-1}{\sum}f_{k} \mathds{1}_{[k,k+1[}(x)
\end{equation*}
$(f_k)_{k=0...m-1}$ sont les termes du signal .On calcule ensuite $F^{(n)}(x)=\int_{-\infty}^{x}F^{(n-1)}(y)dx$ on a par exemple
\begin{equation*}
F^{(1)}(x)=\underset{k\le \lfloor x\rfloor \wedge m~-1}{\sum}f_{k}~~+ f_{\lfloor x\rfloor}
(x-\lfloor x\rfloor)
\end{equation*}
Cette fonction est affine par morceau on peut démontrer la formule suivante par récurrence
\begin{eqnarray*}
F^{(n)}(x) &=& F^{(n)}(\lfloor x\rfloor \wedge m)~~+\mathds{1}_{[0,m[}(x) \underset{0\le k \le n-1}{\sum}F^{(k)}(\lfloor x \rfloor) \frac{(x-\lfloor x \rfloor)^{n-k}}{(n-k)!}\\
          &+&\mathds{1}_{[m,+\infty[}(x)\underset{1\le k \le n-1}{\sum}F^{(k)}(m) \frac{(x-m)^{n-1-k}}{(n-1-k)!}
\end{eqnarray*}
Où la valeur de $F^{(n)}$ se calcule par récurrence on a la relation $\forall k\le m$
\begin{equation*}
F^{(n)}(k+1)=F^{(n)}(k)+\underset{0\le l < n}{\sum} \frac{F^{(l)}(k)}{(n-l)!}
\end{equation*}
Comme dans la méthode précédente on doit calculer une composante constante par morceau afin d'avoir la valeur de $F^{(n)}$ aux entiers  ainsi qu'un terme polynomial de degré $n$ pour effectuer l'interpolation sur des valeurs non-entières.\\
Dans le cas $n=3$ on obtient les formules
\begin{eqnarray*}
F^{(3)}(x)&=&F^{(3)}(\lfloor x\rfloor \wedge m)~~+\mathds{1}_{[0,m[}(x)(x-\lfloor x \rfloor) \left(F^{(2)}(\lfloor x \rfloor)+ \frac{(x-\lfloor x \rfloor)}{2}\left(F^{(1)}(\lfloor x \rfloor)+\frac{(x-\lfloor x \rfloor)}{3} f_{\lfloor x \rfloor}\right)\right)\\
          &+&\mathds{1}_{[m,+\infty[}(x)(x-\lfloor x \rfloor) \left(F^{(2)}(m)+ \frac{(x-\lfloor x \rfloor)}{2}F^{(1)}(m)\right) \\
F^{(3)}(k)&=&  \underset{0\le l<k}{\sum}\left(F^{(2)}(l)+\frac{F^{(1)}(l)}{2}+\frac{f_{l}}{6} \right)  \\
F^{(2)}(k)&=&  \underset{0\le l<k}{\sum}\left(F^{(1)}(l)+\frac{f_{l}}{2} \right)  \\
F^{(1)}(k)&=&  \underset{0\le l<k}{\sum}f_{l} 
\end{eqnarray*}
Si on fixe $0\le l\le m-1$ et on pose
\begin{equation*}
P_l (x)=F^{3}(l) +(x-l) \left(F^{(2)}(l)+ \frac{(x-l)}{2}\left(F^{(1)}(l)+\frac{(x-l)}{3} f_{l}\right)\right)
\end{equation*}
On obtient alors
\begin{eqnarray*}
P_l (l) &=& F^{(3)}(l) \\
P_l (l+1) &=& F^{(3)}(l+1) \\
P_l '(l) &=& F^{(2)}(l) \\
P_l '(l+1) &=& F^{(2)}(l+1)
\end{eqnarray*}
On peut donc obtenir ces formules en utilisant une méthode d'interpolation de Hermite. La fonction $F^{(3)}$ est la spline de cubique passant par le point $F^{(3)}(k)$ en $k$ avec une dérivé égale à $F^{(2)}(k)$, elle est $\mathcal{C}^2$.\\
Cette méthode a cependant un défaut, l'image est initialement interpolée par une fonction constante par morceau, Si on pose $F_1$  est la spline d'ordre $1$ continue, telle que $F_{1}(k+\frac{1}{2})=f_k$ alors
\begin{prop} Si on pose $F_1$  est la spline d'ordre $1$ continue, telle que $F_{1}(k+\frac{1}{2})=f_k$ alors 
\begin{equation*}
( F_1 * g_n^d ) (y) = D_d^{(n)}D_1 F^{(n+1)}(y)
\end{equation*}
Où $F$ est l'interpolation constante par morceau de $(f)$
\end{prop}
\begin{proof}
On a
\begin{equation*}
F_1(x)=\underset{0\le k \le m-1}{\sum} f_k g_2^1 (x-k-\frac{1}{2})=(F^{(0)} *g_1^1 )(x)
\end{equation*}
On obtient alors 
\begin{equation*}
F_1*g_n^d=(F ^{(0)}*g_1^1)*g_n^d=(F ^{(0)}*g_n^d)*g_1^1=(D_d^n F ^{(n)})*g_1^1= D_d^n D_1 F^{(n+1)}
\end{equation*}
\end{proof}
L'interpolation supplémentaire peut donc être obtenue en évaluant $F^{(n+1)}$. La méthode de calcul est donc la même que dans le cas précédent, mais il faut utiliser l'intégrale quatrième de l'image dans la pratique.\\
Il est possible d'utiliser cette méthode pour obtenir une représentation plus régulière du signal de départ mais la courbe n'est pas  interpolante si $2\le n$.\\
%à faire mieux 
Afin d'implémenter cette méthode nous devons déterminer la valeur du paramètre $d$ en fonction du facteur de zoom local. Nous allons réutiliser les résultats du paragraphe précédent, car la fonction $g_3$ est une bonne approximation d'une gaussienne. L'écart type de $g_1$ est $\sigma_1=\frac{d}{\sqrt{12}}$ donc l'écart type $\sigma_3$ est donné par la formule $\sigma_3=\sqrt{3}\sigma_1=\frac{d}{2}$
donc $d=2c\sqrt{z^2 - 1}$.\\
Dans la pratique on utilisera la formule $d=2\sqrt{(0.8)^2 z^2 - (0.7)^2}$ car les images utilisées ne sont en général pas parfaites.\\

\paragraph{Splines cubiques de Hermite}

Il existe des méthodes d'interpolation plus efficace que celle bilinéaire, par exemple celle cubique. Elle peut être employée lors du traitement de l'homographie unidirectionnelle.

Soit $(x_0,x_1,x_2,x_3)$ et $(y_0,y_1,y_2,y_3)$ dans $\mb{R}^4$. On peut construire un polynôme $P$ de degré 3 tel que  
\begin{equation*}
P(x_1)=y_1~~~~P(x_2)=y_2~~~~P'(x_1)= \frac{y_2-y_0}{x_2 -x_0}~~~~P'(x_2)= \frac{y_3-y_1}{x_3 -x_1}
\end{equation*}

Il suffit ensuite pour avoir une valeur de la courbe en $x\in [x_1,x_2]$ d'évaluer $P(x)$.
