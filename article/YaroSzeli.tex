% contient la description des différentes méthodes pour traiter les rotations de la décomposition

Pour les deux rotations dans la décomposition, plusieurs méthodes sont possibles : en plus de la méthode de traitement des affinités multi-étapes \cite{szeliski2010high}, la méthode de Yaroslavsky \cite{unser1995convolution} est connue pour son efficacité à traiter les rotations. On cherchera donc à les comparer. La méthode multi-étapes n'étant qu'une décomposition d'une affinité en quatre opérations, la comparaison se fera aussi entre plusieurs implémentations de cette méthode, en faisant varier la manière d'interpoler (convolution par un \emph{raised cosine-weighted sinc}, interpolation par des splines).

Il est théoriquement incorrect d'utiliser des splines pour cette interpolation (on rappelle qu'on convole avec $h(\frac{\dot{}}{s})$ pour filtrer les fréquences voulues ; le support du filtre est adaptatif). Cependant, dans le cas des rotations, $u_{max}$ et $v_{max}$ valent 1, donc le paramètre $s$ n'a pas d'influence sur les deux premiers $\mathcal R$. Si de plus la rotation est d'angle faible, $r_v$ et $r_h$ sont proches de 1, donc $s$ a très peu d'influence. Donc dans ce dernier cas, convoler avec un filtre à support adaptatif (qui sera finalement un support fixé) ou interpoler par des b-splines ne change rien au niveau théorique.
