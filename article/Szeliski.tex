% contient une description du fonctionnement de Szeliski, pour préciser qu'on possède une méthode "parfaite" dans le cas des affinités
Les affinités sont un cas particulier d'homographie. Elles peuvent être traitées par une méthode multi-étape qui ne crée pratiquement aucun aliasing \cite{szeliski2010high}.

Le principe de cette méthode est de d'abord se ramener à des \emph{shear}-\emph{tilt}-translations selon une direction (affinités de la forme
$\pmatrice{a_0 & a_1 & t\\ 0 & 1 & 0}$ ou $\pmatrice{1 & 0 & 0\\ a_1 & a_0 & t}$). Ces opérations sont alors effectuées en trois étapes : un sur-échantillonnage dans la direction transverse, une opération \emph{shear}-\emph{tilt}-translation (modifié pour que l'opération soit celle voulue) effectuée de manière simple et un sous-échantillonnage dans la direction transverse.

Le principe de la méthode est en fait d'augmenter suffisamment le facteur de sur-échantillonnage pour que l'opération entre les réchantillonnage se fasse sans aliasing, i.e. sans repliement du spectre.

Les trois opérations sont faites via une convolution par un noyau d'interpolation, en l'occurrence une fonction de type \emph{raised cosine}.

On est ainsi capable d'effectuer toutes les affinités qui laissent inchangées une coordonnée (les \emph{shears}-\emph{tilts}-translations). Or une affinité se décompose toujours en deux de ces opérations, on peut donc toujours traiter une affinité de ce type, et ce sans aliasing.

En pratique, bien que cette méthode semble décomposer une affinité en 6 opérations élémentaires (3 pour chacun des 2 \emph{shears}-\emph{tilts}-translations), il est possible de condenser certains réchantillonnage et \emph{shears}-\emph{tilts}-translations, s'ils sont dans la même direction. La décomposition se réduit donc à 4 opérations élémentaires.

