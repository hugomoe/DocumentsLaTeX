\subsubsection*{Comparaison entre la méthode de Yaroslavsky et la méthode de traitement des affinités multi-étapes}

%A voir peut être dans les expériences ?
	Dans les deux paragraphes suivants, la \emph{RMSE} correspond à la norme L2 de la différence entre l'image initiale et l'image ayant subie une rotation, l'erreur L1 est la norme L1 de cette même différence, et l'erreur max est la norme sup de cette différence.\\

	Afin d'observer des différences significatives entre les différentes méthodes de rotations, on effectue 360 rotations de 1 degré (sur trop peu de rotations, les erreurs sont faibles). On obtient les résultats figure \ref{troiscentrotations}. On y voit que la méthode multi-étapes ne résiste pas à tant de rotations, mais que l'interpolation par les b-splines peut être substituée à la convolution par un \emph{raised cosine-weighted sinc} sans création d'\emph{aliasing}, car les rotations sont d'angles faible.
\label{pleinsderotations}
	
	Dans la décomposition géométrique d'une homographie, on ne fait que deux rotations. Il est donc plus pertinent de réaliser seulement 10 rotations de l'image lena.png de 36 degrés.\\

	On ne compare plus la méthode de traitement des affinités multi-étapes avec interpolation par b-spline car l'angle des rotations est cette fois trop grand pour considérer que $r_h$ et $r_v$ sont proches de 1.
	
	Les résultats sont sur la figure \ref{rotalena}. La méthode de traitement des affinités multi-étapes avec le \emph{raised cosine-weighted sinc} comme filtre d'interpolation et la méthode de Yaroslavsky semblent parfaites. Il reste donc à choisir entre ces deux dernières méthodes.\\

En terme de complexité, la méthode de Yaroslavsky est en $O(n log(n))$ (où $n$ est le nombre de pixel de l'image), tandis que la méthode de traitement des affinités multi-étapes est en $O(n)$. Ainsi pour des images de grande taille, il vaut mieux choisir la méthode de traitement des affinités multi-étapes.\\

Pour les expériences qui suivent, la méthode de traitement des affinités multi-étapes, avec le \emph{raised cosine-weighted sinc} comme filtre d'interpolation a été choisie pusiqu'elle est toujours dans les méthodes les plus rapides quelque soit la taille de l'image, et parce qu'elle donne des résultats excellents sur peu de rotations. 
On pourrait remplacer l'interpolation par convolution par une b-spline si on accordait moins d'importance au temps de calcul et si on avait une assurance que les rotations seront faibles (par exemple, si les homographies sont déjà presque unidirectionnelles), .

 \begin{figure}[h]
 \centering
   \subfigPDP{lena.png originale}{lena.png}
   \subfigPDP{lena.png après 360 rotations par interpolation linéaire}{linear_lena.png}
   \subfigPDP{lena.png après 360 rotations par la méthode de traitement des affinités multi-étapes  (filtre d'interpolation : \emph{raised cosine-weighted sinc}))}{raised-cosine_beta0-36_lena.png}
   \subfigPDP{lena.png après 360 rotations par la méthode de traitement des affinités multi-étapes (interpolation par b-spline d'ordre 3)}{b-spline_order3_lena.png}
   \subfigPDP{lena.png après 360 rotations par la méthode de traitement des affinités multi-étapes  (interpolation par b-spline d'ordre 9)}{b-spline_order9_double_lena.png}
   \subfigPDP{lena.png après 360 rotations par Yaroslasky}{lena_360_rotations_yaro.png}
  
 \subfigure[Erreurs des 360 rotations de 1 degré]{\begin{tabular}{|c|c|c|c|c|}
  \hline
  Méthode & \emph{RMSE}  & Erreur L1 & erreur max & durée (sec) \\
  \hline
  interpolation linéaire & 38.090 & 27.526 & 184.13 &  \bf{17.909}\\
  affinité multi-étape, b-spline d'ordre 9 & \bf{6.8430} &  \bf{4.0370} & \bf{86.855} &  6114.1\\
  affinité multi-étape, b-spline d'ordre 3 & 14.869 & 8.5208 & 158.04 & 1268.1\\
  Yaroslavsky & 14.187 & 7.7791 & 211.17 & 1839.9 \\
  affinité multi-étapes, \emph{raised cosine-weighted sinc} &  828.85 & 503.15 & 13638 & 778.42\\
  \hline
\end{tabular}} 
\caption{Effet de 360 rotations de 1 degré (cf. section \ref{pleinsderotations})}
\label{troiscentrotations}
 \end{figure}

 \begin{figure}[h]
   \centering
   \subfigPDP{lena.png originale}{lena.png}
   \subfigPDP{lena.png après dix rotations par interpolation linéaire }{linear_10rot_lena.png}
   \subfigPDP{lena.png après dix rotations par la méthode de traitement des affinités multi-étapes (filtre d'interpolation : \emph{raised cosine-weighted sinc})}{lena_10_rotations_szeli.png}
   \subfigPDP{lena.png après dix rotations par Yaroslasky }{lena_10_rotations_yaro.png}
    \subfigPDP{différence entre lena.png et lena.png après dix rotations par la méthode de traitement des affinités multi-étapes (filtre d'interpolation : \emph{raised cosine-weighted sinc})}{raised-cosine_beta0_36_10rot_lena_error.png}
   \subfigPDP{différence entre lena.png et lena.png après dix rotations par Yaroslasky }{lena_10_rotations_yaro_error.png}
 \subfigure[Erreurs des 10 rotations de $\frac{\pi}{5}$ radians.]{\begin{tabular}{|c|c|c|c|c|}
  \hline
  Méthode & \emph{RMSE}  & Erreur L1 & erreur max  \\
  \hline
  Interpolation linéaire & 14.722 & 8.5485 & 147.43  \\
  Yaroslavsky & 12.817 & 7.3749 & 159.84  \\
  affinité multi-étapes, \emph{raised cosine-weighted sinc} &   \bf{5.9791} & \bf{3.6243} & \bf{72.129} \\
  \hline
 \end{tabular}}
 \caption{Effet de dix rotations (cf. section \ref{pleinsderotations})}
 \label{rotalena}
 \end{figure}
