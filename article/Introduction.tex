% contient l'introduction
	La première partie de cet article (cf partie \ref{Exposition_du_probleme}) traite du problème des homographies en général et fait une présentation non-exhaustive des méthodes actuelles pour traiter les homographies. Elle présente aussi une méthode de traitement des affinité que nous utilisons dans notre méthode de traitement des homographies.

	La seconde partie (cf partie \ref{decomp_geo_hom} ) présente une interprétation géométrique d'une homographie en terme de mouvement de caméra. Elle permet de comprendre la théorie qui justifie notre méthode.

	La partie expérimentale (cf partie \ref{experiences}) montre les performances et le gain en qualité de notre méthode de traitement des homographies par rapport aux méthodes existantes (comme le Ripmap par exemple).

	Les pseudo-codes permettant de mettre en oeuvre notre méthode sont présents en annexe.
