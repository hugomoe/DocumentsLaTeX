PARTOUT
	référer les figure
	les reference peuvent disparaitre sans gêner la phrase
	mettre des captions et des labels sur toutes les figures
	remplacer tous les "homographie particulière" par "homographie unidirectionnelle"
	mettre des phrases dans les propositions et redéfinir les termes
	proposition de shmuel : mettre l'échelle dans les captions (pour connaitre la taille réelle)
	article mathématique : pas de texte libre (disons presque pas...?)
		soit : notation, remarque
	mettre le moins de texte libre possible
	la description d'un algo, si elle est courte, doit être dans le \caption


MAIL MEINHARDT CE WE
	commenté figure 33
	sections 2.2 et 2.3 sont peut être un peu trop succinctes (je crois pas mais bon je note quand meme...) 
	premiere phrase de 2.2,-> manque de transition
           % j'ai l'impression qu'il voulait parler des parties 3.2 et 3.3 ?

MAIL MOREL
	pour nous avoir si bien encadré: pour nous avoir si bien encadrés
	ou plus généralement à tout changement
	de point de vue: 
	ou plus généralement à tout changement
	de point de vue où la caméra tourne autour de son centre optique
	d'entre elle: d'entre elles
	il faut moyenner l'image de départ
	sur une certaine zone,: il faudrait calculer une moyenne pondérée de l'image par un filtre anti-aliasing sur une zone de taille suffisante,
	des coefficients qui représentent des zones
	entières de l'image,: des  moyennes locales de l'image à plusieurs échelles
	On suppose avoir une: On suppose que l'on dispose d'une
	Page 13, bas: en  principe tous les nombres d'un seul chiffre jusqu'à 9 sont écrits en lettres (à moins qu'ils appartiennent à une série comportant des nombres plus grands que 10).
	Dans le lemme 1 il n'est pas précisé ce que c'est que $H(X)$: il faut écrire : Soit $H$ un mouvement de caméra.
	De plus il faudrait donner un aspect plus formel à la terminologie, y compris en transformant la page 19 en une série de définitions. En particulier il me semble que vous avez deux définitions du mouvement de caméra, dont l'une est géométrique et l'autre en plus l'associe aux paramètres \theta, \phi, \psi. Il faut que ces définitions soient explicitées, sinon la proposition 2 par exemple ne s'appuie pas sur une définition formelle.
	un jeux: un jeu
	Lemme 2 : préciser les hypothèses: soit $h$ un mouvement  de caméra à point visé, alors...
	grâce au lemme précédent: toujours donner le numéro du lemme
	
	Dans la Proposition 3: la décomposition 3: la formule de décomposition (3) de la proposition 2
	
	sont des homographie: sont des homographies
	translations ont été omise: translations ont été omises
	unidimentionnels.: unidimensionnels
	la convolutions: la convolution
	Cette suite de fonction: Cette suite de fonctions
	la somme des opérateur de translation: la somme des opérateurs de translation
	est a ne par morceau: est a ne par morceaux
	Les erreures: Les erreurs
	le traitements de l'homographie: le traitement de l'homographie
	expériences réalisé: expériences réalisées
