% contient une exposition du problème (description d'une homographie, source du problème, intérêt du problème, problématique, raison pour laquelle on se bat dans la vie, etc...)



\label{pospb}
	Cet article présente une méthode permettant de déformer une image par une homographie en limitant l'apparition de défauts.
	
	\begin{Def}
 	Une homographie $h$ est une déformation du plan qui transforme les droites en droites ; elle est déterminée par 8 paramètres et s'écrit sous la forme :
	\[h : (x,y)\mapsto\left( \frac{ax+by+p}{rx+sy+t},\frac{cx+dy+q}{rx+sy+t}\right)\]
	\label{definition_homographie}
	\end{Def}
On peut voir l'effet d'une homographie figure \ref{effethom}.\\

 \begin{figure}
 
   \centering
   \subfigPDP{I(x,y)}{hom_before.png}
    \arrowPDP 
   \subfigPDP{I(h(x,y))}{hom_after.png}
   \caption{Effet d'une homographie (cf partie \ref{pospb})}
\label{effethom}
 \end{figure}


