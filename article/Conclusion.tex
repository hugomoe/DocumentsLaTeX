%ici une conclusion chouette

La décomposition géométrique permet ainsi une réalisation des homographies de meilleure qualité, et un contrôle plus rigoureux de l'\emph{aliasing} et du flou introduit. Les solutions proposées ici pour implémenter les différentes étapes ne sont que des exemples. Elles sont coûteuses en calcul bien que linéaires. 

On peut ainsi imaginer des approximations moins satisfaisantes mais beaucoup plus rapide, par exemple en utilisant une méthode triple intégrale pour les rotations. De même le traitements de l'homographie unidirectionnelle peut-être simplifié, par exemple en n'intégrant que sur les entiers et en réalisant une interpolation bilinéaire.

On a ainsi proposé une décomposition géométrique des homographies autour de laquelle de nombreux algorithmes peuvent s'articuler.

%probablement un peu trop emphatique, n'hésitez pas à y retoucher ! 
