% contient la description des différentes possibilités pour traiter l'homographie unidirectionnelle au centre de la décomposition
%simon
%Homobox refait, une relecture s'impose, manque encore les ref vers les pseudos code (elles sont en com)
%Je sais pas si on doit garder la partie sur l'interp de hermite
\subsubsection{Séparation d'une homographie unidirectionelle}
\label{homobox_paragraph}
On en déduit par la formule du paragraphe précédent (formule \ref{formule_decomposition_effective}) que le traitement d'une homographie générale peut se ramener au traitement d'une homographie  unidirectionnelle.\\
On considère une homographie  $h$ de la forme 
\begin{equation*}
h:(x,y)\mapsto \left(\frac{-bx}{1-ax},\frac{-y}{1-ax}\right)
\end{equation*}
(les résultats énoncés pour cette homographie seront valables pour toute homographie unidirectionnelle)

On peut décomposer cette homographie en deux applications $h_1 , h_2$
\begin{equation*}
h_1:(x,y) \mapsto \left(\frac{-bx}{1-ax},y\right)~~~~~~h_2:(x,y) \mapsto \left(x,\frac{-y}{1-ax}\right)
\end{equation*}
On obtient $h=h_1  \circ h_2$, ce qui nous donne le schéma suivant 
\begin{equation*}
f\longrightarrow f'=f\circ h_1 \longrightarrow f''=f'\circ h_2
\end{equation*}
Chacune de ces deux transformations ne modifie l'image que dans une seule direction, cela permet d'effectuer des opérations sur des signaux unidimentionnels. Les méthode de ré-échantillonnage sont donc plus simple ce qui permet un gain de rapidité car les calculs sont moins couteux. De plus les transformations $h_1,h_2$ peuvent être appliqué en ne réalisant que des filtrages horizontaux et verticaux, les direction de filtrage sont simples.\\ 
La première transformation est une homographie en une dimension que l'on doit réaliser sur chaque colonne.\\ 
La seconde est un zoom d'un facteur différent sur chaque ligne.\\
On peut voir sur la figure suivante (figure  \ref{image_separation_f14}) l' effet des transformations $h,h_1,h_2$.\\

\begin{figure}[h!]
\centering

\subfigure[identité]{
\includegraphics[scale =0.30]{damier.png}}
\subfigure[homographie unidirecionnelle $h=h_1 \circ h_2$]{\includegraphics[scale =0.30]{homo_unidirec_f14_2.png}}
\subfigure[transformation $h_1$ ]{\includegraphics[scale =0.30]{homo_unidirec_part_1_f14_2.png}}
\subfigure[transformation $h_2$]{\includegraphics[scale =0.30]{homo_unidirec_part_2_f14_2.png}}
\caption{Séparation des homographies unidirectionnelles (échelle $0.3$) }
\label{image_separation_f14}
\end{figure}

\paragraph{Sous échantillonnage gaussien :}
\label{zoom_gaussien}
Le zoom gaussien est une méthode de sous échantillonnage utilisant la convolutions $f*G_{d}$ , où $G_d$ est un noyaux gaussien d'écart type $d$. Dans nos algorithme on utilisera la méthode développée dans l'article "Sift is an invariant scale" (cf \cite{morel2011sift}).\\
Si $f$ est une image on suppose que $f$ peut s'écrire sous la forme $f=G_c' * f'$, où $f'$ est une image de résolution infinie. Le paramètre $c'$ est le facteur de flou gaussien idéal de l'image $f$. L'expérience montre que le facteur de flou gaussien idéal se situe autour de $c=0.8$ (cf \cite{morel2011sift}).\\
Soit $z\le 1$ posons $f''(x)=f'(zx)$, si l'on cherche à échantillonner la fonction  $x\mapsto f(zx)$,  on doit d'abord s'assurer qu'elle possède un facteur de flou gaussien égal à $0.8$ on sait que 
\begin{equation*}
(f''*G_{c})(x)=(f'*G_{cz})(zx)
\end{equation*}
Grâce à cette relation on peut en déduire la valeur de $d$ à utiliser, sachant que l'image de départ possède un flou gaussien de $c'$ on obtient
\begin{equation*}
(f*G_d)(zx)=(f'*G_c'*G_d)(zx)=(f'*G_{\sqrt{c'^2 + d^2}})(zx)
\end{equation*}
En identifiant ces deux expressions on obtient
\begin{equation}
d=\sqrt{c^2 z^2 - c'^2}
\label{formule_zoom_gaussien}
\end{equation}

Le paramètre $c'$ est difficile à déterminer en pratique, l'expérience montre que pour image "correctement échantillonnée" le facteur de zoom doit être pris égal à $0.7$. Cependant pour certaines images provenant de la photographie on peut prendre $c'=0.5$.\\

\paragraph{Ré-échantillonnage par les images intégrales }
\label{4Integral}
Dans ce paragraphe on présente une méthode permettant de ré-échantillonner une image 1D, cette méthode s'appuie sur une approximation du zoom gaussien (cf \ref{zoom_gaussien}) et utilise les images intégrales .\\


Soit $\Gcal_n^d$ le noyau définit par 
\begin{equation*}
\Gcal_1^d(x)=\frac{1}{d}\mathds{1}_{]-\frac{d}{2},\frac{d}{2}[}(x) ~~~~~~~\Gcal_{n+1}^d= \Gcal_n^d * \Gcal_1^d 
\label{formule_convol_n_int}
\end{equation*}
Cette suite de fonction vérifie la propriété suivante
\begin{prop}
La fonction $x\mapsto \sqrt{n}~\Gcal_n^d(\sqrt{n}~x)$ converge uniformément sur $\mathbb{R}$ vers une courbe gaussienne 
\end{prop}
Si $f$ est une fonction continue par morceau on définit $D_d$ l'opérateur de "dérivation discrète" par $D_d f(x)=\frac{f(x+\frac{d}{2})-f(x-\frac{d}{2})}{d}$  et on pose
\begin{equation*}
F_{n+1}(x)= \int_{-\infty}^{x}F_{n}(y)dy~~~~~~F_{0}(x)= f(x)
\end{equation*}
On a alors le lemme suivant 
\begin{prop} Pour toute fonction $f$ continue par morceau à support compact :
\begin{equation}
 (f*\Gcal_n^d)(y)=D_d^n F_{n}(y)= \frac{1}{d^n}\underset{0 \le k\le n}{\sum} \binom{n}{k}(-1)^{k} F_{n}(y+\frac{(n-2k)d}{2})
\end{equation}
\end{prop}
\begin{proof}
Pour toute fonction $f$ continue par morceau à support compact on a 
\begin{eqnarray*}
(f * \Gcal_1^d )(y)&=&\frac{1}{d} \int_{[y-\frac{d}{2},y+\frac{d}{2}]} f(x) dx\\
               &=&\frac{F_{1}(y+\frac{d}{2})-F_{1}(y-\frac{d}{2})}{d}\\
               &=&D_d F_{1}(y)
\end{eqnarray*}
On montre par récurrence que $ f*\Gcal_n^d=D_d^n F_{n}$.\\
La relation est donc vraie pour $n=1$ , si la propriété est vraie au rang $n$ alors
\begin{equation*}
f*\Gcal_{n+1}^{d}=(f * \Gcal_{n}^d) * \Gcal_{1}^{d}= (D_d^n F_{n})*\Gcal_1^d 
\end{equation*}
Par linéarité de la convolution
\begin{equation*}
(f*\Gcal_{n+1}^{d})= D_d^n (F_{n}*\Gcal_n^d) = D_d^{n+1} F_{n+1}
\end{equation*}
Comme $D_n^d$ est la somme des opérateur de translation
\begin{equation*}
f\mapsto f(\cdot+\frac{d}{2})~~~~~~f\mapsto f(\cdot-\frac{d}{2})
\end{equation*}
On peut développer $D_d^n$ à l'aide de la formule du binome de Newton car les deux opérateurs commutent, on obtient donc
\begin{equation*}
(f*\Gcal_n^d)(y) = \frac{1}{d^n}\underset{0 \le k\le n}{\sum} \binom{n}{k}(-1)^{k} F_{n}(y+\frac{(n-2k)d}{2})
\end{equation*}

\end{proof}
Dans nos algorithmes  on utilisera généralement cette propriété pour $n=3$ mais on calculera l'opérateur $D_d^n$ en faisant des différences successives (algorithme \ref{pseudo_code_convol_4_int}).\\ 


On doit cependant calculer une valeur approchée des fonctions $F_{k}$  car on ne connait que les échantillons de la fonctions $f$.\\
Si le signal discret possède $m$ valeurs non nulles $f_k,~~k=0...m-1$, on peut poser 
\begin{equation*}
F_{0} (x) =\underset{0\le k \le m-1}{\sum}f_{k} \mathds{1}_{[k,k+1[}(x)
\end{equation*}

$(f_k)_{k=0...m-1}$ sont les termes du signal .On calcule ensuite $F_{n}(x)=\int_{-\infty}^{x}F_{n-1}(y)dx$ on a par exemple
\begin{equation*}
F_{1}(x)=\underset{k\le \lfloor x\rfloor \wedge~-1}{\sum}f_{k}~~+ f_{\lfloor x\rfloor}
(x-\lfloor x\rfloor)
\end{equation*}
Cette fonction est affine par morceau on peut démontrer la formule suivante par récurrence
\begin{eqnarray*}
F_{n}(x) &=& F_{n}(\lfloor x\rfloor \wedge m)~~+\mathds{1}_{[0,m[}(x) \underset{0\le k \le n-1}{\sum}F_{k}(\lfloor x \rfloor) \frac{(x-\lfloor x \rfloor)^{n-k}}{(n-k)!}\\
          &+&\mathds{1}_{[m,+\infty[}(x)\underset{1\le k \le n-1}{\sum}F_{k}(m) \frac{(x-m)^{n-1-k}}{(n-1-k)!}
\end{eqnarray*}
Où la valeur de $F_{n}$ se calcule par récurrence on a la relation $\forall k \in \llbracket 0 ;m-1 \rrbracket$
\begin{equation*}
F_{n}(k+1)=F_{n}(k)+\underset{0\le l < n}{\sum} \frac{F_{l}(k)}{(n-l)!}
\end{equation*}
On doit calculer une composante constante par morceau afin d'avoir la valeur de $F_{n}$ aux entiers  ainsi qu'un terme polynomial de degré $n$ pour effectuer l'interpolation sur des valeurs non-entières.\\
Dans la pratique on a utilisé cette formule pour $n=4$, afin d'évaluer $F_4$ on procède de la façon suivante 
\begin{itemize}
\item Si $x\le 0$ alors $F_{n}=0$
\item Si $x\in ]0 , m[$ alors $F_{4}(x)-F_{4}(\lfloor x \rfloor)=P_{\lfloor x \rfloor}(x-\lfloor x \rfloor)$ où $P_k$ sont les polynômes de degrés $4$ définis par
\begin{equation*}
P_k (r) =r \left( F_{3}(k) +\frac{r}{2} \left(F_{2}(k)+ \frac{r}{3}\left(F_{1}(k)+\frac{r}{4} f_{k}\right)\right)\right)~~~~~k=0...m-1
\end{equation*}
\item Si $x\ge m$ alors $F_{4}(x)-F_{4}(m)=Q(x-m)$ où le polynôme $Q$ est définit par
\begin{equation*}
Q(r)=r \left(F_{3}(m)+\frac{r}{2} \left( F_{2}(m) + \frac{r}{3} F_1 (m)\right)\right)
\end{equation*}

\end{itemize}
Ces formules sont utilisées dans l'lgorithme permettant d'évaluer l'intégrale quatrième de l'image (algorithme \ref{pseudo_code_eval_4_int}).\\
Pour calculer la valeur $F_4$ aux entiers on peut applique la relation de récurrence suivante
\begin{eqnarray*}
F_{1}(k+1)&=&  F_{1}(k)+f_{k}  \\
F_{2}(k+1)&=&  F_{2}(k)+F_{1}(k)+\frac{f_{k}}{2}   \\
F_{3}(k+1)&=&  F_{3}(k)+F_{2}(k)+\frac{F_{1}(k)}{2}+\frac{f_{k}}{6}   \\
F_{4}(k+1)&=&  F_{4}(k)+F_{3}(k)+\frac{F_{2}(k)}{2}++\frac{F_{1}(k)}{6}+\frac{f_{k}}{24}  
\end{eqnarray*}
Ces formules sont utilisées dans le pseudo code (algorithme \ref{pseudo_code_built_4_int}).\\
Dans cette méthode on réalise un ré-échantillonnage du signal $(f_k)$ en l'interpolant d'abord par la fonction $F_{0}$, puis on ré-échantillonne en appliquant une convolution avec un noyaux $G_n^d$. On utilise dans nos algrithmes les fonctions $G_3^d$ car elles se sont de bonnes approximations de courbes gaussiennes.\\ %placer figure 
Dans cette méthode l'image est initialement interpolée par une fonction constante par morceau. Lorsque le paramètre $d$ est pris très petit le ré-échantillonnage $F_{0}*G_3^d$ est équivalent à la méthode du point le plus proche.\\
On peut cependant corriger ce problème

\begin{prop} Si on pose $\tilde{F}_0$  est la spline d'ordre $1$ continue, telle que $\tilde{F}_0(k+\frac{1}{2})=f_k$ alors 
\begin{equation*}
( \tilde{F}_0 * \Gcal_n^d ) (x) = D_d^{(n)}D_1 F_{n+1}(x)
\end{equation*}
Où $F$ est l'interpolation constante par morceau de $(f)$ 
\end{prop}
\begin{proof}
On a
\begin{equation*}
\tilde{F}(x)=\underset{0\le k \le m-1}{\sum} f_k \Gcal_2^1 (x-k-\frac{1}{2})=(F_{0} *\Gcal_1^1 )(x)
\end{equation*}
On obtient alors 
\begin{equation*}
\tilde{F}*\Gcal_n^d=(F_{0}*\Gcal_1^1)*\Gcal_n^d=(F_{0}*\Gcal_n^d)*\Gcal_1^1=(D_d^n F_{n})*\Gcal_1^1= D_d^n D_1 F_{n+1}
\end{equation*}
\end{proof}
L'interpolation supplémentaire peut donc être obtenue en évaluant $F_{n+1}$. La méthode de calcul est donc la même que dans le cas précédent, la convolution avec $\Gcal_3^d$ nécessite  l'utilisation l'intégrale quatrième de l'image dans la pratique.\\
Il est possible d'utiliser cette méthode pour obtenir une représentation plus régulière du signal de départ mais la courbe $F_0 * \Gcal_p^1~$ n'est pas  interpolante si $2\le p$.\\
%à faire mieux 
Afin d'implémenter cette méthode nous devons déterminer la valeur du paramètre $d$ en fonction du facteur de zoom local $z$ que l'on doit effectuer en ce point. Généralement nous prendrons $z$ égal à la dérivé de la transformation au point où l'on ré-échantillonne .\\ 
On réutilise les résultats du paragraphe précédent (cf \ref{zoom_gaussien}), car les  fonctions $\Gcal_3^d$ sont de bonnes approximations de courbes gaussiennes. L'écart type $\sigma$ de $\Gcal_3^d$ est donné par la formule 
\begin{equation*}
\sigma=\frac{d}{2}
\end{equation*}
Par la formule du zoom gaussien (formule\ref{formule_zoom_gaussien}) on en déduit que si $z\ge 1$ alors
\begin{equation*}
d=2\sqrt{c^2 z^2 - c'^2}~~~~~c=0.8~~~~c'=0.7
\end{equation*}
La valeur de $c'$ peut être choisi en fonction de l'image d'entrée, pour des images très nettes si l'on souhaite que le résultat final ne soit pas alliasé on doit prendre une valeur de $c'$ autour de $0.5$, pour la grande majorité des image que l'on a utilisé lors de nos expériences une valeur de $0.7$ était suffisante. Une valeur trop faible de $c'$ aboutit à un flou excessif.
