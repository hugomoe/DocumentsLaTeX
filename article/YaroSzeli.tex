% contient la description des différentes méthodes pour traiter les rotations de la décomposition

Pour les deux rotations dans la décomposition, plusieures méthodes sont possibles : la méthode de Yaroslavsky \cite{unser1995convolution} et la méthode de traitement des affinités multi-étapes \cite{szeliski2010high}.

\subsubsection{Méthode de Yaroslavsky}

La méthode de Yaroslavsky consiste à décomposer une rotation en trois \emph{shears}.

Sous forme matricielle une rotation s'écrit de manière générale :
\begin{equation*}
	H=\begin{pmatrix}
	cos \theta&-sin \theta\\sin \theta&cos \theta
	\end{pmatrix}
	\end{equation*}

La décomposition en trois \emph{shears} est la suivante :
\begin{equation*}
	H=\begin{pmatrix}
	cos \theta&-sin \theta\\sin \theta&cos \theta
	\end{pmatrix}=\begin{pmatrix}
	1&-tan \frac{\theta}{2}\\0&1
	\end{pmatrix}\begin{pmatrix}
	1&0\\sin \theta&1
	\end{pmatrix}\begin{pmatrix}
	1&-tan \frac{\theta}{2}\\0&1
	\end{pmatrix}
	\end{equation*}

	Pour faire la rotation complète on fait chaque \emph{shear} par Fourier et on recentre l'image à chaque étape.
	En pratique on se ramène toujours à des rotations d'angle dans $[\frac{-\pi}{4},\frac{\pi}{4}]$ (on peut en effet faire des rotations d'angle $\pi$ et $\frac{\pi}{2}$ de manière exacte...)  


\subsubsection{Comparaison entre la méthode de Yaroslavsky et la méthode de traitement des affinités multi-étapes}

%A voir peut être dans les expériences ?
Afin de choisir la méthode de rotation on fait 10 rotations de l'image "Lena" de $\frac{\pi}{5}$. Les résultats sont sur la figure \ref{rotalena}. Visuellement les rotations semblent parfaites (mais si on observe bien il y a un peu plus d'effet Gibbs pour la méthode de traitement des affinités multi-étapes que pour Yaroslavsky). Si on fait la différence entre l'image orignal et l'image qui a subie dix rotations par la méthode de traitement des affinités multi-étapes on a en une $RMS=5.979119$, une norme $L1=3.624343$, et une norme $sup=72.128845$. Entre l'image initiale et l'image qui à subie dix rotations par la méthode de Yaroslavsky on a une $RMS=12.817376$, une norme $L1=7.374875$, et une norme $sup=159.840759$. \\

En terme de complexité, la méthode de Yaroslavsky est en $O(n log(n))$ (où $n$ est le nombre de pixel de l'image), tandis que la méthode de traitement des affinités multi-étapes est en $O(n)$. Ainsi pour des images de grande taille, il vaut mieux choisir la méthode de traitement des affinités multi-étapes.

 \begin{figure}
 
   \centering
   \subfigPDP{Lena originale}{lena.png}
   \subfigPDP{Lena après dix rotations par Szeliski}{lena_10_rotations_szeli.png}
   \subfigPDP{Lena après dix rotations par Yaroslasky}{lena_10_rotations_yaro.png}
   \caption{Effet de dix rotations}
\label{rotalena}
 \end{figure}
