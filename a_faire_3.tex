Note correction morel

p.7 these definitions, à préciser (en fait il n’y a pas de definition a proprement parler dans le texte actuel).

p.11 Lemma 1 :  Il faut absolument faire une figure correspondant aux notations précédentes et à cette démonstration, où apparaissent tous  les points et vecteurs: $X$, $H(X)$, $C_0$, $w$, $\delta$, $F$.

p.13-14 : préciser à quoi font référence les schémas figures 7 et 8

p.15 : in Remarks pour le deuxième item (qui n’est plus un item) :  Je ne comprends pas ce que vous voulez dire, expliquer mieux ou enlever.

Relire 2.2 (j’ai essayer de présenter de manière plus claire, j’ai un peu changer l’ordre et relu)

Relire 2.3 et Conclusion (je n’ai rien toucher) 
Faut-il supprimer la partie 2.3 (qui est redondante avec les experiments il me semble)

p.34 : Il n'est pas possible de juger du succès de la méthode sans disposer d'une "vérité terrain" qui peut être obtenue en faisant un très gros zoom in (d'un facteur 5 ou 10) par zero-padding, suivi de l'application de l'homographie par interpolation bilinéaire, et suivi par le zoom arrière du facteur inverse, par un noyau gaussien d'écart type 0.8 n, où n est  le facteur de zoom.

p.49 : J'avoue que je ne comprends pas cette explication. 
je l’ai supprimer, cela ne me semble pas important`

(globalement les histoires d’origines ne sont pas clair du tout)

p.50 : Je ne trouve pas la définition de $\mu$ et $\nu$. Donner une référence à l'équation les définissant, ou le définir maintenant.

p.50 : There is in fact an origin given with images ?? Je ne comprends pas ce que vous voulez dire.

(globalement l’histoire des column / row n’est pas compréhensible…)

p.53 : En fait la référence à la section 2.2.1 est absolument insuffisante. Il faut dire à quelles formules précises vous vous référez et pour cela il faut avoir numéroté ces formules.

p.60 : $y$ et $a_i$ ne font pas partie des Data
