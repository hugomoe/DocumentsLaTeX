\documentclass[a4paper,11pt]{article}

\usepackage[utf8]{inputenc}
\usepackage[T1]{fontenc}
\usepackage[french]{babel}
\usepackage{amsmath, amsfonts, amssymb}
\usepackage{stmaryrd}
\usepackage[french,onelanguage,boxed]{algorithm2e}
\usepackage{graphicx}
\usepackage{color}
% \usepackage{fullpage}



\newcommand{\affinite}[1]{\left(\begin{array}{c c|c} #1 \end{array}\right)}
\newcommand{\pmatrice}[1]{\begin{pmatrix} #1 \end{pmatrix}}

\newcommand{\pgcd}{\wedge}
\newcommand{\ppcm}{\vee}

\newcommand{\red}[1]{\textcolor{red}{#1}}
\newcommand{\blue}[1]{\textcolor{blue}{#1}}
\newcommand{\green}[1]{\textcolor{green}{#1}}

\newcommand{\Chi}{\raisebox{.40ex}{\ensuremath{\chi}}}

\newcommand{\bloc}[2]{\ \\ \underline{#1 :}\\ #2 \ \\}



\newenvironment{algorithme}{\begin{algorithm}[H]}{\end{algorithm}}



\DeclareMathOperator{\supp}{supp}
\DeclareMathOperator{\Sp}{Sp}
\DeclareMathOperator{\Tr}{Tr}

\title{Pseudo-codes}

\begin{document}
	\maketitle
	\tableofcontents
	% contient l'abstract

Cet article présente une méthode de ré-échantillonnage d'une image par une homographie en décomposant cette dernière en plusieurs transformations, affines et homographiques. En passant par des affinités qu'on peut traiter de manière bien plus efficace que les homographies, on se ramène donc au traitement d'une famille d'homographies particulières : ces homographies seront unidirectionnelles, au sens où elles conservent les verticales, et pourront se décomposer en deux applications unidimensionnelles (qui traite chaque ligne ou chaque colonne indépendamment). Les résultats obtenus sont comparés avec les méthodes déjà existantes, tel que le RipMap.

	\section*{Introduction}
		% contient l'introduction
	Cet article traite du problème des homographies qui sont des applications projectives correspondant en fait à des mouvements de caméra. Elles ont donc de nombreuses applications en traitement d'image comme par exemple le \emph{texturing} (dans le jeux vidéo, l'animation, etc...), dans la constitution de panorama \cite{brown2007automatic} ou encore le recalage précis d'une suite d'images prises du même endroit (ce qui est la première étape de plusieurs méthodes de super-résolution).

	Le traitement des homographie n'est pas un problème trivial puisqu'il s'agit de faire des zooms différents pour chaque points de l'image, et qu'il faut éviter le flou et l'aliasing. Une méthode naïve de traitement des homographies comme décrite première partie produit par exemple beaucoup d'aliasing. Dans cet article nous présentons une méthodes permettant de réduire au maximum le flou et l'aliasing.

	La première partie de cet article (cf partie \ref{Exposition_du_probleme}) traite du problème des homographies en général et fait une présentation non-exhaustive des méthodes actuelles pour traiter les homographies. Elle présente aussi une méthode de traitement des affinité que nous utilisons dans notre méthode de traitement des homographies. 

	La seconde partie (cf partie \ref{decomp_geo_hom} ) présente une interprétation géométrique d'une homographie en terme de mouvement de caméra. Elle permet de comprendre la théorie qui justifie notre méthode.

	La partie expérimentale (cf partie \ref{experiences}) montre les performances et le gain en qualité de notre méthode de traitement des homographies par rapport aux méthodes existantes (comme le Ripmap par exemple).

	Les pseudo-codes permettant de mettre en oeuvre notre méthode sont présents en annexe.

	\section{/Exposition du problème/}
		\subsection{Position du problème}
			% contient une exposition du problème (description d'une homographie, source du problème, intérêt du problème, problématique, raison pour laquelle on se bat dans la vie, etc...)




	Nous nous intéressons dans cet article a l'élaboration d'une méthode la plus exacte possible permettant de déformer une image par une fonction homographique. \\
 	Une fonction homographique est une déformation du plan qui transforme les droites en droites ; elle est déterminée par 8 paramètres et s'écrit sous la forme :
	\[H : (x,y)\mapsto\left( \frac{ax+by+p}{rx+sy+t},\frac{cx+dy+q}{rx+sy+t}\right)\] 
On peut voir l'effet d'un homographie figure \ref{effethom}.\\

 \begin{figure}
 
   \centering
   \subfigPDP{I(x,y)}{hom_before.png}
    \arrowPDP 
   \subfigPDP{I(H(x,y))}{hom_after.png}
   \caption{Effet d'une homographie}
\label{effethom}
 \end{figure}

	Les homographies correspondent en fait à des changements de perspective. Elles ont donc de nombreuses applications en traitement d'image comme par exemple le texturing (dans le jeux vidéo,l'animation,etc...), dans la constitution de panorama ou encore le recalage précis de suite d'images prises du même endroit (ce qui est la première étape de plusieurs méthodes de super-résolution).



		\subsection{Méthodes existantes}
			% contient une explication du fonctionnement du Mip Map et du Rip Map, et des schémas explicatifs. En bref, un résumé de l'article de Williams
% parler éventuellement de la méthode naïve, ou d'autres méthodes actuelles qui ne fonctionnent pas
  Afin de traiter des homographies plusieures méthodes ont été développées. Cette partie présente de manière non-exhaustive les méthodes déjà existantes (méthode naïve, Mipmap et Ripmap). Elles seront comparées avec notre nouvelle méthode dans les expériences.


\ssse{Méthode naïve}

On peut imaginer une méthode naïve : on utilise simplement la formule de l'homographie. Ainsi pour chaque point de l'image d'arrivée, on peut calculer la zone de l'image de départ correspondant via l'homographie inverse. On se heurte à un premier problème, en effet on ne tombe pas précisément sur un pixel de l'image de départ mais entre plusieurs pixels. 

On peut par exemple prendre la couleur du pixel le plus proche. Il apparait alors de l'\emph{aliasing} : en effet les pixels qu'on a modélisés par des points ont visuellement une épaisseur. Ainsi quand il y a localement un \emph{zoom-out} il faut moyenner l'image de départ sur une certaine zone.

%image aliasé à fond

\begin{figure}[h!]
\centering
\includegraphics[scale=0.5]{imagereciproque.jpg}
\caption{Image réciproque d'un pixel}
\end{figure}

En pratique on utilise plutôt le Mipmap. C'est une méthode utilisée en texturing, présentée par Williams en 1983 \cite{williams1983pyramidal}.

\ssse{Présentation du Mipmap}

Le principe est de précalculer des coefficients qui représentent des zones entières de l'image, pour pouvoir ensuite calculer en temps réel une homographie. 

Pour cela on choisit de supposer que l'image est carrée de taille une puissance de 2 (quitte à faire un premier zoom). On calcule ensuite la valeur de certains carrés de l'image de taille une puissance de 2.
Le Mipmap est donc représenté par une suite d'images chacune deux fois plus petite que la précédente. Ainsi c'est un suite de \emph{zoom-out} de l'image d'origine de facteur une puissance de 2.

\begin{figure}[h!]
\centering
\caption{Un exemple de Mipmap}
\includegraphics[scale=0.4]{MipMap_real} %scale=0.4 ça fait vachement petit, non ?
\end{figure}

Quand on cherche la valeur d'un pixel de l'image d'arrivée, on approche la zone de l'image de départ à laquelle il correspond à l'aide des différentielles de l'homographie inverse. On obtient alors un parallélogramme, qu'on essaye d'approcher avec des carrés. %ou qui s'approxime par des carrés

%image de parallélogramme avec les différentielles, avec au mieux schéma image de départ / arrivé qui explique la correspondance entre un pixel est un parallélogramme

On appelle distance d'un pixel la taille du carré choisi pour l'approximation (car plus un pixel est "loin", plus les carrés qui l'approchent sont grands). 

On suppose avoir une formule qui nous donne la distance d'un point quelconque. La géométrie du Mipmap ne permettant que des distances puissance de 2, on fait une approximation trilinéaire : 

\begin{itemize}
  \item d'une part, on fait une interpolation linéaire entre deux étages dont les profondeurs encadrent la distance.
  \item d'autre part, dans chaque étage du Mipmap, on fait une interpolation bilinéaire entre les quatre carrés qui encadrent le pixel.
\end{itemize}

\begin{figure}[h!]
\centering
\caption{Schéma d'interpolation trilinéaire}
\includegraphics[scale=0.5]{intertrilineaire.jpg}
\end{figure}

\ssse{La fonction de distance}

Toutes les fonctions de distance dépendent de la différentielle de l'homographie inverse.
On note $(u,v)$ les coordonnées dans l'image d'origine et $(x,y)$ celles dans la fenêtre d'arrivée.

Le choix de la distance est crucial car si $d$ est trop grand, l'image est inutilement floutée (on parle d'\emph{over-blurring}), et si $d$ est trop petit l'image est aliasée.

Il est aisé de calculer les coefficients des dérivées partielles. On note $H$ l'homographie inverse.

%re image prgm, avec les dérivé partielle indiqués

% On a jugé de la performance des méthodes "à vue". On compte à terme l'évaluer sur des cosinus/sinus.

On note $\dd_{(x,y)} H$ la différentielle de $H$ en $(x,y)$.

On note $(u,v)=H(x,y)$.

\sssse{Méthode du déterminant}
$$D(x,y) = \sqrt{\det \txt{d}_{(x,y)} H}$$
Cette formule est justifiée car le déterminant de $\txt{d}_{(x,y)} H$ est l'aire du parallélogramme, donc un carré de côté $\sqrt{\det \txt{d}_{(x,y)} H}$ est de même aire que le parallélogramme qu'il approche.

\begin{figure}[h!]
\centering
\includegraphics[scale=0.5]{methode_determinant.jpg}
\end{figure}


Le résultat est relativement satisfaisant.

%schéma avec un ârallélogramme et le carré correspondant

\sssse{Méthode du plus grand côté}
$$ D(x,y) = \max \left(\sqrt{\left(\frac{\dr u}{\dr x}\right)^2 + \left(\frac{\dr v}{\dr x}\right)^2},\sqrt{\left(\frac{\dr u}{\dr y}\right)^2 + \left(\frac{\dr v}{\dr y}\right)^2}\right)$$
On prend le plus grand côté du parallélogramme comme côté du carré. Cette formule viens d'un article de Heckbert \cite{heckbert1983texture}.
% On prend le plus grand côté du parallélogramme comme côté du carré \cite{heckbert1983texture}. Le résultat est bon.

\begin{figure}[h!]
\centering
\includegraphics[scale=0.5]{methode_plus_grand_cote.jpg}
\end{figure}


Ce résultat est satisfaisant.

%schéma avec un ârallélogramme et le carré correspondant

\sssse{Méthode des diagonales}
$$D(x,y) = \max \left( \sqrt{\left(\frac{\dr v}{\dr x}-\frac{\dr  u}{\dr  x}\right)^2+\left(\frac{\dr v}{\dr y}-\frac{\dr  u}{\dr y}\right)^2}, \sqrt{\left(\frac{\dr u}{\dr x}+\frac{\dr  v}{\dr  x}\right)^2+\left(\frac{\dr u}{\dr y}+\frac{\dr  v}{\dr  y}\right)^2} \right)$$
On prend la plus grande diagonale du parallélogramme. Ici l'image est trop floue.

\begin{figure}[h!]
\centering
\includegraphics[scale=0.5]{methode_grande_diagonale.jpg}
\end{figure}

%schéma avec un parallélogramme et le carré correspondant

\sssse{Méthode des valeurs singulières}

On peut décomposer toute matrice $A$ en trois matrices $A = PDQ^{-1}$ où $P$ et $Q$ sont orthogonales et $D$ est diagonale.

Pour cela on diagonalise $AA^t$, et $A^tA$, on prend $P$ et $Q$ des vecteurs propres normalisés de ces matrices, et $D$ la racine de leur diagonalisée. Il faut veiller à ce que $P$ et $Q$ correspondent bien \cite{abdi2007singular}.

Les coefficients de $D$ sont appelés valeurs singulières de $A$. On prend $D(x,y)$ le maximum des deux valeurs singulières de $A$.


\begin{figure}[h!]
\centering
\includegraphics[scale=0.5]{methode_valeur_singuliere.jpg}
\end{figure}

On a une interprétation géométrique comme un mouvement de caméra de ces valeurs \cite{morel2009asift}.

\sssse{Conclusion}

Les méthodes du déterminant, du plus grand côté du parallélogramme et du maximum des valeurs singulières sont satisfaisantes.

On a une préférence pour le plus grand côté et la valeur singulière, en effet le déterminant semble introduire un peu plus d'\emph{aliasing}.

\ssse{Algorithme amélioré : Le Ripmap}

Une des grandes faiblesses est l'isotropie du Mipmap : il ne privilégie aucune direction. Ainsi si le parallélogramme à approcher est un rectangle très plat, il n'y a pas d'approximation raisonnable à l'aide d'un carré.

Pour tenter de résoudre ce problème on utilise un Ripmap \cite{akenine2008real}. C'est en fait un Mipmap où l'on a aussi calculé la moyenne des pixels de tous les rectangles dont les côtés sont des puissances de deux.

Ainsi la fonction de distance ne renvoie plus une seule valeur mais une pour chaque côté du rectangle. On réalise alors une interpolation quadrilinéaire (bilinéaire entre les étages et bilinéaire dans chacun d'eux).%bibilinéaire ? sérieusement ?



%un vrai ripmap, que je ferais moi
\begin{figure}[h!]
\centering
\caption{Un exemple de Ripmap}
\includegraphics[scale=0.4]{Ripmap_real}
\end{figure}


%schéma d'interpolations pour le ripmap, avec 4 images est l'endroit ou le point tombe dans chacun
\begin{figure}[h!]
\centering
\caption{Schéma d'interpolation quadrilinéaire}
\includegraphics[scale=0.5]{interbibilineaire.jpg}
\end{figure}


On a utilisé le plus petit rectangle qui contient le parallélogramme. Ainsi la formule est :
$$D(x,y) = \left( \left|\frac{\dr u}{\dr x}\right|+\left|\frac{\dr u}{\dr y}\right|,\left|\frac{\dr v}{\dr x}\right|+\left|\frac{\dr v}{\dr y}\right|\right)$$
On a de plus décalé les points pour que le rectangle considéré soit bien celui qui contient le parallélogramme.

\begin{figure}[h!]
\centering
\includegraphics[scale=0.5]{methode_distance_ripmap.jpg}
\end{figure}

Cela améliore certes la méthode, mais ne résout pas par exemple le cas d'un rectangle fin en diagonale.

		\subsection{Cas des affinités}
			% contient une description du fonctionnement de Szeliski, pour préciser qu'on possède une méthode "parfaite" dans le cas des affinités

%Le traitement par décomposition des homographies utilise une méthode de traitement des affinités. Elle est présentée dans la partie ci-dessous.

Affinities are a particular case of homographies for which there exists a efficient anti-aliasing method, and which will be used later to resample all homographies.

\subsection{Principle}
	
	This multi-pass resampling method has been introduced by Szeliski, Winder et Uyttendaele \cite{szeliski2010high}. It has been thought to minimize apparition of aliasing.

	%Le principe de cette méthode est d'abord de se ramener à des affinités qui n'agissent que selon une direction (affinités de la forme $\pmatrice{a_0 & a_1 & t\\ 0 & 1 & 0}$ ou $\pmatrice{1 & 0 & 0\\ a_1 & a_0 & t}$). Par abus de langage, on parlera de \emph{shear}, bien qu'il s'agisse de la composée d'un cisaillement, d'une dilatation unidirectionnelle et d'une translation, tous dans la même direction. Chacun des \emph{shears} est ensuite effectué en trois étapes : un sur-échantillonnage dans la direction transverse, un \emph{shear} et un sous-échantillonnage dans la direction transverse.
	
	The principle is to bring back the problem to a particular case of affine map, called shear, which only act towards one direction (their matrix form is $\pmatrice{a_0 & a_1 & t\\ 0 & 1 & 0}$ ou $\pmatrice{1 & 0 & 0\\ a_1 & a_0 & t}$). Each shear is computed in three steps : an oversampling, a shear and a subsampling.
	
	%Le principe de la méthode est en fait de choisir le facteur de sur-échantillonnage suffisamment grand pour que l'opération globale se fasse sans \emph{aliasing}, i.e. sans repliement du spectre (figures \ref{szeliski_decompoNaive} et \ref{szeliski_decompoSzeliski}).
	
	The aim is to choose the oversampling factor large enough so that the global operation avoids aliasing (figures \ref{szeliski_decompoNaive} et \ref{szeliski_decompoSzeliski}).
	\label{label_des_deux_prems_fig_de_szeli}
	\begin{figure}
		\centering
		\subfigure[Spectrum of the initial image]{\includegraphics[scale=0.5]{szeliski_naiveEntree.png}}
		\subfigure[Spectrum of the final image : aliasing]{\includegraphics[scale=0.5]{szeliski_naiveSortie.png}}
		\caption{Spectrum of a naive resampling by a shear (cf. partie \ref{label_des_deux_prems_fig_de_szeli})}
		\label{szeliski_decompoNaive}
	\end{figure}
		
	\begin{figure}
		\centering
		\subfigure[Spectrum of the initial image]{\includegraphics[scale=0.5]{szeliski_szeliskiEntree.png}}
		\subfigure[Spectrum of the image after oversampling]{\includegraphics[scale=0.5]{szeliski_szeliskiSurechantillonnage.png}}
		\subfigure[Spectrum of the image after an oversampling and a shear]{\includegraphics[scale=0.5]{szeliski_szeliskiShear.png}}
		\subfigure[Spectrum of the final image (after an oversampling, a shear and a subsampling, coupled with a low-pass filter)]{\includegraphics[scale=0.5]{szeliski_szeliskiSortie.png}}
		\caption{Spectrum of an image, resampled by a shear with the multi-pass resampling method (cf. section \ref{label_des_deux_prems_fig_de_szeli}). Experiment figure \ref{experiments_decompoSzeliski_sinc} }
		\label{szeliski_decompoSzeliski}
	\end{figure}
	
	%Ces trois opérations sont toutes des \emph{shears} (éventuellement réduits à une dilatation) et seront nommées $\mathcal R$ ; selon qu'elles modifient l'image verticalement (en laissant l'abscisse inchangée) ou horizontalement (en laissant l'ordonnée inchangée), on parlera de $\mathcal R_v$ ou de $\mathcal R_h$ . Ces opérations $\mathcal R$ sont réalisées par une convolution par un noyau d'interpolation, en l'occurrence une fonction de type \emph{raised cosine-weighted sinc}, pour pouvoir atténuer, voire supprimer, les fréquences au-delà d'un certain seuil.
	
	Those three operations are all shears (eventually reduced to a dilatation) and will be called $\mathcal R$ ; depending on what direction they modify (vertically or horizontally), 	they are called $\mathcal R_v$ or $\mathcal R_h$. Those operations $\mathcal R$ are done by convolving with a interpolation kernel, a raised cosine-weighted sinc, to extenuate or delete frequences beyond a threshold.
	
	\begin{equation}
	h : x \mapsto \sinc(\frac{x}{T})\frac{\cos(\frac{\pi\beta x}{T})}{1-\frac{4\beta^2x^2}{T^2}}
	\label{szeliski_definition_raisedCosineWeightedSinc}
	\end{equation}
	The period $T$ is equal to one (the size of a pixel).
	The parameter $\beta$, called roll-off factor, is a crucial choice. If $\beta$ is around $0.36$ (which the parameter used for most of the experiments in appendix), the support of $h$ can be reasonably approximate as contained in $[-4,4]$, and so the convolution reduce to a sum of nine terms (figure \ref{szeliski_plotRaisedCosine}), as long as the support is not modified.
	
	
%	Le paramètre $\beta$, intitulé \emph{roll-off factor}, peut varier. En prenant $\beta$ autour de $0.36$ (paramètre utilisé pour la plupart des expériences en annexe), on peut considérer que le support de $h$ est compris dans $[-4,4]$, et donc réduire la convolution à une somme d'au plus neuf termes (figure \ref{szeliski_plotRaisedCosine}), tant que le support du filtre n'est pas modifié.
	\label{label_figure_dom_reel_fourier_jt}
	\begin{figure}
		\centering
		\subfigure[Space domain]{\includegraphics[scale=0.3]{raisedCosine-weightedSinc.jpg}}
		\subfigure[Frequency domain]{\includegraphics[scale=0.3]{raisedCosine-weightedSinc_spectrum.jpg}}
		\caption{Representations of the \emph{raised cosine-weighted sinc} filter with different parameters $\beta$ (cf. section \ref{label_figure_dom_reel_fourier_jt})}
		\label{szeliski_plotRaisedCosine}
	\end{figure}
	
	In practice, the support of the filter vary, depending on the zoom factor $s$, hence the formula to convolve with $h$ (here horizontally)
	\begin{equation}
	img_f[i][j] = \displaystyle{\sum_k}h\left(\frac{a_0i+a_1j+t-k}{s}\right)img[k][j]
	\label{formule_convolution_discrete}
	\end{equation}
	where $img_f$ is the output image and $img$ the input image.
	
	
%	En pratique, le support du filtre est adaptatif, et dépend d'un facteur de zoom $s$, d'où la formule de convolution par $h$ (ici convolution selon les abscisses) :
%	où $img_f$ est l'image finale et $img$ l'image initiale.
	
	Multiplying by $s$ the size of support of support (by calling $h(\frac{x}{s})$ instead of $h(x)$, with $s\geq 1$) divide by $s$ the bandwidth of this filter. This protect against aliasing during downsamplings. The coefficient $s$ also depend on the \emph{maximal preserved frequencies} $u_{max}$ and $v_{max}$ which will be presented later.
	
%	En multipliant par $s$ la taille du support du filtre (en appelant $h(\frac{x}{s})$ au lieu de $h(x)$, avec $s\geq 1$), on divise par $s$ la largeur de la bande passante de ce filtre. Cela permet, lors d'un sur-échantillonnage, d'avoir une protection contre l'\emph{aliasing}. Le coefficient $s$ dépend aussi des \emph{fréquences conservées maximales} $u_{max}$ et $v_{max}$ qui seront définies et expliquées plus loin.
	
	The algorithms \ref{szeliski_rh} and \ref{szeliski_rv} dealing with the $\mathcal R_h$ and $\mathcal R_v$ operations are obtained by applying this formula (\ref{formule_convolution_discrete}) for every $i$ and $j$ in the output image. \label{szeliski_rv_rh_section}
	
%	En appliquant la formule précédente pour toutes les valeurs de $i$ et $j$ de l'image d'arrivée, on obtient les algorithmes \ref{szeliski_rh} et \ref{szeliski_rv} traitant les opérations $\mathcal R_h$ et $\mathcal R_v$. \label{szeliski_rv_rh_section}
	
	So it is possible to compute, in three steps $\mathcal R$ and without aliasing, any shear. Since an affinity can be decomposed in two shears (proposition \ref{propositionDecompositionAffinite}), any affinity can be process without aliasing (algorithme \ref{szeliski_affine}).
	
%	On est ainsi capable d'effectuer, en trois étapes $\mathcal R$ et sans \emph{aliasing}, tous les \emph{shears}. Or une affinité se décompose toujours en deux \emph{shears} (proposition \ref{propositionDecompositionAffinite}), on peut donc toujours traiter une affinité quelconque, et ce, sans \emph{aliasing} (algorithme \ref{szeliski_affine}).

	\begin{thm}
	The multi-pass resampling method \cite{szeliski2010high} prevent aliasing while preserving the spectrum. %SHMUEL RELIT STP, je ne savas pas traduire
%	Le méthode multi-étapes \cite{szeliski2010high} empêche le repliement du spectre tout en conservant le maximum du spectre.
	\end{thm}
	\begin{proof}
	The effect of the different steps on the spectrum is presented figure \ref{szeliski_decompoSzeliski} and experimentally verified figure \ref{experiments_decompoSzeliski_sinc}.
	
	The downsampling is processed without aliasing by expanding the support of the filter $h$.
	\end{proof}
	
	%\begin{proof}
%	L'effet des différentes étapes sur le spectre est présenté en figure \ref{szeliski_decompoSzeliski} et vérifié en pratique figure \ref{experiments_decompoSzeliski_sinc}.
	
%	Le sous-échantillonnage est effectué sans repliement grâce à un élargissement du support du filtre $h$ (support adaptatif).
	%\end{proof}
	
	\emph{A priori} this method decompose any affinity in six steps $\mathcal R$ (three per shears). In fact it is possible to merge some $\mathcal R$ if they act in the same direction. The decomposition can then be reduced to four stapes $\mathcal R$ : a vertical up-sampling, an horizontal shear, a vertical shear and finally an horizontal downsampling.
	
%	Cette méthode semble donc décomposer une affinité en six opérations $\mathcal R$ (trois pour chacun des deux \emph{shears}) ; en pratique, il est possible de réunir certains $\mathcal R$, s'ils sont dans la même direction. La décomposition peut alors se réduire à quatre opérations $\mathcal R$ : un sur-échantillonnage vertical, un \emph{shear} horizontal, un \emph{shear} vertical puis enfin un sous-échantillonnage horizontal.
	
\subsection{Steps of the algorithm}
	\label{szeliski_affine_section}
	
	The details of the processing of the steps of the multi-pass resampling algorithm for affinity are described in this section. The corresponding algorithm can be found in appendix (algorithms \ref{szeliski_rh}, \ref{szeliski_rv} and \ref{szeliski_affine}).
	
	Let $A$ be the matrix of the inverse affinity of the one supposed to be processed. Let's denote

\[A = \pmatrice{a_{00} & a_{01} & t_0\\ a_{10} & a_{11} & t_1}\]

%	On détaille donc ici les étapes de l'algorithme de traitement d'une affinité par cette méthode multi-étape. Le pseudo-code correspondant est présent en annexe (algorithmes \ref{szeliski_rh}, \ref{szeliski_rv} et \ref{szeliski_affine}).
	
%	On suppose avoir reçu en entrée une image à modifier et la matrice $A$ correspondant à l'affinité inverse de celle à effectuer : l'antécédent du point $(i,j)$ de l'image de sortie se situe donc en $A(i,j)$ sur l'image d'entrée.
	
%	On notera
	
	\subsubsection{Eventual transposition}
		\label{szeliski_transpoOpt_section}
		
		The $\mathcal R$ can compressed the image on a few pixels and then expand it (e.g. when $A$ is rotation of angle $\frac{\pi}{2}$), this is called the bottleneck problem \cite{wolberg1990digital}.

%	Les $\mathcal R$ peuvent, si l'affinité est par exemple une rotation d'angle $\frac{\pi}{2}$, compresser l'image sur peu de pixels puis tenter de la dilater ; cet effet est un \emph{bottleneck problem} déjà connu \cite{wolberg1990digital}.
		
		To prevent this the image and mateix can be transposed. Let's denote
		
	%	On effectue donc éventuellement une transposition (de l'image et de la matrice) pour éviter cet effet : en notant
		\[\hat a_{00} = \frac{a_{00}}{\sqrt{a_{00}^2+a_{11}^2}}\]
		\[\hat a_{01} = \frac{a_{01}}{\sqrt{a_{00}^2+a_{11}^2}}\]
		\[\hat a_{10} = \frac{a_{10}}{\sqrt{a_{10}^2+a_{11}^2}}\]
		\[\hat a_{11} = \frac{a_{11}}{\sqrt{a_{10}^2+a_{11}^2}}\]
		%on transpose dans le cas où $|\hat a_{00}|+|\hat a_{11}|<|\hat a_{01}|+|\hat a_{10}|$.
		the transposition is done when $|\hat a_{00}|+|\hat a_{11}|<|\hat a_{01}|+|\hat a_{10}|$.
		It corresponds to the algorithm \ref{szeliski_transpoOpt}.
		%Cette transposition correspond à l'algorithme \ref{szeliski_transpoOpt}.
		
	\subsubsection{Decomposition of $A$}
		\label{szeliski_decompositionDeA_section}
		\label{szeliski_frequencesMax_section}
		
		The \emph{maximal preserved frequency}  $u_{max}$ and $v_{max}$ are the frequencies beyond which the spectrum will be erased by the application of $A$. They are defined in the frequency domain by the figure \ref{uMax_vMax}. Frenquencies beyond them can be filtered without altering the output spectrum. These definitions assume that the spectrum has been normalized on the square $[-1,1]^2$.
		
%		On introduit les fréquences conservées maximales $u_{max}$, $v_{max}$ : ce sont les fréquences, horizontales et verticales, au delà desquelles les fréquences seront effacées par l'application de $A$ ; elles sont donc définies dans le domaine de Fourier par la figure \ref{uMax_vMax}. Les fréquences au-delà de $u_{max}$ horizontalement et de $v_{max}$ verticalement peuvent donc être filtrées, cela ne changera pas le spectre de sortie. Ces définitions se font après après avoir ramené le spectre sur le carré $[-1,1]^2$.
		
		\begin{figure}
		\centering
		\subfigure[Spectrum of the input image]{\includegraphics[scale=0.5]{uMax_vMax_spectreEntree.png}}
		\subfigure[Spectrum of the output image]{\includegraphics[scale=0.5]{uMax_vMax_spectreSortie.png}}
		\subfigure[Spectrum of the input image]{\includegraphics[scale=0.5]{uMax_vMax_spectreUtile.png}}
		\caption{Frequencies preserved by the affinity. The dark areas corresponds to frequencies erased}
		\label{uMax_vMax}
		\end{figure}
		
 
		 $u_{max}$ and $v_{max}$ can be found by finding the intersection of the input spectrum (that is to say the square $[-1,1]^2$) and the inverse image of the square by $^t\!A$ (because doing  $^t\!A$ in the frequency domain is equivalent to do $A^{-1}$ in the spatial domain ; as before $A$ is the inverse of the affinity applied to the input image). The algorithms \ref{szeliski_intersectionsVerticales} and \ref{szeliski_intersectionsHorizontales} make the intersection of a segment (which bounds are $(U_1,V_1)$ and $(U_2,V_2)$) and one of the sides of the square $[-1,1]^2$. Those two algorithms are used in the algorithm \ref{pseudo-code_umax_vmax} to compute $u_{max}$ and $v_{max}$.


		%En pratique, $u_{max}$ et $v_{max}$ s'obtiennent en intersectant le spectre d'entrée (le carré $[-1,1]^2$) et l'image réciproque du carré $[-1,1]^2$ par $^t\!A$ (qui correspond à l'opération dans Fourier ; comme précédemment $A$ désigne l'inverse de l'affinité qu'on applique à l'image). Les algorithmes \ref{szeliski_intersectionsVerticales} et \ref{szeliski_intersectionsHorizontales} réalisent l'intersection d'un segment (d'extrémités $(U_1,V_1)$ et $(U_2,V_2)$) et d'un côté du carré $[-1,1]^2$. L'algorithme \ref{pseudo-code_umax_vmax} s'appuie sur ces algorithmes pour calculer $u_{max}$ et $v_{max}$.
		
		From $A$'s coefficients and the values of $u_{max}$ and $v_{max}$, the coefficients of the vertical upsampling and the horizontal one can be deduced \cite{szeliski2010high} as follows,
		\[r_v \geq \max (1,|a_{01}|u_{max}+\min (1,|a_{11}|v_{max}))\]
		\[r_h \geq \max (1,|a_{10}/a_{11}|r_vv_{max}+\min (1,|b_0|u_{max})).\]

		%À partir des coefficients de $A$ et des valeurs de $u_{max}$ et $v_{max}$, on peut en déduire les coefficients de sur-échantillonnage verticaux et horizontaux nécessaires \cite{szeliski2010high} :
		%\[r_v \geq \max (1,|a_{01}|u_{max}+\min (1,|a_{11}|v_{max}))\]
		%\[r_h \geq \max (1,|a_{10}/a_{11}|r_vv_{max}+\min (1,|b_0|u_{max}))\]

		By geometric condiderations, one can always reduces the upsamplings to $r_h \leq 3$ and $r_v \leq 3$. Indeed, in both unidirectional mappings  ( (those splitted into three $\mathcal R$), $r_h$ and $r_v$ can be reduced to three, thanks to the filtering which attenuates the frequencies beyond $u_{max}$ and $v_{max}$ respectively. The figure \ref{rvleq3} presents that geometric reasoning for a mapping on columns (which is a mapping on rows in the frequencie domain).

		%Pour des raisons géométriques, on peut toujours réduire les sur-échantillonnages à $r_h \leq 3$ et $r_v \leq 3$. En effet, pour chacune des deux opérations unidirectionnelles (celles qui seront décomposées en trois $\mathcal R$), le $r_h$ (respectivement $r_v$) peut être réduit à 3, grâce au filtrage qui atténue les fréquences au delà de $u_{max}$ (respectivement $v_{max}$). La figure \ref{rvleq3} présente ce raisonnement géométrique pour une opération sur les colonnes (qui se traduit par une opération sur les lignes dans le domaine de Fourier).
		
		\begin{figure}
		\centering
		\subfigure[$r_v$ not bounded]{\includegraphics[width=120mm]{rvleq3_notrvleq3.png}}
		\subfigure[$r_v$ bounded by 3 ; the fact the horizontals spacings are unchanged has been used]{\includegraphics[width=120mm]{rvleq3_rveq3.png}}
		\caption{Reduction of $r_v$ down to $r_v \leq 3$. Beyond $v_{max}$, the frequencies are filtered. $s$ is the prameter of the filter which reduces the bandwidth. Then the $r_v$ needed to avoid aliasing can be reduced. $r_v$ is the ratio of the length of the dotted rectangle (which is the spectrum after the upsampling) and that of the square (which is the initial spectrum)}
		\label{rvleq3}
		\end{figure}


		\begin{prop}
		\label{propositionDecompositionAffinite}
		Let $A$ be an affine transform given with it's projective form, $A = \pmatrice{a_{00} & a_{01} & t_0\\ a_{10} & a_{11} & t_1\\ 0&0&1}$.
		%Soit $A$ une affinité donnée sous forme projective : $A = \pmatrice{a_{00} & a_{01} & t_0\\ a_{10} & a_{11} & t_1\\ 0&0&1}$.
		
		$A$ can be split as follows,
		%$A$ se décompose de la manière suivante :
		\[
			A = 
			\pmatrice{1 & 0 & 0\\ 0 & \frac{a_{11}}{r_v} & 0\\ 0 & 0 & 1}
			\pmatrice{\frac{b_{0}}{r_h} & \frac{a_{01}}{r_v} & t_2\\ 0 & 1 & 0\\ 0 & 0 & 1}
			\pmatrice{1 & 0 & 0\\ \frac{a_{10}r_v}{a_{11}r_h} & r_v & \frac{t_1r_v}{a_{11}}\\ 0 & 0 & 1}
			\pmatrice{r_h & 0 & 0\\ 0 & 1 & 0\\ 0 & 0 & 1}
		\]
where $b_0 = a_{00} - \frac{a_{01}a_{10}}{a_{11}};$ et : $t_2 = t_0 - \frac{a_{01}t_1}{a_{11}}$.
\end{prop}
	\subsubsection{The $\mathcal R$ mappings}
	%\subsubsection{Applications des $\mathcal R$}
		
		All that remains is to do the four $\mathcal R$ mapping.
		%Il ne reste qu'à appliquer les quatre opérations $\mathcal R$.
		
		For each ones, the first step is to keep in memory all the values of $h(\frac{k+\varphi}{s})$ whith
		%Pour chacune, on commence par stocker les différentes valeurs de $h(\frac{k+\varphi}{s})$ où :
		
\begin{itemize}
		\item $\varphi \in \frac{l}{2^b}$ with $l \in \rrbracket0,2^b-1\llbracket with $b$ the number of precision bit needed.
		\item $k$ is an integer such that $\frac{k}{s}$ is in the support of $h$ ($h(x)$ being nearly zero if $|x|$ is large enough, one assumes that $h$ is compactly supported).
		\end{itemize}
		Thus all the values of $h$ used for the convolution can be easily found.



%\begin{itemize}
%		\item $\varphi$ parcourt $2^b$ nombres rationnels de $[0,1]$ avec $b$ le nombre de bit de précision voulu
%		\item $k$ parcourt les entiers tels que $\frac{k}{s}$ soit dans le support de $h$ ($h(x)$ étant presque nul quand $|x|$ est assez grand, on suppose %$h$ à support compact)
%		\end{itemize}
%		On a ainsi accès rapidement à toutes les valeurs de $h$ qui peuvent être nécessaires à la convolution.

	\section{/Exposition de la(des) solution(s)/}
		\subsection{Décomposition d'une homographie en mouvement de caméra}
			The new method of homographies resampling is based on the decomposition of an homographie which can be seen as camera moves. The section below presents that decompostion.


%La nouvelle méthode de traitement des homographies repose sur la décomposition d'une homographie qui permet d'interpréter cette dernière en terme de mouvement de caméra. La partie ci-dessous présente cette décomposition.

\ssse{Modelisation of a camera mouvement}
%\ssse{Modélisation de mouvement de caméra}
\label{mouv_de_camera}

An \emph{a priori} special case of homographies $h$ that can be viewed as an ideal camera movement is studied there. It will be shown afterward the it is a general case.
%On étudie ici un cas a priori particulier d'homographie $h$ : les homographies que l'on peut interpréter comme un mouvement de caméra idéale. On montrera par la suite que c'est en fait un cas général.

Assume that the filmed scene is a plane. This assume that the filmed scene does not have any relief, or that it is negligible with respect to the distance between the plan and the camera, so that the relief is not seen. The figure \ref{shmdecomp} illustrates the modelisation used of the ideal camera. Therefore the ideal camera can be seen as the projection of a plane on another plan by going throughout the focus $F$, without taking all the lens and electronic systems that are in real camera into account.

%Nous modéliserons la situation en supposant que la scène filmée est plane. Cela suppose que l'on filme une surface soit sans aucun relief, soit avec un relief négligeable devant la distance à la caméra, afin qu'il ne soit pas perceptible. La figure \ref{shmdecomp} illustre la modélisation utilisée pour la caméra idéale. La caméra idéale se modélise donc par la projection d'un plan sur un autre en passant par un foyer $F$, en négligeant les lentilles ou les dispositifs correcteurs présents dans les caméras réelles.

\begin{figure}[h!]

\centering
\includegraphics[width=10cm]{shema_decomp.png}
\caption{Illustration of a camera mouvement $(X_v =0)$ (In order to be clearer the translations have not been shown. $F$ is the focus of the camera, the red plane is the image plane of the camera.  One point of the image plane is obtained by the projection of a point of the plane $(x,y)$ through the point $F$. (cf. part \ref{mouv_de_camera})}
\label{shmdecomp}
\end{figure}

Let's consider the vector space $\mathbb{R}^{3}$  of $O$ the origin $(0,0,0)$  and $(\xbf_0,\ybf_0,\zbf_0)$ the standard basis.
%On se place dans l'espace vectoriel $\mathbb{R}^{3}$ on note $O$ l'origine $(0,0,0)$ et $(\xbf_0,\ybf_0,\zbf_0)$ la base canonique.
\begin{itemize}
\item Let $F$ and $C_0$ be two diffrents points in $\mathbb{R}^{3}$.
\item Let $\mathcal{P}$ be the affine plane through $C_0$ and of normal vector $\overrightarrow{FC_0}$.
\item Let $\wbf$ be the vector $\frac{\overrightarrow{FC_0}}{|| \overrightarrow{FC_0}||}$
\item Let $\delta$ be the distance $|| \overrightarrow{FC_0}||$
\end{itemize}
\begin{Def}
The projective mapping $H$, is the mapping which receives a point $X$ in $\mathbb{R}^{3}$ and return point of intersection of the straight line $(XF)$ and the plane $\mathcal{P}$. $H$ depends on $(F,\wbf,\delta)$.
%L'application projective $H$, est l'application qui à un point $X$ de $\mathbb{R}^{3}$ associe le point d'intersection entre la droite $(XF)$ et le plan $\mathcal{P}$. $H$ dépend du triplet $(F,\wbf,\delta)$.
\end{Def}
\begin{remarques}
\begin{itemize}
\item $F$ is the position of the focus of the camera and the plane $\mathcal{P}$ is the plane of the camera on which the image is projected. The optical axis of the camera is the straight line $(FC_0)$ of direction vector $\wbf$, $\delta$ is the the distance between the focus and the screen.
\item A is a point in $\mathbb{R}^3$  the mapping $H$ return the point of $\mathcal{P}$ corresponding to it's image troughout the camera.
\end{itemize}
\end{remarques}
Let $\mathcal{P}'$ be the affine plane of $\mathbb{R}^{3}$ containing $F$ and parallel to the plane $\mathcal{P}$.
\begin{lem}
$H$ is defined on $\mathbb{R}^3 \setminus \mathcal{P}'$ and,
\begin{equation}
H(X) = C_0 +  \delta \frac{\overrightarrow{XF}-(\overrightarrow{XF}\cdot \wbf )\wbf}{\wbf \cdot \overrightarrow{XF}.} 
\label{formul_lem_app_proj}
\end{equation}
\label{lem_app_proj}
\end{lem}
\begin{proof}
If $X\in \mathbb{R}^3 \setminus \{F\}$ the straight ligne $(XF)$ and have an intersection point with $\mathcal{P}$ if and only if it is not parallel to  $\mathcal{P}$ so if and only if $X\in \mathbb{R}^3 \setminus \mathcal{P}'$. In that case there exist $t_X\in \mathcal{R}$ such that
\begin{equation*}
H(X)=X+t_{X}\overrightarrow{XF}.
\end{equation*}
As $H(X)\in P_{2}$ then
\begin{equation*}
\overrightarrow{FH(X)}\cdot \wbf =\delta.
\end{equation*}
By using the formula of $H(X)$ one can obtain
\begin{equation*}
t_{X}=1+\frac{\delta}{\wbf \cdot \overrightarrow{XF}},
\end{equation*}
 It then can be deduced that
\begin{equation*}
H(X) = C_0 +  \delta \frac{\overrightarrow{XF}-(\overrightarrow{XF}\cdot \wbf )\wbf}{\wbf \cdot \overrightarrow{XF}}.
\end{equation*}
\end{proof}
\begin{remarque}
The plane $\mathcal{P}'$ is the blindspot of the camera, in the case of a real camera that blindspot should be all what is behind the camera.
\end{remarque}
Let $P$ be the plan $(O,\xbf_0,\ybf_0)$.
\begin{Def} Let's call target point the intersection point of the straight line $(FC_0)$ and the plane $P$ if it exists. The target point exist if and only if $(FC_0)$ is not parallel to $\mathcal{P}$. When it exists it is called $X_v$ and then
\begin{equation*}
X_v=F-\wbf \delta'~~~~~~\delta'=\frac{\overrightarrow{OF}\cdot \zbf_0}{\wbf \cdot \zbf_0}
\label{formule_point_vise}
\end{equation*}
\label{point_vise}
\end{Def}
\begin{remarque}
The targer point is the point $P$ targetted by the camera, the camera moves are so that the camera is always targetting that point. It is possible that a camera does not have a target point, in that case the point is at infinity, the camera targets the horizon.
\end{remarque}
Assume for what's next the the target point exists.
\begin{Def}
The projective planar mapping $H^*$ related to $H$ is the restriction of $H$ to $P\setminus (\mathcal{P}'\cap P)$ 
If $\wbf \perp P $ then $H^*$ is defined on $P$ otherwise $H^*$ is defined on $P\setminus D$ where $D$ is the straight line
\begin{equation*}
D=\left\{ X\in P | \overrightarrow{XF}\cdot \wbf = 0\right\}.
\end{equation*}
\end{Def}
\begin{remarque}
\begin{itemize}
\item Let's say that the filmed scene is close enough to a plane so that any relief is negligible and can be interpreted as the plane $P$ . Then from a point $X\in P$ of the filmed scene, the mapping $H^*$ return the point in $P_2$ which is it's image troughout the camera.
\item The straight line $D'=\{ X \in \mathcal{P} | \overrightarrow{XF} \cdot \zbf_0 =0 \}$ is called the horizon of $H^*$.
\end{itemize}

\end{remarque}
Let's give to the affine plan $\mathcal{P}$ a direct normalized basis $(C,\ubf,\vbf)$ 
\begin{Def}
 The homography related with $H^*$ in the basis $(C,\ubf,\vbf)$ is the mapping
\begin{equation}
h : (x,y)  \mapsto \left( \overrightarrow{CH^*(X)}\cdot \ubf , \overrightarrow{CH^*(X)}\cdot \vbf \right)
\label{formule_homographie_H}
\end{equation}


où $X=x~\xbf_0 + y~\ybf_0 $.
\label{def_homographie_H}
\end{Def}
Let $D,D'$ be the sraight lines in $\mathbb{R}^2$ obtained when $P$ and $\mathcal{P}$ have the basis $(O,\xbf,\ybf)$ and $(C,\ubf,\vbf)$ then $h:\mathbb{R}^2  \setminus D \mapsto \mathbb{R}^2  \setminus D'$ is a bijective mapping.
\begin{remarque}
From a point $X\in P$ of coordinates $(x,y)$  the mapping $h$ return the coordinates of the point $H^*(X)$ of the numeric image coming from the camera. 
\end{remarque}
The orientation of the basis $(\ubf,\vbf,\wbf)$ with respect to the basis $(\xbf_0,\ybf_0,\zbf_0)$ can be defined by the three Euler anglespeut être définie grâce aux trois angles d'Euler $(\phi , \theta ,\psi )$ (figure \ref{img_angles})
\begin{itemize}
\item $\phi$ est is the first rotation around the axis $(X_v ,\zbf_0)$
\item $\theta$ is the second rotation around to the axis $(X_v , \zbf_0 )$
\item $\psi$ est is the third rotation around the axis $(X_v , \wbf )$
\end{itemize}
On note $\cbf= \left (\overrightarrow{C_0 C}\cdot \ubf , \overrightarrow{C_0 C}\cdot \vbf \right)$ et $\xbf_v=\left( \overrightarrow{O X_v}\cdot \ubf , \overrightarrow{O X_v}\cdot \vbf \right )$.\\
\begin{prop}
The homography $h$ can be factorized as follows
 
\begin{equation}
h = \tau_{\cbf}   \circ R_{\psi} \circ z_{\frac{\delta}{\delta'}} \circ h_{\theta,\delta'} \circ R_{\phi} \circ \tau_{\xbf_v}
\label{formul_decomp}
\end{equation}
 $R_{\alpha}$ is the rotation is the rotation of angle $\alpha$,$z_\lambda$ is the zoom of factor $\lambda$, $\tau_\ybf$ and the translation along the vector $-\ybf$ and $h_{\theta,\delta'}$ is the undirectional homography (cf. definition \ref{homo_uni_direc})
\begin{equation}
h_{\theta,\delta'}(x,y)=\left(\frac{-cos(\theta)x}{1-\frac{sin(\theta)}{\delta'}x} ,\frac{-y}{1-\frac{sin(\theta)}{\delta'}x}\right)
\label{mise_perspective}
\end{equation}
\label{prop_decomp}
\end{prop}
\begin{Def}
A unidirectional homography is a mapping $h:\mathbb{R}^{2} \ra \mathbb{R}^{2}$ defined by
\begin{equation*}
h(x,y)=\left ( \frac{ax+p}{rx+t} , \frac{cy+p}{rx+t} \right)
\end{equation*}
with $a,p,c,q,r,t$ some reals numbers.
\label{homo_uni_direc}
\end{Def}
\begin{figure}[h!]
\centering
\subfigure[rotation of angle $\phi$]{\includegraphics[width=5cm]{graphe1.jpg}}
\subfigure[rotation of angle $\theta$]{\includegraphics[width=5cm]{graphe2.jpg}}
\caption{(cf part \ref{figure_de_rotations_18})}
\label{img_angles}
\end{figure}

\begin{figure}[h!]
\centering
\includegraphics[width=5cm]{graphe3.jpg}
\caption{self rotation (cf part \ref{figure_de_rotations_18})}
\label{decompgeo_rotationPropre}
\end{figure}





\begin{proof}
Using formula (\ref{formule_homographie_H}), definition (\ref{def_homographie_H}) and the formula (\ref{formul_lem_app_proj}) of the lemma (\ref{lem_app_proj}), we have
\begin{equation*}
h((x,y))=\left( \delta \frac{\overrightarrow{XF}\cdot \ubf}{\overrightarrow{XF} \cdot \wbf} +\overrightarrow{CC_0 } \cdot \ubf, \delta \frac{\overrightarrow{XF}\cdot \vbf}{\overrightarrow{XF} \cdot  \wbf}+\overrightarrow{CC_0 }\cdot \vbf \right)
\end{equation*}
where $X=x \xbf_0 + y \ybf_0 $.\\

By introducing the target point $X_v$ (definition \ref{point_vise} formula \ref{formule_point_vise}) and the translation $\tau_c$ we have
\begin{equation}
(\tau_\cbf^{-1} \circ h)((x,y)) = \left ( -\delta \frac{\overrightarrow{X_vX}\cdot \ubf}{\delta' -\overrightarrow{X_vX} \cdot \wbf},-\delta \frac{\overrightarrow{X_vX}\cdot \vbf}{\delta' -\overrightarrow{X_vX} \cdot \wbf} \right)
\label{decomp_formul_intermediaire_1}
\end{equation}


Intermediate basis $(\xbf_1 ,\ybf_1 ,\zbf_1)$  $(\xbf_2 ,\ybf_2 ,\zbf_2)$ can be defined, in order to decompose the three rotations $\phi,\theta,\psi$ (figures \ref{img_angles} and \ref{decompgeo_rotationPropre} ).

Then we have the relations
\begin{equation*}
\ubf=\cos(\psi)\xbf_{2}+\sin(\psi)\ybf_{2} , \vbf=-\sin(\psi)\xbf_{2}+\cos(\psi)\ybf_{2} \text{ et } \wbf=\zbf_2.
\end{equation*}

Denoting $R_{s}$ the rotation of angle $s$, we obtained, by using the formula (\ref{decomp_formul_intermediaire_1}) 
\begin{equation*}
(\tau_{\cbf}^{-1} \circ h)((x,y)) = R_{\psi}\left(\frac{\delta \overrightarrow{X_v X}\cdot \xbf_{2} }{\delta'-\zbf_2 \cdot \overrightarrow{X_v X}},\frac{\delta \overrightarrow{X_v X}\cdot \ybf_{2}}{\delta'-\zbf_2 \cdot \overrightarrow{X_v X}}  \right).
\end{equation*}
Let $i$ be the mapping such that $i(x,y)=x \xbf_0 + y \ybf_0$. Since $i(\xbf_v)=\overrightarrow{O X_v}$ we have
\begin{equation*}
(R_{\psi}^{-1} \circ \tau_{\cbf}^{-1}  \circ h)((x,y))=\delta \left(\frac{-i(\tau_{\xbf_v} ((x,y)))\cdot \xbf_{2} }{\delta'-\zbf_2 \cdot i(\tau_{\xbf_v} ((x,y)))},\frac{-i(\tau_{\xbf_v} ((x,y)))\cdot \ybf_{2}}{\delta'-\zbf_2 \cdot i(\tau_{\xbf_v} ((x,y)))}  \right) 
\end{equation*}

Since $\zbf_{2}=cos(\theta)\zbf_{1}+sin(\theta)\xbf_{1}$, $\xbf_{2}=cos(\theta)\xbf_{1}-sin(\theta)\zbf_{1}$ (figure \ref{img_angles}) and $\zbf_{1}\perp P_{1}$, we have

\begin{equation*}
(R_{\psi}^{-1} \circ \tau_{\cbf}^{-1}  \circ h)((x,y))=\frac{\delta}{\delta'}\left(\frac{-\cos(\theta)i(\tau_{\xbf_v} ((x,y)))\cdot \xbf_{1} }{1-\frac{sin(\theta)}{\delta'}\xbf_{1}\cdot i(\tau_{\xbf_v}((x,y)))}, \frac{-i(\tau_{\xbf_v} ((x,y)))\cdot \ybf_{1}}{1-\frac{sin(\theta)}{\delta'}\xbf_{1}\cdot i(\tau_{\xbf_v}((x,y)))}  \right) 
\end{equation*}

Let $h_{\theta,\delta'}$ be defined by

\begin{equation*}
h_{\theta,\delta'}(x',y')=\left(\frac{-\cos(\theta)x'}{1-\frac{\sin(\theta)}{\delta'}x'} ,\frac{-y'}{1-\frac{\sin(\theta)}{\delta'}x'}\right)
\end{equation*}

Then

\begin{equation*}
(R_{\psi}^{-1} \circ \tau_{\cbf}^{-1} \circ h)((x,y))= \frac{\delta}{\delta'}h_{\theta,\delta'}\left ( i(\tau_{\xbf_v}((x,y))) \cdot \xbf_{1}, i(\tau_{\xbf_v}((x,y))) \cdot \ybf_{1}\right)
\end{equation*}
\label{figure_de_rotations_18}
Since $\xbf_{1}=\cos(\phi)\xbf_{0}+\sin(\phi)\ybf_{0}$ et $\ybf_{1}=-\sin(\phi)\xbf_{0}+\cos(\phi)\ybf_{0}$ (figure \ref{img_angles}), we have

\begin{eqnarray*}
(R_{\psi}^{-1} \circ \tau_{\cbf}^{-1} \circ h)((x,y)) &=& \frac{\delta}{\delta'}h_{\theta,\delta'}\left ( R_{\phi}(i(\tau_{\xbf_v}((x,y))) \cdot \xbf_{0}, i(\tau_{\xbf_v}((x,y))) \cdot \ybf_{0})\right)\\
                                               &=&\frac{\delta}{\delta'} (h_{\theta,\delta'}\circ R_{\phi} \circ \tau_{\xbf_v})((x,y))
\end{eqnarray*}

Denoting zooms $z_{\lambda}:X\rightarrow \lambda X$, the claim is verified since

\begin{equation*}
h = \tau_{\cbf} \circ R_{\psi} \circ z_{\frac{\delta}{\delta'}} \circ h_{\theta,\delta'} \circ R_{\phi} \circ \tau_{\xbf_v}
\end{equation*}

\end{proof}


\begin{remarques}
\begin{itemize}
\item Resampling an image by the homography $h$ simulate a change of the point of view on the input image. This has parameters $(\phi,\theta,\psi,\delta,\delta',\xbf_v,\cbf_v)$, which are not independent.
\item The case in which the camera is targeting the horizon has not been discussed, translations $\tau_\cbf$ allow to avoid this.
\end{itemize}
\end{remarques}



\begin{remarque}
The previous method does not modeled every affine transform. Indeed the function $h$ defined by 
\begin{equation*}
h = \tau_{\cbf}   \circ R_{\psi} \circ z_{\frac{\delta}{\delta'}} \circ h_{\theta,\delta'} \circ R_{\phi} \circ \tau_{\xbf_v}
\end{equation*}
is affine if and only if $\theta=0$. In this case we have
\begin{equation*}
h= \tau_{\cbf} \circ z_{-\frac{\delta}{\delta'}} \circ R_{\phi+\psi} \circ \tau_{\xbf_{v}}
=\tau' \circ z_{-\frac{\delta}{\delta'}} \circ  R_{\phi+\psi}
\end{equation*}
However the affinity can be seen as a limit case. Let $k$ be the ratio $\frac{\delta}{\delta'}$. If $\delta'$ and $\delta$ tense to $+\infty$, the function $h_{\theta,\delta'}$, the function $h_{\theta,\delta'}$ tense to $h_{\theta,\infty}$ defined by
\begin{equation*}
h_{\theta,\infty}=(x,y)=(-\cos(\theta)x,-y)
\end{equation*}
Physically, it is equivalent to moving away from the plane while increasing the focal distance in order to keep the size of the output image constant.

If $h_\infty = z_{-\frac{\delta}{\delta'}} \circ \tau_{\xbf} \circ R_{\phi} \circ h_{\theta,\infty} \circ R_{\psi} \circ \tau_{\xbf_{v}}$, the linear part $h_{\infty}$ can be represented by a matrix $2\times2$

\begin{equation*}
R_{\psi} \cdot 
\begin{pmatrix}
-k\cos(\theta)&0\\
0&-k
\end{pmatrix}
\cdot R_{\phi}
\end{equation*}

If $M$ is an invertible matrix $2\times 2$, then we have the following lemma using the singular value decomposition \cite{morel2009asift}.
\begin{lem}
There exists two rotation matrix $R_1$ and $R_2$ and a diagonal matrix $D$ such that $M = R_1 \cdot D \cdot R_2$.
\label{decomp_valeur_sing}
\end{lem}

Using the lemma (\ref{decomp_valeur_sing}) it can be deduced that for all bijective affinity $A$, there exists a camera movement $h$ such that $h_\infty = A$. Moreover it can be assumed that $h$ has no output translation.
\end{remarque}






\subsubsection{Application of the decomposition to homographies}
The previous results show that all homographies can be decomposed using the following Theorem

%Le résultat précédent montre que certaine homographie peuvent se décomposer de la
\begin{thm}
Let $h$ be an homography, if $h$ is not an affine map then there exists parameters $(\phi,\theta,\psi,\delta,\delta',(x_1,y_1),(x_2,y_2))$ such that
\begin{equation*}
h = \tau_{(x_2,y_2)} \circ R_{\psi} \circ z_{\frac{\delta}{\delta'}} \circ h_{\theta,\delta'} \circ R_{\phi} \circ \tau_{(-x_1,-y_1)}
\end{equation*}
This decomposition is not unique. More precisely for all $\lambda \in ]0,1[$,

  \begin{equation*}
h = \tau_{(x_2,y_2)} \circ z_{\frac{\delta}{\delta'}}  \circ R_{\psi} \circ h_{\theta,\delta'} \circ R_{\phi} \circ \tau_{(-x_1,-y_1)}
  \end{equation*}
  where 
 \begin{equation*}
x_2=\frac{ar+sb+\hat r \lambda}{r^2 +s^2}, y_2=\frac{cr+sd+\hat s \lambda}{r^2 +s^2}, (x_1 , y_1) = h^{-1}(x_{2},y_{2})
  \end{equation*}
 \begin{equation*}
 \cos( \phi )= - \frac{r}{\sqrt{r^2 + s^2}}, \sin( \phi )= - \frac{s}{\sqrt{r^2 + s^2}},\cos( \psi ) =- \frac{\hat r}{\sqrt{\hat r^2 + \hat s^2}}, \sin( \psi ) = \frac{\hat s}{\sqrt{\hat r^2 + \hat s^2}}
 \end{equation*}
 \begin{equation*}
 \frac{\delta}{\delta'}=|\lambda|\sqrt{\frac{\hat r^2 + \hat s^2}{r^2 + s^2}}^{3}, \cos(\theta)=\lambda, \sin(\theta)=\sqrt{1-\lambda^2}, \delta'=  \frac{\sqrt{(r^2 + s^2)(1-\lambda^2)}}{|\lambda| (\hat r^2+\hat s^2)}
 \end{equation*}
\label{thepropdecomp}
\end{thm}

\begin{corollaire} If $h$ an homography and $h$ is not affine, then there exists a translation $\tau$, two rotations $R_\phi ,R_\psi$ and a unidirectional homography $\tilde{h}$ such that
\begin{equation}
h=\tau \circ R_\psi \circ \tilde{h} \circ R_\phi
\label{formule_decomposition_effective}
\end{equation}
and this decomposition is not unique.
\end{corollaire}

		This formula is used in the algorithm \ref{pseudoCodeDecompo}.
		
		\begin{proof}
	 Using theorem (\ref{thepropdecomp}),there exists $(\phi,\theta,\psi,\delta,\delta',\xbf_v,\cbf)$ such that 
	 \begin{equation*}
	 h = \tau_{\cbf} \circ R_{\psi} \circ z_{\frac{\delta}{\delta'}} \circ h_{\theta,\delta'} \circ R_{\phi} \circ \tau_{\xbf_v}
	 \end{equation*}
	 $R_\psi$ and $z_{\frac{\delta}{\delta'}}$ commute. Let $\tau'$ be the theorem such that $\tau' \circ R_\phi =  R_\phi \circ \tau_{\xbf_v}$.\\
	 Then with $\tilde{h} = z_{\frac{\delta}{\delta'}} \circ 
	 h_{\theta,\delta'} \circ \tau'$ it can be easily verified that $\tilde{h}$ is indeed a unidirectional homography.
	 \end{proof}
	\label{ref_schema_decomp_cool}
	\begin{figure}
		\centering
		\subfigure[Input image]{
		\centering
		{\includegraphics[scale=0.24]{vue_fps_identity.png}}
		{\includegraphics[scale=0.35]{vue_tps_identity.png}}}
		\subfigure[After a first rotation (of angle $\phi$)]{
		\centering
		{\includegraphics[scale=0.24]{vue_fps_rotation_phi.png}}
		{\includegraphics[scale=0.35]{vue_tps_rotation_phi.png}}}
		\subfigure[After the unidirectional homography]{
		\centering
		{\includegraphics[scale=0.24]{vue_fps_hom_part.png}}
		{\includegraphics[scale=0.35]{vue_tps_hom_part.png}}}
		\subfigure[Output image (after the rotation of angle  $\psi$)]{
		\centering
		{\includegraphics[scale=0.24]{vue_fps_rotation_psi.png}}
		{\includegraphics[scale=0.35]{vue_tps_rotation_psi.png}}}
		\caption{Steps to process an homography, represented as camera moves, on the left the view from the camera, on the left a motionless point of view. The translations are not represented to simplify the situation. On motionless views, $F$ is the focus of the camera, the red plane is the image plane of the camera. (cf. \ref{ref_schema_decomp_cool})}
		\label{schema_decomp_cool}
		\label{SchemaEtapesDecompoGeometrique}
	\end{figure}
	\clearpage

		\subsection{Traitement des homographies particulières}
			% contient la description des différentes possibilités pour traiter l'homographie particulière au centre de la décomposition
%simon
\subsubsection{Séparation d'une homographie particulière }

On considère une homographie $h$ de la forme 
\begin{equation*}
h:(x,y)\mapsto \left(\frac{-bx}{1-ax},\frac{-y}{1-ax}\right)
\end{equation*}
On peut décomposer cette homographie en deux applications $h_1 , h_2$
\begin{equation*}
h_1:(x,y) \mapsto \left(\frac{-bx}{1-ax}    ,y\right)~~~~~~h_2:(x,y) \mapsto \left(x,\frac{-y}{1-ax}\right)
\end{equation*}
On obtient $h=h_1  \circ h_2$, ce qui nous donne le schéma suivant 
\begin{equation*}
f\longrightarrow f'=f\circ h_1 \longrightarrow f''=f'\circ h_2
\end{equation*}
Chacune de ces deux transformations ne modifie l'image que dans une seule direction, cela permet d'effectuer des opérations sur des signaux unidimentionnels . Cette méthode n'est en revanche pas séparable.\\ 
La première transformation est une homographie en une dimension que l'on doit réaliser sur chaque ligne.\\ %sur ?
La seconde est un zoom d'un facteur différent sur chaque colonne.

\paragraph{Différentes méthodes de ré-échantillonage naif :}
%peut etre inutile
Le but ici est de présenter plusieurs méthodes permettant réaliser l'homographie par séparation.

\subparagraph{Méthode du point le plus proche :}
Il s'agit de la méthode la plus basique pour effectuer un ré-échantillonnage. On interpole l'image  par une fonction constante par morceau et on réévalue l'image $F\circ h $. Cette méthode est séparable, c'est-à-dire qu'il revient d'effectuer l'homographie $h$ et les deux applications $h_1 , h_2$ successivement.

\subparagraph{Méthode d'interpolation bilinéaire}
Cette méthode permet de réduire l'\emph{aliasing} par rapport à la méthode du point le plus proche, on interpole l'image en utilisant une fonction affine par morceaux $F$ et on évalue ensuite $F\circ h$. Cette méthode est aussi séparable.

%a étoffer
\paragraph{Sous échantillonnage gaussien :}
Dans le cas d'un zoom gaussien on utilise la convolution $f*G_{d}$. Le paramètre $d$ doit être choisi tel que 
\begin{equation*}
z^2 c^2=c^2 + d^2     ~~~~~~~c= 0.8
\end{equation*}
On obtient donc la formule $d=c\sqrt{z^2 - 1}$, nous renvoyons à l'article \cite{morel2011sift} pour les détails de ce raisonnement et une justification de la valeur expérimentale de c.\\

\paragraph{Sous échantillonnage utilisant les images intégrale }
Soit $d>0$, on veut convoler une fonction d'une variable par une approximation d'une gaussienne d'écart type $\delta$.\\
Soit $g_n^d$ le noyau défini par 
\begin{equation*}
g_1^d(t)=\frac{1}{d}\mathds{1}_{]-\frac{d}{2},\frac{d}{2}[}(t) ~~~~~~~g_{n+1}^d= g_n^d * g_1^d
\end{equation*}
Si on pose $G_n^d(x)=\sqrt{n}g_n^d(\sqrt{n} x)$ on peut montrer que 
\begin{equation*}
\widehat{G_n^d}(\omega)\underset{n\rightarrow\infty}{\rightarrow} \exp\left(-\frac{\omega^2 d^2}{12}\right)
\end{equation*}
\begin{proof}
On a $\widehat{g_1^d}(\omega)=\text{sinc}\left(\frac{\omega d}{2}\right)~~$  donc $~~\widehat{g_n^d}(\omega)=\text{sinc}\left(\frac{\omega d}{2}\right)^n~~$ et 
$~~\widehat{G_n^d}(\omega)=\text{sinc}\left(\frac{\omega d}{2\sqrt{n}}\right)^n~~$\\
on obtient le résultat voulu en réalisant un développement limité de cette fonction 
\end{proof}
Si $f,g$ sont deux fonctions continues à support compact et $d>0$ on a le résultat suivant
\begin{equation*}
(f*(g*g_1^d))(y)=\frac{(F^{(1)}*g)(y+\frac{d}{2})-(F^{(1)}*g)(y-\frac{d}{2})}{d}~~~~~~F^{(1)}(x)=\int_{-\infty}^{x}f(y) dy
\end{equation*}
\begin{proof}
Par intégration par partie (au sens des distributions), comme les fonctions $f$ et $g$ sont à support compact on obtient  :
\begin{eqnarray*}
(f * g *g_1^d )(y)&=&\frac{1}{d} \int_{\mathbb{R}} f(x)~(g * \mathds{1}_{]-\frac{d}{2},\frac{d}{2}[})(y-x) dx\\
                 &=& -\frac{1}{d} \int_{\mathbb{R}} F^{(1)}(x)~(g * (-\delta_{\frac{d}{2}}+\delta_{-\frac{d}{2}}))(x-y) dx\\
                 &=& \frac{1}{d} \int_{\mathbb{R}} F^{(1)}(x)~g(x-y-\frac{d}{2} )dx -\frac{1}{d} \int_{\mathbb{R}} F^{(1)}(x)~g(x-y+\frac{d}{2})dx\\
                 &=& \frac{(F^{(1)} * g)(y+\frac{d}{2})-(F^{(1)} * g )(y-\frac{d}{2})}{d}
\end{eqnarray*}
\end{proof}
On peut poser $D_d$ l'opérateur de "dérivation discrète" définit par $D_d f=f*\frac{f(.+\frac{d}{2})-f(.-\frac{d}{2})}{d}$ et réécrire la formule précédente 
\begin{equation*}
f * g *g_1^d =D_d (F^{(1)}*g)
\end{equation*}
On en déduit par récurrence la formule 
\begin{equation*}
(f*g_n^d)(y)=D_d ^n F^{(n)}= \frac{1}{d^n}\underset{0 \le k\le n}{\sum} \binom{n}{k}(-1)^{k} F^{(n)}(y+\frac{(n-2k)d}{2})
\end{equation*}
La fonction $F^{(k)}$ est la somme de la fonction  $F^{(k-1)}$, la seconde égalité se démontre en développant l'opérateur $D_d ^n$ par la formule du binome de Newton.\\
Dans nos algorithmes  on utilisera cette formule pour $n=3$ :
\begin{equation*}
(f*g_3)(y)=\frac{1}{d^3}(F^{(3)}(y+\frac{3d}{2})-3F^{(3)}(y+\frac{d}{2})+3F^{(3)}(y-\frac{d}{2})-F^{(3)}(y-\frac{3d}{2}))
\end{equation*}
On doit cependant calculer une valeur approchée des fonctions $F^{(k)}$  car on ne connait que les échantillons de la fonctions $f$.\\
On peut utiliser la méthode suivante on pose par récurrence
\begin{equation*}
F^{(n+1)}(k)=\underset{l\le k-1}{\sum}F^{(n)}(l)
\end{equation*}
On utilise ensuite une méthode d'interpolation par splines cubiques afin de pouvoir évaluer cette fonction en des valeurs non entière.\\
Une autre méthode consiste à poser 

\begin{equation*}
F^{(0)} (x) =\underset{0\le k \le m-1}{\sum}f_{k} \mathds{1}_{[k,k+1[}(x)
\end{equation*}
$(f_k)_{k=0...m-1}$ sont les termes du signal .On calcule ensuite $F^{(n)}(x)=\int_{-\infty}^{x}F^{(n-1)}(y)dx$ on à par exemple
\begin{equation*}
F^{(1)}(x)=\underset{k\le \lfloor x\rfloor \wedge m~-1}{\sum}f_{k}~~+ f_{\lfloor x\rfloor}
(x-\lfloor x\rfloor)
\end{equation*}
Cette fonction est affine par morceau on peut démontrer la formule suivante par récurrence
\begin{eqnarray*}
F^{(n)}(x) &=& F^{(n)}(\lfloor x\rfloor \wedge m)~~+\mathds{1}_{[0,m[}(x) \underset{0\le k \le n-1}{\sum}F^{(k)}(\lfloor x \rfloor) \frac{(x-\lfloor x \rfloor)^{n-k}}{(n-k)!}\\
          &+&\mathds{1}_{[m,+\infty[}(x)\underset{1\le k \le n-1}{\sum}F^{(k)}(m) \frac{(x-m)^{n-1-k}}{(n-1-k)!}
\end{eqnarray*}
Où la valeur de $F^{(n)}$ se calcule par récurrence on a la relation $\forall k\le m$
\begin{equation*}
F^{(n)}(k+1)=F^{(n)}(k)+\underset{0\le l < n}{\sum} \frac{F^{(l)}(k)}{(n-l)!}
\end{equation*}
Comme dans la méthode précédente on doit calculer une composante constante par morceaux afin d'avoir la valeur de $F^{(n)}$ aux entiers  ainsi qu'un terme polynomial de degré $n$ pour effectuer l'interpolation sur des valeurs non-entières.\\
Dans le cas $n=3$ on obtient les formules
\begin{eqnarray*}
F^{(3)}(x)&=&F^{(3)}(\lfloor x\rfloor \wedge m)~~+\mathds{1}_{[0,m[}(x)(x-\lfloor x \rfloor) \left(F^{(2)}(\lfloor x \rfloor)+ \frac{(x-\lfloor x \rfloor)}{2}\left(F^{(1)}(\lfloor x \rfloor)+\frac{(x-\lfloor x \rfloor)}{3} f_{\lfloor x \rfloor}\right)\right)\\
          &+&\mathds{1}_{[m,+\infty[}(x)(x-\lfloor x \rfloor) \left(F^{(2)}(m)+ \frac{(x-\lfloor x \rfloor)}{2}F^{(1)}(m)\right) \\
F^{(3)}(k)&=&  \underset{0\le l<k}{\sum}\left(F^{(2)}(l)+\frac{F^{(1)}(l)}{2}+\frac{f_{l}}{6} \right)  \\
F^{(2)}(k)&=&  \underset{0\le l<k}{\sum}\left(F^{(1)}(l)+\frac{f_{l}}{2} \right)  \\
F^{(1)}(k)&=&  \underset{0\le l<k}{\sum}f_{l} 
\end{eqnarray*}
Si on fixe $0\le l\le m-1$ et on pose
\begin{equation*}
P_l (x)=F^{3}(l) +(x-l) \left(F^{(2)}(l)+ \frac{(x-l)}{2}\left(F^{(1)}(l)+\frac{(x-l)}{3} f_{l}\right)\right)
\end{equation*}
On obtient alors
\begin{eqnarray*}
P_l (l) &=& F^{(3)}(l) \\
P_l (l+1) &=& F^{(3)}(l+1) \\
P_l '(l) &=& F^{(2)}(l) \\
P_l '(l+1) &=& F^{(2)}(l+1)
\end{eqnarray*}
On peut donc obtenir ces formules en utilisant une méthode d'interpolation de Hermite. La fonction $F^{(3)}$ est la spline de cubique passant par le point $F^{(3)}(k)$ en $k$ avec une dérivé égale à $F^{(2)}(k)$, elle est $\mathcal{C}^2$.\\
Cette méthode a cependant un défaut, l'image est initialement interpolé par une fonction constante par morceau, on a donc utiliser une interpolation linéaire du signal de départ 
\begin{equation*}
F_1 ^{(0)}(x)=\underset{0\le k \le m-1}{\sum} f_k g_2^1 (x-k-\frac{1}{2})=(F_0^{(0)} *g_1^1 )(x)
\end{equation*}
On obtient alors 
\begin{equation*}
F_1 ^{(0)}*g_n^d=(F_0 ^{(0)}*g_1^1)*g_n^d=(F_0 ^{(0)}*g_n^d)*g_1^1=(D_d^n F_0 ^{(n)})*g_1^1=D_1 D_d^n F_0^{(n+1)}
\end{equation*}
L'interpolation supplémentaire peut donc être obtenue en évaluant $F_0^{(n+1)}$. La méthode de calcul est donc la même que dans le cas précédent.\\
Il est possible d'utiliser cette méthode pour obtenir une représentation $F_n^{(0)}$ plus régulière du signal de départ mais la courbe n'est pas  interpolante si $2\le n$.\\
%à faire mieux 
Afin d'implémenter cette méthode nous devons déterminer la valeur du paramètre $d$ en fonction du facteur de zoom local. Nous allons réutiliser les résultats du paragraphe précédent, car la fonction $g_3$ est une bonne approximation d'une gaussienne. L'écart type de $g_1$ est $\sigma_1=\frac{d}{\sqrt{12}}$ donc l'écart type $\sigma_3$ est donné par la formule $\sigma_3=\sqrt{3}\sigma_1=\frac{d}{2}$
donc $d=2c\sqrt{z^2 - 1}$.\\
Dans la pratique on utilisera la formule $d=2\sqrt{(0.8)^2 z^2 - (0.7)^2}$ car les images utilisées ne sont en général pas parfaites.\\

\paragraph{Sur-échantillonage par splines :}
%Pas encore fait
\subparagraph{Splines cubiques de Hermite :}
Si $(x_0,x_1,x_2,x_3)$ et $(y_0,y_1,y_2,y_3)$  permet de construire un polynôme $P$ de degrès 3 tel que  
\begin{equation*}
P(x_1)=y_1~~~~P(x_2)=y_2~~~~P'(x_1)= \frac{y_2-y_0}{x_2 -x_0}~~~~P'(x_2)= \frac{y_3-y_1}{x_3 -x_1}
\end{equation*}

		\subsection{Traitement des rotations}
			% contient la description des différentes méthodes pour traiter les rotations de la décomposition

Pour les deux rotations dans la décomposition, nous avons testés plusieures méthodes : la méthodes de Yaroslavsky \cite{unser1995convolution} et la méthode de traitement des affinités multiétapes \cite{szeliski2010high}.

\subsubsection{Méthode de Yaroslavsky}

La méthode de Yaroslavsky consiste à décomposer une rotation en trois "shear".//

Sous forme matricielle un rotation s'écrit de manière générale :
\begin{equation*}
	H=\begin{pmatrix}
	cos \theta&-sin \theta\\sin \theta&cos \theta
	\end{pmatrix}
	\end{equation*}

La décomposition en trois shear est la suivante :
\begin{equation*}
	H=\begin{pmatrix}
	cos \theta&-sin \theta\\sin \theta&cos \theta
	\end{pmatrix}=\begin{pmatrix}
	1&-tan \frac{\theta}{2}\\0&1
	\end{pmatrix}\begin{pmatrix}
	1&0\\sin \theta&1
	\end{pmatrix}\begin{pmatrix}
	1&-tan \frac{\theta}{2}\\0&1
	\end{pmatrix}
	\end{equation*}

	Pour faire la rotation complète on fait chaque shear par fourier et on recentre l'image après chaque shear.\\
	En pratique on se ramène toujours à des rotations d'angle entre $[\frac{-\pi}{4},\frac{\pi}{4}]$ (on peut en effet faire des rotations d'angle $\pi$ et $\frac{\pi}{2}$ de manière exacte...)  

	\section{Expériences}
		% contient les différentes comparaisons entre les différentes méthodes

	\appendix
	\section{Algorithme}
		% contient tous les pseudo-codes, qui seront en annexe

\end{document}
