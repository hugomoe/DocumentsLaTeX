PARTOUT
	référer les figure
	les reference peuvent disparaitre sans gêner la phrase
	mettre des captions et des labels sur toutes les figures
	remplacer tous les "homographie particulière" par "homographie unidirectionnelle"
	mettre des phrases dans les propositions et redéfinir les termes
	proposition de shmuel : mettre l'échelle dans les captions (pour connaitre la taille réelle)
	article mathématique : pas de texte libre (disons presque pas...?)
		soit : notation, remarque
	mettre le moins de texte libre possible
	la description d'un algo, si elle est courte, doit être dans le \caption



INTRO
motiver le problème (simule camera (jv) + fusion d’image + le pb n’est pas trivial (combine zoom avant/arrière))
analyse biblio
	impossibilité d’interplation 
	zoom arrière zero-padding
	Ripmap/Mipmap
->revendication de methode
	toute homo
	decomp géométrique
	un peu plus lente
seulement ici mettre le plan (l'intro qu'on avait déjà)
l'intro fait une page au moins





HOMOGRAPHIE
	va dans l’intro 5 ligne
	c’est un mvt d’une caméra idéale

METHODE NAIVE
	mentionne dans l’introduction 

MIPMAP
	indiquer quelle distance on a choisi
	dans les images distance : mettre label.
	conclusion : la mettre de manière informelle.
	nous avons essayés et nous avons trouvé...

AFFINITE
	formule convolution discrète avec zoom (reference pseudo-code supp adaptatif)
	graphique avec le cosine weighted (éventuellement avec différents beta)
	reference pseudo-code / figure (e.g. umax/vmax, transpoopt, dcompo affinite)


DECOMP GEOM 
	-mettre convergence en espace ? (L2 et uniformément).
	-page 26 : dire n=4


MVT DE CAMERA
	proposition de shmuel : décrire rapidement ce qu'est une caméra parfaite (projection) notamment via fig11
	mettre lemme plutôt que « finalement on en déduit »
	mettre des props / Lemme
		tous les lemmes et propositions doivent rappeler leurs notations
		ds le lemme : "sous les notations précédentes..."
	Prop 2 = revoir les notations (h ‘est’ un mvet de cam -> h une appli de R^2, h s'identifie à)

DECOMP HOMO
	include \begin{proof} \end{proof}

HOMO PARTICULIERE (-> HOMO UNIDIRECTIONNELLE)
	Rentrer « on en déduit » de fin de partie precedente, on le met là
	inclure image qui montre la séparation (damier par exemple), chacun en partant de l'image originale
	Lemme pour la convergence des convolées
	"la spline de cubique", "si on pose F1 est la spline"

TRAITEMENT DES ROTATIONS
	dans les expériences, mettre la différence à l'originale (et pas l'image)
	rms, l1 etc : begin{tab} end{tab} et mettre en gras les meilleurs (ou tabular ?)

	

EXPERIENCES
	utiliser le même rescale pour les différents Fourier d'une même série d'expérience
	mettre en comparaison la méthode naïve

CONCLUSION
	on propose une solution optimale dans le sens suivant. mipmap < decomposition pour des résultats super exacts.
	ouverture dans la conclusion : on peut mettre surement plus rapide pour les rotations sans perdre trop de qualité.
