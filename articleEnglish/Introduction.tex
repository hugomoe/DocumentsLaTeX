% contient l'introduction

An homographies is a projective mapping on a two dimensional space which is a good model of camera motion and point of view's changes. Therefore they are used a lot in image processing such as in texturing \cite{heckbert1983texture} (for instance in video games, animation...), for panoramic image stitching \cite{brown2007automatic}. It is not easy to resample an image by an homography because it involves differents zoom-in or zoom-out for each pixel ; and aliasing and overblur have to be avoid. Because of the need of zoom-out, splines can't be used for the resampling. Currently, homographies are resampled by the Mipmap method \cite{williams1983pyramidal} which is described in the appendix \ref{Mipmap_pseudo_code_jt} or by variants of this method. Sometimes it is combined with multiple-sample anisotropic filtering  \cite{barkans1997high}. The use of the Mipmap makes it possible to do zoom-out ; however it sometimes produces a lot of aliasing. Others filters such as the Elliptic Weighted Average (EWA) ones directly convolve the image with a gaussian filter \cite{greene1986creating}. Unfortunately gaussian filtering is known to create aliasing and to overblur. Besides this filter is isotropic in the warped coordinates, and it incorrectly filters out corner frequencies in the spectrum. In addition the use of the EWA filter takes a lot of time, even if faster algorithmes based on Mipmapping have been developped \cite{mccormack1999feline,huttner1999fast}. The new method of resempling presented in this article is compared (in the experiments part) to an anisotropic variant of the Mipmap : the Ripmap \cite{akenine2008real} which is described in the appendix\ref{Ripmap_pseudo_code_jt}.

	An efficient methode to resample affine transforms, which are a particular case of homographies, has been recently developed \cite{szeliski2010high}. This paper presents a new method which makes it possible to resample any homographies thanks to a geometric-based spliting and using that efficient method of affine transforms resampling. Compared to the Ripmap, that new method reduces blur and aliasing being however a little slower. 
	
	Part \label{szeliski_section} presents the method of affine transforms resampling, which is used is the new geometric-based decomposition.
	
	Part \ref{decomp_geo_hom} describes the new geometric-based decomposition of homographies and how to resample the special homography coming from that decomposition.
	
	Part \ref{experiences} by comparing that new resampling method to pre-existing ones shows that the first one reduces blur and aliasing.
	
	
	
	



	La partie \label{szeliski_section} présente une méthode de traitement des affinités qui est utilisée dans la méthode par décomposition.

	La partie \ref{decomp_geo_hom} présente une interprétation géométrique d'une homographie en terme de mouvement de caméra. Elle permet de comprendre la théorie qui justifie cette nouvelle méthode. Elle explique aussi comment décomposer une homographie à partir de cette interprétation.

	La partie \ref{experiences} montre les performances et le gain en qualité de la méthode de traitement des homographies par décomposition par rapport aux méthodes existantes (notamment le Ripmap).

	Les pseudo-codes permettant de mettre en oeuvre cette nouvelle méthode, ainsi qu'une description du Mipmap et du Ripmap sont présents en annexe.
